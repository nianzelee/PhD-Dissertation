\documentclass[12pt,a4paper]{dmathesis}

%% The following is to control the format of your thesis
%%%%%%%%%%%%%%%%%%%%%%%%%%%%%%%%%%%%%%%%%%%%%%%%%%%%%%%%%%%%%%%%%%%%%%%%%%
%     This is format.tex file needed for the dmathesis.cls file.  You    %
%  have to  put this file in the same directory with your thesis files.  %
%                Written by M. Imran 2001/06/18                          %
%                 No Copyright for this file                             %
%                 Save your time and enjoy it                            %
%                                                                        %
%%%%%%%%%%%%%%%%%%%%%%%%%%%%%%%%%%%%%%%%%%%%%%%%%%%%%%%%%%%%%%%%%%%%%%%%%%
%%%%%  Put packages you want to use here and 'fancyhdr' is required   %%%%
%%%%%%%%%%%%%%%%%%%%%%%%%%%%%%%%%%%%%%%%%%%%%%%%%%%%%%%%%%%%%%%%%%%%%%%%%%
\usepackage{fancyhdr}
\usepackage{epsfig}
\usepackage{cite}
\usepackage{graphicx}
\usepackage{amsmath}
\usepackage{amsthm}
\usepackage{amssymb}
\usepackage{latexsym}
\usepackage{epic}
\usepackage{algorithm}
\usepackage{algorithmic}
\usepackage{CJK}
\usepackage{xwatermark}
\usepackage{tikz}
\usepackage{hyperref}
\usepackage[capitalise,nosort,nameinlink]{cleveref}
\usepackage[nottoc]{tocbibind}
%%%%%%%%%%%%%%%%%%%%%%%%%%%%%%%%%%%%%%%%%%%%%%%%%%%%%%%%%%%%%%%%%%%%%%%%%%
%%%%%                 Set line spacing = 1.5 here                   %%%%%%
%%%%%%%%%%%%%%%%%%%%%%%%%%%%%%%%%%%%%%%%%%%%%%%%%%%%%%%%%%%%%%%%%%%%%%%%%%
\renewcommand{\baselinestretch}{2.0}
%%%%%%%%%%%%%%%%%%%%%%%%%%%%%%%%%%%%%%%%%%%%%%%%%%%%%%%%%%%%%%%%%%%%%%%%%%
%%%%%                      Your fancy heading                       %%%%%%
%%%%% For the final copy you need to remove '\bfseries\today' below %%%%%%
%%%%%%%%%%%%%%%%%%%%%%%%%%%%%%%%%%%%%%%%%%%%%%%%%%%%%%%%%%%%%%%%%%%%%%%%%%
\pagestyle{fancy}
\renewcommand{\chaptermark}[1]{\markright{\chaptername\ \thechapter.\ #1}}
\renewcommand{\sectionmark}[1]{\markright{\thesection.\ #1}{}}
\lhead[\fancyplain{}{}]%
{\fancyplain{}{\bfseries\rightmark}}
\chead[\fancyplain{}{}]%
{\fancyplain{}{}}
\rhead[\fancyplain{}{}]%
{\fancyplain{}{\bfseries}}
\lfoot[\fancyplain{}{}]%
{\fancyplain{}{}}
\cfoot[\fancyplain{}{}]%
{\fancyplain{}{\thepage}}
\rfoot[\fancyplain{}{}]%
{\fancyplain{}{}}%{\bfseries\today}}
%%%%%%%%%%%%%%%%%%%%%%%%%%%%%%%%%%%%%%%%%%%%%%%%%%%%%%%%%%%%%%%%%%%%%%%%%%
%%%%%%%%%%%%Here you set the space between the main text%%%%%%%%%%%%%%%%%%
%%%%%%%%%%%%%%%%%%%and the start of the footnote%%%%%%%%%%%%%%%%%%%%%%%%%%
%%%%%%%%%%%%%%%%%%%%%%%%%%%%%%%%%%%%%%%%%%%%%%%%%%%%%%%%%%%%%%%%%%%%%%%%%%
\addtolength{\skip\footins}{5mm}
\addtolength{\footskip}{5mm}
\addtolength{\headsep}{10mm}
\addtolength{\textheight}{-15mm}
%%%%%%%%%%%%%%%%%%%%%%%%%%%%%%%%%%%%%%%%%%%%%%%%%%%%%%%%%%%%%%%%%%%%%%%%%%
%%%%%      Define new counter so you can have the equation           %%%%%
%%%%%    number 4.2.1a for example, this a gift from J.F.Blowey      %%%%%
%%%%%%%%%%%%%%%%%%%%%%%%%%%%%%%%%%%%%%%%%%%%%%%%%%%%%%%%%%%%%%%%%%%%%%%%%%
\newcounter{ind}
\def\eqlabon{
  \setcounter{ind}{\value{equation}}\addtocounter{ind}{1}
  \setcounter{equation}{0}
  \renewcommand{\theequation}{\arabic{chapter}%
    .\arabic{section}.\arabic{ind}\alph{equation}}}
\def\eqlaboff{
  \renewcommand{\theequation}{\arabic{chapter}%
    .\arabic{section}.\arabic{equation}}
  \setcounter{equation}{\value{ind}}}
%%%%%%%%%%%%%%%%%%%%%%%%%%%%%%%%%%%%%%%%%%%%%%%%%%%%%%%%%%%%%%%%%%%%%%%%%%
%%%%%%%%%%%%           New theorem you want to use              %%%%%%%%%%
%%%%%%%%%%%%%%%%%%%%%%%%%%%%%%%%%%%%%%%%%%%%%%%%%%%%%%%%%%%%%%%%%%%%%%%%%%
\newtheorem{definition}{Definition}[chapter]
\newtheorem{proposition}{Proposition}[chapter]
\newtheorem{example}{Example}[chapter]
\newtheorem{theorem}{Theorem}[chapter]
\newtheorem{corollary}{Corollary}[theorem]
\newtheorem{lemma}[theorem]{Lemma}
%%%%%%%%%%%%%%%%%%%%%%%%%%%%%%%%%%%%%%%%%%%%%%%%%%%%%%%%%%%%%%%%%%%%%%%%%%
%%%%%%%    Bold font in math mode, a gift from J.F.Blowey       %%%%%%%%%%
%%%%%%%%%%%%%%%%%%%%%%%%%%%%%%%%%%%%%%%%%%%%%%%%%%%%%%%%%%%%%%%%%%%%%%%%%%
\def\bv#1{\mbox{\boldmath$#1$}}
%%%%%%%%%%%%%%%%%%%%%%%%%%%%%%%%%%%%%%%%%%%%%%%%%%%%%%%%%%%%%%%%%%%%%%%%%%
%%%%%%%        New symbol which is not defined in Latex         %%%%%%%%%%
%%%%%%%                 a gift from J.F.Blowey                  %%%%%%%%%%
%%%%%%%%%%%%%%%%%%%%%%%%%%%%%%%%%%%%%%%%%%%%%%%%%%%%%%%%%%%%%%%%%%%%%%%%%%
% The Mean INTegral
% to be used in displaystyle
\def\mint{\textstyle\mints\displaystyle}
% to be used in textstyle
\def\mints{\int\!\!\!\!\!\!{\rm-}\ }
%
% The Mean SUM
% to be used in displaystyle
\def\msum{\textstyle\msums\displaystyle}
% to be used in textstyle
\def\msums{\sum\!\!\!\!\!\!\!{\rm-}\ }
%%%%%%%%%%%%%%%%%%%%%%%%%%%%%%%%%%%%%%%%%%%%%%%%%%%%%%%%%%%%%%%%%%%%%%%%%%
%%%%%%%%%%            Define your short cut here              %%%%%%%%%%%%
%%%%%%%%%%%%%%%%%%%%%%%%%%%%%%%%%%%%%%%%%%%%%%%%%%%%%%%%%%%%%%%%%%%%%%%%%%
\def\poincare{Poincar\'e }
\def\holder{H\"older }

%%%%%%%%%%%%%%%%%%%%%%%%%%%%%%%%%%%%%%%%%%%%%%%%%%%%%%%%%%%%%%%%%%%%%%%%%%
%%%%%%%%%%   Added by Flotisable base on senior's template    %%%%%%%%%%%%
%%%%%%%%%%%%%%%%%%%%%%%%%%%%%%%%%%%%%%%%%%%%%%%%%%%%%%%%%%%%%%%%%%%%%%%%%%
\makeatletter
\newcommand{\cjk}[1]{\begin{CJK}{UTF8}{bkai}#1\end{CJK}}

% defaults
\newcommand{\title@zh}{}
\newcommand{\title@en}{}
\newcommand{\student@zh}{}
\newcommand{\student@en}{}
\newcommand{\advisor@zh}{}
\newcommand{\advisor@en}{}
\newcommand{\school@zh}{\cjk{國立臺灣大學}}
\newcommand{\school@en}{National Taiwan University}
\newcommand{\institute@zh}{\cjk{電子工程學研究所}}
\newcommand{\institute@en}{Graduate Institute of Electronics Engineering}
\newcommand{\time@en}{June 2021}
\newcommand{\watermark@file}{fig/watermark.pdf}
\newcommand{\doi}{DOI}
\newcommand{\ackname@toc}{Acknowledgements}
\newcommand{\ackname@title}{Acknowledgements}
\newcommand{\abstractzhname}{\cjk{摘要}}
% end defaults

% name settings commands
\newcommand{\settitle}[2]
{
  \renewcommand{\title@zh}{\cjk{#1}}
  \renewcommand{\title@en}{#2}
}
\newcommand{\setstudent}[2]
{
  \renewcommand{\student@zh}{\cjk{#1}}
  \renewcommand{\student@en}{#2}
}
\newcommand{\setadvisor}[2]
{
  \renewcommand{\advisor@zh}{\cjk{#1}}
  \renewcommand{\advisor@en}{#2}
}
\newcommand{\setschool}[2]
{
  \renewcommand{\school@zh}{\cjk{#1}}
  \renewcommand{\school@en}{#2}
}
\newcommand{\setinstitute}[2]
{
  \renewcommand{\institute@zh}{\cjk{#1}}
  \renewcommand{\institute@en}{#2}
}
\newcommand{\setwatermarkfile}[1]
{
  \renewcommand{\watermark@file}{#1}
}
\newcommand{\setdoi}[1]
{
  \renewcommand{\doi}{#1}
}
\newcommand{\ackname}[2][Acknowledgements]
{
  \renewcommand{\ackname@toc}{#1}
  \renewcommand{\ackname@title}{#2}
}
% end name settings commands

% watermark and doi
\newlength{\watermark@x@shift}
\newlength{\watermark@y@shift}
\newlength{\doi@width}
\setlength{\watermark@x@shift}{26mm}
\setlength{\watermark@y@shift}{-24mm}
\newcommand{\addwatermark}
{
  \settowidth{\doi@width}{\doi}
  \newwatermark
  [
    scale=0.5,
    xpos=\watermark@x@shift+0.5\paperwidth-73mm/2-2.5cm,
    ypos=\watermark@y@shift+0.5\paperheight-73mm/2-2.5cm,
    allpages
  ]
  {\tikz[opacity=0.5]{\node{\includegraphics{\watermark@file}}}}
  \newwatermark
  [
    fontsize=12pt,
    fontseries=m,
    color=black,
    xpos=\watermark@x@shift+0.5\paperwidth-0.5\doi@width-1cm,
    ypos=\watermark@y@shift-0.5\paperheight+6pt+1cm,
    allpages
  ]{\doi}
}
% end watermark and doi

% frontmatter environments
\newenvironment{acknowledgements}
{
  \addcontentsline{toc}{chapter}{\numberline{}\ackname@toc}

  \begin{center}
    {\Large\bf \ackname@title}
  \end{center}

  \begin{CJK}{UTF8}{bkai}
    }
    {
  \end{CJK}

  \begin{flushright}
    \student@zh{}\\
    \student@en{}\\
  \end{flushright}
  \emph{\school@en}\\
  \emph{\time@en}

  \newpage
}

\newcommand{\abstract@zh@keywords}{}
\newenvironment{abstractzh}[1]
{
  \addcontentsline{toc}{chapter}{\numberline{}Chinese Abstract}
  \renewcommand{\abstract@zh@keywords}{#1}

  \begin{CJK}{UTF8}{bkai}
    \begin{center}\Large \bf
      \title@zh{}
    \end{center}
    \begin{center}\large \bf
      研究生: \student@zh{} \hspace{1cm} 指導教授: \advisor@zh{} \hspace{1cm} 博士 \\
    \end{center}
    \begin{center}\large \bf
      \school@zh{}\institute@zh{}
    \end{center}
    \vspace{3mm}
    \begin{center} \large \bf
      \abstractzhname{}
    \end{center}
    }
    {
    \vspace{4mm}

    \vspace{8pt}
    \textbf{\emph{關鍵字: \abstract@zh@keywords}}
    \vspace{8pt}

  \end{CJK}

  \newpage
}

\newcommand{\abstract@en@keywords}{}
\newenvironment{abstracten}[1]
{
  \addcontentsline{toc}{chapter}{\numberline{}Abstract}
  \renewcommand{\abstract@en@keywords}{#1}

  \begin{center}\Large\bf
    \title@en{}
  \end{center}
  \begin{center}\large
    Student: \student@en{} \hspace{1cm} Advisor: Dr. \advisor@en{}\\
  \end{center}
  \begin{center}\large
    \institute@en{}\\
    \school@en
  \end{center}
  \vspace{3mm}

  \begin{center}
    {\large\bf \abstractname}
  \end{center}
}
{
  \vspace{4mm}

  \vspace{8pt}
  \textbf{\emph{Keywords: \abstract@en@keywords}}
  \vspace{8pt}

  \newpage
}
% end frontmatter environments
\makeatother

% structure related commands
\newcommand{\frontmatter}
{
  \pagenumbering{roman}
}
\newcommand{\mainmatter}
{
  \newpage
  \pagenumbering{arabic}
  \parskip=12pt
}
% end structure related commands


% preamble
% packages
%%% nzlee: use command \input if you don't need to compile a chapter alone
%\usepackage{subfiles}
\usepackage{rotating}
\usepackage{xspace}
\usepackage{tcolorbox}
\usepackage{tikz-qtree}
\usepackage{circuitikz}
\usetikzlibrary{positioning,arrows}
% Plots with BenchExec
\usepackage{standalone}
\usepackage[
    group-digits=integer, group-minimum-digits=4, % group digits by thousands
    free-standing-units, unit-optional-argument, % easier input of numbers with units
]{siunitx}
\usepackage{pgfplots}
\pgfplotsset{
    compat=1.9,
    log ticks with fixed point, % no scientific notation in plots
    table/col sep=tab, % only tabs are column separators
    unbounded coords=jump, % better have skips in a plot than appear to be interpolating
    filter discard warning=false, % Don't complain about empty cells
}
\SendSettingsToPgf % use siunitx formatting settings in PGF, too
% end packages

%%% nzlee: This thesis combines the following four papers
%%% Solving Stochastic Boolean Satisfiability under Random-Exist Quantification (IJCAI '17)
%%% Solving Exist-Random Quantified Stochastic Boolean Satisfiability via Clause Selection (IJCAI '18)
%%% Towards Formal Evaluation and Verification of Probabilistic Design (TC '18)
%%% Dependency Stochastic Boolean Satisfiability: A Logical Formalism for NEXPTIME Decision Problems with Uncertainty (AAAI '21)

\settitle
{論隨機布林可滿足性:\\積體電路應用、當代演算法、與複雜度推廣}
{Stochastic Boolean Satisfiability:\\VLSI Applications, Modern Algorithms, and Generalization to NEXPTIME-Completeness}
\setstudent{李念澤}{Niann-Tzer Li}
\setadvisor{江介宏}{Jie-Hong Roland Jiang}
\setwatermarkfile{./fig/watermark.pdf}
%%% TODO: update the DOI
%\setdoi{doi:10.6342/NTU201902986}
\ackname{\cjk{致謝}}

\addwatermark

% symbols
\newcommand{\booldom}{\mathbb{B}} % Boolean domain
\newcommand{\limply}{\rightarrow} % logical implication
\newcommand{\vl}[1]{\texttt{var}(#1)} % variable of a literal
\newcommand{\as}{\tau} % an assignment
\newcommand{\av}[1]{\mathcal{A}(#1)} % all assignments over a variable set
\newcommand{\pf}{\phi} % propositional formula (quantifier-free)
\newcommand{\vf}[1]{\texttt{vars}(#1)} % variable of a formula
\newcommand{\pcf}[2]{#1|_{#2}} % positive cofactor
\newcommand{\ncf}[2]{#1|_{\lnot#2}} % negative cofactor
\newcommand{\Qf}{\Phi} % quantified formula
\newcommand{\base}{X_d} % a base set of a formula
\newcommand{\random}[1]{\rotatebox[origin=c]{180}{$\mathsf{R}$}^{#1}} % randomized quantifier
\newcommand{\spb}[1]{\Pr[#1]} % satisfying probability
\newcommand{\wt}{\omega} % weight of a variable
\newcommand{\sat}[1]{\texttt{SAT}(#1)} % formula is satisfiable
\newcommand{\unsat}[1]{\texttt{UNSAT}(#1)} % formula is unsatisfiable
\newcommand{\model}[1]{#1.\mathrm{model}} % formula is unsatisfiable
\newcommand{\select}{\psi} % selection formula
\newcommand{\cx}{C^X} % sub-clause of X variables
\newcommand{\cy}{C^Y} % sub-clause of Y variables
\newcommand{\sv}[1]{s_{#1}} % selection variable
\newcommand{\dep}[1]{D_{#1}} % dependency set
\newcommand{\uvs}{V_{\Qf}^{\forall}} % universal variable set
\newcommand{\evs}{V_{\Qf}^{\exists}} % existential variable set
\newcommand{\rvs}{V_{\Qf}^{\random{}}} % random variable set
\newcommand{\skf}{\mathcal{F}} % Skolem function set
\newcommand{\nodeval}[1]{#1.\mathrm{value}} % BDD node value
\newcommand{\nodevar}[1]{#1.\mathrm{var}} % BDD node variable
\newcommand{\nodevisit}[1]{#1.\mathrm{visited}} % BDD node visited flag
\newcommand{\nodethen}[1]{#1.\mathrm{then}} % BDD node then
\newcommand{\nodeelse}[1]{#1.\mathrm{else}} % BDD node else
\newcommand{\nodesp}[1]{#1.\mathrm{sp}} % BDD node satisfying probability
\DeclareMathOperator*{\argmax}{arg\,max} % maximizing argument
\DeclareMathOperator*{\argmin}{arg\,min} % minimizing argument

% tools
\newcommand{\tool}[1]{\texttt{#1}\xspace}
\newcommand{\definetool}[2]{\newcommand{#1}{\tool{#2}}\xspace}
\definetool{\ressat}{reSSAT}
\definetool{\ressatb}{reSSAT-b}
\definetool{\erssat}{erSSAT}
\definetool{\maxplan}{MAXPLAN}
\definetool{\zander}{ZANDER}
\definetool{\dcssat}{DC-SSAT}
\definetool{\complan}{ComPlan}
\definetool{\minisat}{MiniSat}
\definetool{\cachet}{Cachet}
\definetool{\abc}{ABC}
\definetool{\cudd}{CUDD}
\definetool{\prism}{PRISM}
\definetool{\timeout}{TO}
\definetool{\memout}{MO}

% constant words
\newcommand{\word}[1]{\textsc{#1}\xspace}
\newcommand{\defineword}[2]{\newcommand{#1}{\word{#2}}\xspace}
\defineword{\true}{true}
\defineword{\false}{false}
\defineword{\disjoin}{or}
\defineword{\conjoin}{and}
\defineword{\nand}{nand}
\defineword{\xor}{xor}

% customize package "amsthm"
\renewcommand\qedsymbol{$\blacksquare$}

% customize package "cleveref"
\newcommand{\creflastconjunction}{, and~}
\crefname{equation}{Eq.}{Eqs.}
\crefname{algorithm}{Alg.}{Algs.}
\crefname{line}{line}{lines}
\crefalias{ALC@unique}{line}
\crefalias{ALC@line}{line}

% customize package "algorithmic"
\renewcommand{\algorithmicrequire}{\textbf{Input:}}
\renewcommand{\algorithmicensure}{\textbf{Output:}}

% Fix line counters if multiple algorithms exist
\makeatletter % Make '@' a normal letter so that it can be used in *.tex files
\@addtoreset{ALC@line}{algorithm}
\@addtoreset{ALC@unique}{algorithm}
\makeatother % Undo the change to '@'