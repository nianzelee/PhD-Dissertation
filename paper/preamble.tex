\documentclass[12pt, a4paper]{dmathesis}

%% The following is to control the format of your thesis
%%%%%%%%%%%%%%%%%%%%%%%%%%%%%%%%%%%%%%%%%%%%%%%%%%%%%%%%%%%%%%%%%%%%%%%%%%
%     This is format.tex file needed for the dmathesis.cls file.  You    %
%  have to  put this file in the same directory with your thesis files.  %
%                Written by M. Imran 2001/06/18                          %
%                 No Copyright for this file                             %
%                 Save your time and enjoy it                            %
%                                                                        %
%%%%%%%%%%%%%%%%%%%%%%%%%%%%%%%%%%%%%%%%%%%%%%%%%%%%%%%%%%%%%%%%%%%%%%%%%%
%%%%%  Put packages you want to use here and 'fancyhdr' is required   %%%%
%%%%%%%%%%%%%%%%%%%%%%%%%%%%%%%%%%%%%%%%%%%%%%%%%%%%%%%%%%%%%%%%%%%%%%%%%%
\usepackage{fancyhdr}
\usepackage{epsfig}
\usepackage{cite}
\usepackage{graphicx}
\usepackage{amsmath}
\usepackage{amsthm}
\usepackage{amssymb}
\usepackage{latexsym}
\usepackage{epic}
\usepackage{algorithm}
\usepackage{algorithmic}
\usepackage{CJK}
\usepackage{xwatermark}
\usepackage{tikz}
\usepackage{hyperref}
\usepackage[capitalise,nosort,nameinlink]{cleveref}
\usepackage[nottoc]{tocbibind}
%%%%%%%%%%%%%%%%%%%%%%%%%%%%%%%%%%%%%%%%%%%%%%%%%%%%%%%%%%%%%%%%%%%%%%%%%%
%%%%%                 Set line spacing = 1.5 here                   %%%%%%
%%%%%%%%%%%%%%%%%%%%%%%%%%%%%%%%%%%%%%%%%%%%%%%%%%%%%%%%%%%%%%%%%%%%%%%%%%
\renewcommand{\baselinestretch}{2.0}
%%%%%%%%%%%%%%%%%%%%%%%%%%%%%%%%%%%%%%%%%%%%%%%%%%%%%%%%%%%%%%%%%%%%%%%%%%
%%%%%                      Your fancy heading                       %%%%%%
%%%%% For the final copy you need to remove '\bfseries\today' below %%%%%%
%%%%%%%%%%%%%%%%%%%%%%%%%%%%%%%%%%%%%%%%%%%%%%%%%%%%%%%%%%%%%%%%%%%%%%%%%%
\pagestyle{fancy}
\renewcommand{\chaptermark}[1]{\markright{\chaptername\ \thechapter.\ #1}}
\renewcommand{\sectionmark}[1]{\markright{\thesection.\ #1}{}}
\lhead[\fancyplain{}{}]%
{\fancyplain{}{\bfseries\rightmark}}
\chead[\fancyplain{}{}]%
{\fancyplain{}{}}
\rhead[\fancyplain{}{}]%
{\fancyplain{}{\bfseries}}
\lfoot[\fancyplain{}{}]%
{\fancyplain{}{}}
\cfoot[\fancyplain{}{}]%
{\fancyplain{}{\thepage}}
\rfoot[\fancyplain{}{}]%
{\fancyplain{}{}}%{\bfseries\today}}
%%%%%%%%%%%%%%%%%%%%%%%%%%%%%%%%%%%%%%%%%%%%%%%%%%%%%%%%%%%%%%%%%%%%%%%%%%
%%%%%%%%%%%%Here you set the space between the main text%%%%%%%%%%%%%%%%%%
%%%%%%%%%%%%%%%%%%%and the start of the footnote%%%%%%%%%%%%%%%%%%%%%%%%%%
%%%%%%%%%%%%%%%%%%%%%%%%%%%%%%%%%%%%%%%%%%%%%%%%%%%%%%%%%%%%%%%%%%%%%%%%%%
\addtolength{\skip\footins}{5mm}
\addtolength{\footskip}{5mm}
\addtolength{\headsep}{10mm}
\addtolength{\textheight}{-15mm}
%%%%%%%%%%%%%%%%%%%%%%%%%%%%%%%%%%%%%%%%%%%%%%%%%%%%%%%%%%%%%%%%%%%%%%%%%%
%%%%%      Define new counter so you can have the equation           %%%%%
%%%%%    number 4.2.1a for example, this a gift from J.F.Blowey      %%%%%
%%%%%%%%%%%%%%%%%%%%%%%%%%%%%%%%%%%%%%%%%%%%%%%%%%%%%%%%%%%%%%%%%%%%%%%%%%
\newcounter{ind}
\def\eqlabon{
  \setcounter{ind}{\value{equation}}\addtocounter{ind}{1}
  \setcounter{equation}{0}
  \renewcommand{\theequation}{\arabic{chapter}%
    .\arabic{section}.\arabic{ind}\alph{equation}}}
\def\eqlaboff{
  \renewcommand{\theequation}{\arabic{chapter}%
    .\arabic{section}.\arabic{equation}}
  \setcounter{equation}{\value{ind}}}
%%%%%%%%%%%%%%%%%%%%%%%%%%%%%%%%%%%%%%%%%%%%%%%%%%%%%%%%%%%%%%%%%%%%%%%%%%
%%%%%%%%%%%%           New theorem you want to use              %%%%%%%%%%
%%%%%%%%%%%%%%%%%%%%%%%%%%%%%%%%%%%%%%%%%%%%%%%%%%%%%%%%%%%%%%%%%%%%%%%%%%
\newtheorem{definition}{Definition}[chapter]
\newtheorem{proposition}{Proposition}[chapter]
\newtheorem{example}{Example}[chapter]
\newtheorem{theorem}{Theorem}[chapter]
\newtheorem{corollary}{Corollary}[theorem]
\newtheorem{lemma}[theorem]{Lemma}
%%%%%%%%%%%%%%%%%%%%%%%%%%%%%%%%%%%%%%%%%%%%%%%%%%%%%%%%%%%%%%%%%%%%%%%%%%
%%%%%%%    Bold font in math mode, a gift from J.F.Blowey       %%%%%%%%%%
%%%%%%%%%%%%%%%%%%%%%%%%%%%%%%%%%%%%%%%%%%%%%%%%%%%%%%%%%%%%%%%%%%%%%%%%%%
\def\bv#1{\mbox{\boldmath$#1$}}
%%%%%%%%%%%%%%%%%%%%%%%%%%%%%%%%%%%%%%%%%%%%%%%%%%%%%%%%%%%%%%%%%%%%%%%%%%
%%%%%%%        New symbol which is not defined in Latex         %%%%%%%%%%
%%%%%%%                 a gift from J.F.Blowey                  %%%%%%%%%%
%%%%%%%%%%%%%%%%%%%%%%%%%%%%%%%%%%%%%%%%%%%%%%%%%%%%%%%%%%%%%%%%%%%%%%%%%%
% The Mean INTegral
% to be used in displaystyle
\def\mint{\textstyle\mints\displaystyle}
% to be used in textstyle
\def\mints{\int\!\!\!\!\!\!{\rm-}\ }
%
% The Mean SUM
% to be used in displaystyle
\def\msum{\textstyle\msums\displaystyle}
% to be used in textstyle
\def\msums{\sum\!\!\!\!\!\!\!{\rm-}\ }
%%%%%%%%%%%%%%%%%%%%%%%%%%%%%%%%%%%%%%%%%%%%%%%%%%%%%%%%%%%%%%%%%%%%%%%%%%
%%%%%%%%%%            Define your short cut here              %%%%%%%%%%%%
%%%%%%%%%%%%%%%%%%%%%%%%%%%%%%%%%%%%%%%%%%%%%%%%%%%%%%%%%%%%%%%%%%%%%%%%%%
\def\poincare{Poincar\'e }
\def\holder{H\"older }

%%%%%%%%%%%%%%%%%%%%%%%%%%%%%%%%%%%%%%%%%%%%%%%%%%%%%%%%%%%%%%%%%%%%%%%%%%
%%%%%%%%%%   Added by Flotisable base on senior's template    %%%%%%%%%%%%
%%%%%%%%%%%%%%%%%%%%%%%%%%%%%%%%%%%%%%%%%%%%%%%%%%%%%%%%%%%%%%%%%%%%%%%%%%
\makeatletter
\newcommand{\cjk}[1]{\begin{CJK}{UTF8}{bkai}#1\end{CJK}}

% defaults
\newcommand{\title@zh}{}
\newcommand{\title@en}{}
\newcommand{\student@zh}{}
\newcommand{\student@en}{}
\newcommand{\advisor@zh}{}
\newcommand{\advisor@en}{}
\newcommand{\school@zh}{\cjk{國立臺灣大學}}
\newcommand{\school@en}{National Taiwan University}
\newcommand{\institute@zh}{\cjk{電子工程學研究所}}
\newcommand{\institute@en}{Graduate Institute of Electronics Engineering}
\newcommand{\time@en}{June 2021}
\newcommand{\watermark@file}{fig/watermark.pdf}
\newcommand{\doi}{DOI}
\newcommand{\ackname@toc}{Acknowledgements}
\newcommand{\ackname@title}{Acknowledgements}
\newcommand{\abstractzhname}{\cjk{摘要}}
% end defaults

% name settings commands
\newcommand{\settitle}[2]
{
  \renewcommand{\title@zh}{\cjk{#1}}
  \renewcommand{\title@en}{#2}
}
\newcommand{\setstudent}[2]
{
  \renewcommand{\student@zh}{\cjk{#1}}
  \renewcommand{\student@en}{#2}
}
\newcommand{\setadvisor}[2]
{
  \renewcommand{\advisor@zh}{\cjk{#1}}
  \renewcommand{\advisor@en}{#2}
}
\newcommand{\setschool}[2]
{
  \renewcommand{\school@zh}{\cjk{#1}}
  \renewcommand{\school@en}{#2}
}
\newcommand{\setinstitute}[2]
{
  \renewcommand{\institute@zh}{\cjk{#1}}
  \renewcommand{\institute@en}{#2}
}
\newcommand{\setwatermarkfile}[1]
{
  \renewcommand{\watermark@file}{#1}
}
\newcommand{\setdoi}[1]
{
  \renewcommand{\doi}{#1}
}
\newcommand{\ackname}[2][Acknowledgements]
{
  \renewcommand{\ackname@toc}{#1}
  \renewcommand{\ackname@title}{#2}
}
% end name settings commands

% watermark and doi
\newlength{\watermark@x@shift}
\newlength{\watermark@y@shift}
\newlength{\doi@width}
\setlength{\watermark@x@shift}{26mm}
\setlength{\watermark@y@shift}{-24mm}
\newcommand{\addwatermark}
{
  \settowidth{\doi@width}{\doi}
  \newwatermark
  [
    scale=0.5,
    xpos=\watermark@x@shift+0.5\paperwidth-73mm/2-2.5cm,
    ypos=\watermark@y@shift+0.5\paperheight-73mm/2-2.5cm,
    allpages
  ]
  {\tikz[opacity=0.5]{\node{\includegraphics{\watermark@file}}}}
  \newwatermark
  [
    fontsize=12pt,
    fontseries=m,
    color=black,
    xpos=\watermark@x@shift+0.5\paperwidth-0.5\doi@width-1cm,
    ypos=\watermark@y@shift-0.5\paperheight+6pt+1cm,
    allpages
  ]{\doi}
}
% end watermark and doi

% frontmatter environments
\newenvironment{acknowledgements}
{
  \addcontentsline{toc}{chapter}{\numberline{}\ackname@toc}

  \begin{center}
    {\Large\bf \ackname@title}
  \end{center}

  \begin{CJK}{UTF8}{bkai}
    }
    {
  \end{CJK}

  \begin{flushright}
    \student@zh{}\\
    \student@en{}\\
  \end{flushright}
  \emph{\school@en}\\
  \emph{\time@en}

  \newpage
}

\newcommand{\abstract@zh@keywords}{}
\newenvironment{abstractzh}[1]
{
  \addcontentsline{toc}{chapter}{\numberline{}Chinese Abstract}
  \renewcommand{\abstract@zh@keywords}{#1}

  \begin{CJK}{UTF8}{bkai}
    \begin{center}\Large \bf
      \title@zh{}
    \end{center}
    \begin{center}\large \bf
      研究生: \student@zh{} \hspace{1cm} 指導教授: \advisor@zh{} \hspace{1cm} 博士 \\
    \end{center}
    \begin{center}\large \bf
      \school@zh{}\institute@zh{}
    \end{center}
    \vspace{3mm}
    \begin{center} \large \bf
      \abstractzhname{}
    \end{center}
    }
    {
    \vspace{4mm}

    \vspace{8pt}
    \textbf{\emph{關鍵字: \abstract@zh@keywords}}
    \vspace{8pt}

  \end{CJK}

  \newpage
}

\newcommand{\abstract@en@keywords}{}
\newenvironment{abstracten}[1]
{
  \addcontentsline{toc}{chapter}{\numberline{}Abstract}
  \renewcommand{\abstract@en@keywords}{#1}

  \begin{center}\Large\bf
    \title@en{}
  \end{center}
  \begin{center}\large
    Student: \student@en{} \hspace{1cm} Advisor: Dr. \advisor@en{}\\
  \end{center}
  \begin{center}\large
    \institute@en{}\\
    \school@en
  \end{center}
  \vspace{3mm}

  \begin{center}
    {\large\bf \abstractname}
  \end{center}
}
{
  \vspace{4mm}

  \vspace{8pt}
  \textbf{\emph{Keywords: \abstract@en@keywords}}
  \vspace{8pt}

  \newpage
}
% end frontmatter environments
\makeatother

% structure related commands
\newcommand{\frontmatter}
{
  \pagenumbering{roman}
}
\newcommand{\mainmatter}
{
  \newpage
  \pagenumbering{arabic}
  \parskip=12pt
}
% end structure related commands


% preamble
% packages
%%% nzlee: use input instead if you don't need to compile a chapter alone
%\usepackage{subfiles}
\usepackage{syntax}
\usepackage{listings}
\usepackage{graphicx}
\usepackage{subfig}
\usepackage{hyperref}
\usepackage{float}
\usepackage{rotating}
\usepackage{xspace}
% end packages

%%% nzlee: This thesis combines the following four papers
%%% Solving Stochastic Boolean Satisfiability under Random-Exist Quantification (IJCAI '17)
%%% Solving Exist-Random Quantified Stochastic Boolean Satisfiability via Clause Selection (IJCAI '18)
%%% Towards Formal Evaluation and Verification of Probabilistic Design (TC '18)
%%% Dependency Stochastic Boolean Satisfiability: A Logical Formalism for NEXPTIME Decision Problems with Uncertainty (AAAI '21)

\settitle
{我的博士論文}
{Stochastic Boolean Satisfiability:\\ Novel Algorithms, VLSI Applications, and Generalization to NEXPTIME-Completeness}
\setstudent{李念澤}{Niann-Tzer Li}
\setadvisor{江介宏}{Jie-Hong Roland Jiang}
\setwatermarkfile{./fig/watermark.pdf}
%%% TODO: update the DOI
\setdoi{doi:10.6342/NTU201902986}
\ackname{\cjk{致謝}}

%\addwatermark

% package settings
\setlength{\grammarparsep}{4pt}
\renewcommand{\lstlistingname}{Program}
\lstset
{
  basicstyle=\ttfamily,
  captionpos=b,
  lineskip=-1pt
}
\iffalse
  \chemsetup
  {
    modules=reactions,
    formula=mhchem,
    mhchem/version=4,
    reactions/tag-open=(,
    reactions/tag-close=)
  }
  \numberwithin{reaction}{chapter}
\fi
% end package settings

% self-defined commands
\newcommand{\random}[1]{\rotatebox[origin=c]{180}{$\mathsf{R}$}^{#1}}

% tools
\newcommand{\tool}[1]{\texttt{#1}\xspace}
\newcommand{\definetool}[2]{\newcommand{#1}{\tool{#2}}\xspace}
\definetool{\ressat}{reSSAT}
\definetool{\erssat}{erSSAT}
\definetool{\maxplan}{MAXPLAN}
\definetool{\zander}{ZANDER}
\definetool{\dcssat}{DC-SSAT}

% constant words
\newcommand{\word}[1]{\textsc{#1}\xspace}
\newcommand{\defineword}[2]{\newcommand{#1}{\word{#2}}\xspace}
\defineword{\true}{true}
\defineword{\false}{false}
% end constant words

% cross reference related commands
\makeatletter
\newcommand{\sec@name}{\chaptername}
\newcommand{\eq@name}{Equation}
\newcommand{\fig@name}{\figurename}
\newcommand{\lst@name}{\lstlistingname}
\newcommand{\tab@name}{\tablename}
\newcommand{\re@name}{Reaction}
\newcommand{\subfig@name}{\figurename}
\newcommand{\secref}{\@ifstar{\secref@star}{\secref@nostar}}
\newcommand{\secref@star}[1]{\ref{sec:#1}}
\newcommand{\secref@nostar}[1]{\sec@name{}~\ref{sec:#1}}
\newcommand{\seclabel}[1]{\label{sec:#1}}
\renewcommand{\eqref}{\@ifstar{\eqref@star}{\eqref@nostar}}
\newcommand{\eqref@star}[1]{(\ref{eq:#1})}
\newcommand{\eqref@nostar}[1]{\eq@name{}~(\ref{eq:#1})}
\newcommand{\eqlabel}[1]{\label{eq:#1}}
\newcommand{\figref}{\@ifstar{\figref@star}{\figref@nostar}}
\newcommand{\figref@star}[1]{\ref{fig:#1}}
\newcommand{\figref@nostar}[1]{\fig@name{}~\ref{fig:#1}}
\newcommand{\figlabel}[1]{\label{fig:#1}}
\newcommand{\lstref}{\@ifstar{\lstref@star}{\lstref@nostar}}
\newcommand{\lstref@star}[1]{\ref{lst:#1}}
\newcommand{\lstref@nostar}[1]{\lstlistingname{}~\ref{lst:#1}}
\newcommand{\lstlabel}[1]{\label{lst:#1}}
\newcommand{\tabref}{\@ifstar{\tabref@star}{\tabref@nostar}}
\newcommand{\tabref@star}[1]{\ref{tab:#1}}
\newcommand{\tabref@nostar}[1]{\tab@name{}~\ref{tab:#1}}
\newcommand{\tablabel}[1]{\label{tab:#1}}
\newcommand{\reref}{\@ifstar{\reref@star}{\reref@nostar}}
\newcommand{\reref@star}[1]{(\ref{re:#1})}
\newcommand{\reref@nostar}[1]{\re@name{}~(\ref{re:#1})}
\newcommand{\relabel}[1]{\label{re:#1}}
\newcommand{\subfigref}{\@ifstar{\subfigref@star}{\subfigref@nostar}}
\newcommand{\subfigref@star}[1]{\subref{fig:#1}}
\newcommand{\subfigref@nostar}[1]{\subfig@name{}~\subref{fig:#1}}
\makeatother
% end cross reference related commands
% end self-defined commands
% end preamble
