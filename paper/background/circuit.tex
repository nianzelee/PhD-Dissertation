\section{Boolean-logic circuit}
\label{sect:circuit}

A Boolean-logic circuit, defined below, will be extended to model a probabilistic design in~\cref{chap:prob-eval}.

\begin{definition}[Boolean Network]\label{def:Boolean network}
    A \emph{(combinational) Boolean network} is a directed acyclic
    graph $G=(V,E)$, with vertices $V$ and edges $E \subseteq V \times
        V$. Two non-empty disjoint subsets $V_I$ and $V_O$ of $V$ are
    identified; a vertex $v \in V_I$ (resp. $V_O$) is referred to as a
    \emph{primary input} (PI) (resp. \emph{primary output} (PO)). Each
    vertex $v \in V$ is associated with a Boolean variable $b_v$; each
    vertex $v \in V \backslash V_I$ is associated with a Boolean
    function $f_v$ letting $b_v \leftrightarrow f_v$. An edge $(u,v)
        \in E$ indicates $f_v$ refers to $b_u$ as an input variable; $u$
    is called a \emph{fanin} of $v$, and $v$ a \emph{fanout} of $u$.
\end{definition}
To ease readability, in the sequel we shall not distinguish a
vertex $v$ and its corresponding Boolean variable $b_v$, and
simply denote $b_v$ with $v$.

For satisfiability (SAT) testing, a Boolean network (circuit) can
be converted in linear time to a \emph{conjunctive normal form}
(CNF) formula through Tseitin transformation~\cite{Tseitin1983},
which will be used in our development of stochastic SAT and model
counting methods.

\subsection{And-inverter graph}