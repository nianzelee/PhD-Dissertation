\section{Stochastic Boolean satisfiability}
\label{sect:ssat}

An SSAT formula $\Qf$ over variables $\{x_1,\ldots,x_n\}$ has the form:
$Q_1x_1,\ldots,Q_nx_n.\pf$,
where each $Q_i\in\{\exists,\random{p}\}$ and $\pf$ is quantifier-free.
Symbol $\exists$ denotes an existential quantifier with its conventional semantics.
Symbol $\random{p}$ denotes a randomized quantifier~\cite{Papadimitriou1985}, which requires
the quantified variable to evaluate to $\top$ with probability $p\in[0,1]$.
Given an SSAT formula $\Qf$, the quantification structure $Q_1 x_1, \ldots, Q_n x_n$ is called the \emph{prefix},
and the quantifier-free Boolean formula $\phi$ is called the \emph{matrix}.

Let $x$ be the outermost variable in the prefix of an SSAT formula $\Qf$.
The satisfying probability of $\Qf$, denoted by $\spb{\Qf}$, is defined by the following four rules:
\begin{enumerate}
    \item[a)] $\spb{\top}=1$,
    \item[b)] $\spb{\bot}=0$,
    \item[c)] $\spb{\Qf}=\max\{\spb{\ncf{\Qf}{x}},\spb{\pcf{\Qf}{x}}\}$, if $x$ is existentially quantified,
    \item[d)] $\spb{\Qf}=(1-p)\spb{\ncf{\Qf}{x}}+p\spb{\pcf{\Qf}{x}}$, if $x$ is randomly quantified by $\random{p}$,
\end{enumerate}
where $\ncf{\Qf}{x}$ and $\pcf{\Qf}{x}$ denote the SSAT formulas obtained by eliminating the outermost quantifier of $x$ via substituting the value of $x$ in the matrix with $\bot$ and $\top$, respectively.

The \textit{decision version} of SSAT is stated as follows.
Given an SSAT formula $\Qf$ and a threshold $\theta\in[0,1]$, decide whether $\spb{\Qf}\geq\theta$.
On the other hand, the \textit{optimization version} asks to compute the exact value of $\spb{\Qf}$.
The decision version of SSAT is PSPACE-complete~\cite{Papadimitriou1985}.

\iffalse
    \subsection{Reduction from E-MAJSAT to Model Counting}
    Given an E-MAJSAT formula $\Phi=\exists X \invR Y. \phi(X,Y)$, our goal is to evaluate $Pr[\Phi]$. By the rules of SSAT formula, it's to find $\max\{Pr[\Phi|_{\tau}]:\forall \tau \in X \rightarrow \mathbb{B}\}$. Each $\Phi|_{\tau} = \invR Y.\phi (Y)$ is merely random quantified SSAT formula since all $X$ variables have been eliminated. If we assign each random quantified variable $r_i$ a weight $p_i$ corresponding to randomized quantifier $\invR^{p_i} r_i$, $Pr[\Phi|_{\tau}]$ can be seen as a Boolean formula $\phi$ with a weight function. So we can use any model counting algorithm to find the value for it.

    When we are trying all the assignments on $X$, it is common for two assignments $\tau_1$, $\tau_2$ such that $\phi|_{\tau_1} \supseteq \phi|_{\tau_2}$. The implication $\phi|_{\tau_1} \rightarrow \phi|_{\tau_2}$ can be induced, so if there is an assignment $\alpha_1$ such that $\phi|_{\tau_1 \wedge \alpha_1} = \top$ then $\phi|_{\tau_2 \wedge \alpha_1} = \top$. Take the weight function	into consideration, the weights of all satisfying assignments of $\phi|_{\tau_2}$ must be great equal then $\phi|_{\tau_1}$. It gives us the propositions below:

    Given an E-MAJSAT formula $\Phi=\exists X \random{} Y. \phi(X,Y)$ and an assignment $\tau$ on $X$, cofactoring the matrix with $\tau$ results in a formula $\phi|_{\tau}$ referring only to variables in $Y$.
    The prefix $\random{} Y$ induces a weight function $\omega: Y \rightarrow [0,1]$ for each variable $y \in Y$, where $\omega(y)$ equals the probability annotated on the randomized quantifier of $y$.
    As a result, the conditional satisfying probability $\Pr[\Phi|_{\tau}]$, which equals the weight of the formula $\phi|_{\tau}$ under the weight function $\omega$, can be obtained by invoking a weighted model counter on the formula $\phi|_{\tau}$ with the weight function $\omega$.
    In the sequel, the invocation of a weighted model counter is expressed by $\mathtt{WeightModelCount}(\random{} Y. \phi|_{\tau})$, which returns the conditional satisfying probability $\Pr[\random{} Y. \phi|_{\tau}]$.
\fi