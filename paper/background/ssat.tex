\section{Stochastic Boolean satisfiability}
\label{sect:background-ssat}

An SSAT formula $\Qf$ over variables $\{x_1,\ldots,x_n\}$ has the form:
$Q_1x_1,\ldots,Q_nx_n.\pf$,
where each $Q_i\in\{\exists,\random{p}\}$ and $\pf$ is quantifier-free.
Symbol $\exists$ denotes an existential quantifier with its conventional semantics.
Symbol $\random{p}$ denotes a randomized quantifier~\cite{Papadimitriou1985}, which requires
the quantified variable to evaluate to $\top$ with probability $p\in[0,1]$.
Given an SSAT formula $\Qf$, the quantification structure $Q_1 x_1, \ldots, Q_n x_n$ is called the \emph{prefix},
and the quantifier-free Boolean formula $\phi$ is called the \emph{matrix}.

Let $x$ be the outermost variable in the prefix of an SSAT formula $\Qf$.
The satisfying probability of $\Qf$, denoted by $\spb{\Qf}$, is defined by the following four rules:
\begin{enumerate}
    \item[a)] $\spb{\top}=1$,
    \item[b)] $\spb{\bot}=0$,
    \item[c)] $\spb{\Qf}=\max\{\spb{\ncf{\Qf}{x}},\spb{\pcf{\Qf}{x}}\}$, if $x$ is existentially quantified,
    \item[d)] $\spb{\Qf}=(1-p)\spb{\ncf{\Qf}{x}}+p\spb{\pcf{\Qf}{x}}$, if $x$ is randomly quantified by $\random{p}$,
\end{enumerate}
where $\ncf{\Qf}{x}$ and $\pcf{\Qf}{x}$ denote the SSAT formulas obtained by eliminating the outermost quantifier of $x$ via substituting the value of $x$ in the matrix with $\bot$ and $\top$, respectively.

The \textit{decision version} of SSAT is stated as follows.
Given an SSAT formula $\Qf$ and a threshold $\theta\in[0,1]$, decide whether $\spb{\Qf}\geq\theta$.
On the other hand, the \textit{optimization version} asks to compute the exact value of $\spb{\Qf}$.
The decision version of SSAT is PSPACE-complete~\cite{Papadimitriou1985}.

An SSAT formula can also be interpreted from a game-theoretical viewpoint.
The randomized quantifiers represent the nondeterministic factors in a stochastic game.
The existential quantifiers model the moves of an agent who plays under such uncertainty.
The satisfying probability of the SSAT formula corresponds to the maximum winning probability of the agent.
A \textit{Skolem function} for an existentially quantified variable is the agent's strategy to assign this variable.
Note that the Skolem function for a variable can only depend on its preceding variables in the prefix.
A set of optimal Skolem functions achieves the maximum winning probability.

\begin{example}
    Consider an SSAT formula $\Qf$:
    \begin{align*}
        \random{0.5}x_1,\exists y_1,\random{0.5}x_2,\exists y_2.
        (x_1\lor\lnot y_1)(\lnot x_1\lor y_1)
        (\lnot x_1\lor\lnot x_2\lor y_2)(x_1\lor\lnot y_2)(x_2\lor\lnot y_2).
    \end{align*}
    According to the computational rules for the satisfying probability of SSAT, we have $\spb{\Qf}=1$.
    The maximum winning probability can be achieved by assigning $y_1$ to $f_1(x_1)=x_1$ and $y_2$ to $f_2(x_1,x_2)=x_1\land x_2$.
    The set of functions $\{f_1,f_2\}$ is a set of optimal Skolem functions for $\Qf$.
\end{example}