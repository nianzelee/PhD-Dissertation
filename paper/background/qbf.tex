\section{Quantified Boolean formula}
\label{sect:qbf}

\subsection{Clause selection}
\textit{Clause selection}~\cite{Janota2015,Rabe2015} is a recently proposed technique for QBF solving.
Given a CNF formula $\phi(X,Y)$ over a set of variables $X \cup Y$ with $X \cap Y = \emptyset$, we divide each clause $C \in \phi$ into two sub-clauses $C^X$ and $C^Y$, where $C^X$ (resp. $C^Y$) consists of the literals whose variables are in $X$ (resp. $Y$). For example, for $C=(x_1 \vee x_2 \vee y_1 \vee y_2)$, we have $C^X=(x_1 \vee x_2)$ and $C^Y=(y_1 \vee y_2)$. Clearly, $C=C^X \vee C^Y$.

A clause $C$ is said to be \emph{selected} by an assignment $\tau$ over $X$ if every literal in $C^X$ is assigned to $\bot$ by $\tau$; $C$ is said to be \emph{deselected} by $\tau$ if some literal in $C^X$ is assigned to $\top$ by $\tau$; $C$ is said to be \emph{undecided} if it is neither selected nor deselected.
We also use $\phi|_{\tau}$ to denote the set of clauses selected by the assignment $\tau$. A \emph{selection variable} $s_C$ is introduced for each clause $C$ and defined by $s_C \equiv \neg C^X$.
Hence, $s_C$ is an indicator of the selection of clause $C$.
That is, $s_C=\top$ (resp. $s_C=\bot$) indicates $C$ is selected (resp. deselected).
Let $S$ be the set of selection variables for clauses in $\phi(X,Y)$.
The formula $\psi(X,S)=\bigwedge_{C \in \phi}(s_C \equiv \neg C^X)$ is called the \emph{selection relation} of $\phi(X,Y)$.

\begin{example}\label{ex:select}
    Consider a CNF formula $\phi(X,Y)$ over two sets of variables $X=\{e_1,e_2,e_3\}$ and $Y=\{r_1,r_2,r_3\}$. $\phi(X,Y)$ consists of four clauses:
    \begin{itemize}
        \item[] $C_1: (e_1 \vee r_1 \vee r_2)$
        \item[] $C_2: (e_1 \vee e_2 \vee r_1 \vee r_2 \vee \neg r_3)$
        \item[] $C_3: (\neg e_2 \vee \neg e_3 \vee r_2 \vee \neg r_3)$
        \item[] $C_4: (\neg e_1 \vee e_3 \vee r_3)$
    \end{itemize}
    A selection variable $s_i$ is introduced for each clause, and $S=\{s_1,s_2,s_3,s_4\}$. The selection relation $\psi(X,S)$ of $\phi(X,Y)$ equals
    \begin{eqnarray*}
        \psi(X,S) = (s_1 \equiv \neg e_1)\wedge(s_2 \equiv \neg e_1 \wedge \neg e_2) \wedge (s_3 \equiv e_2 \wedge e_3) \\ \wedge(s_4 \equiv e_1 \wedge \neg e_3).
    \end{eqnarray*}
    Consider the complete assignment $\tau_1=\neg e_1 \neg e_2 \neg e_3$ over $X$. It selects $C_1$ and $C_2$, and deselects $C_3$ and $C_4$, as can be seen from the selection relation cofactored by $\tau_1$, which results in $\psi(X,S)|_{\tau_1}=s_1s_2\neg s_3 \neg s_4$.
    Consider the partial assignment $\tau_2=\neg e_1 e_3$ over $X$.
    It selects $C_1$, deselects $C_4$, and leaves $C_2$ and $C_3$ undecided.
    Notice that the two complete assignments $\neg e_1 \neg e_2 e_3$ and $\neg e_1 e_2 e_3$ consistent with $\tau_2$ select $\{C_1, C_2\}$ and $\{C_1, C_3\}$, respectively.
    The clause $C_1$ selected by the partial assignment $\tau_2$ lies in the intersection of the sets of clauses selected by the two complete assignments consistent with $\tau_2$.
\end{example}