\section{Model counting}
\label{sect:background-model-counting}

\subsection{Exact/Approximate model counting}
The \textit{model counting}~\cite{SATHandbook-ModelCounting} problem asks to find the number of the satisfying assignments of a Boolean formula $\pf$.
The exact algorithms compute the precise count $\#\pf$ of the satisfying assignments.
The approximate algorithms compute bounds of the precise count $\#\pf$ with a confidence level.
One common formulation is the $(\epsilon,\delta)$ approximate model counting,
which asks to find an answer that is sufficiently close to the precise count with high enough probability.
This formulation can be characterized by the inequality
$\spb{(1+\epsilon)^{-1}\#\pf\leq A\leq (1+\epsilon)\#\pf}\geq 1-\delta$,
where the parameters $\epsilon$ and $\delta$ can be configured to trade precision against scalability.

\subsection{Weighted model counting}
The weighted version asks to compute the weight of a formula $\pf$ given a weighting function $\wt:\vf{\pf}\mapsto[0,1]$.
The weight of a positive literal $x$ (resp. a negative literal $\lnot x$) is defined to be $\wt(x)$ (resp. $1-\wt(x)$).
The weight of an assignment $\as$, denoted as $\wt(\as)$, equals the product of the weights of its individual literals.
The weight of the formula $\pf$, denoted as $\wt(\pf)$, is the summation of the weights of its satisfying assignments.