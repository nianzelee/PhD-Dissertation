\section{Model counting}
\label{sect:model-counting}

\subsection{Exact/Approximate model counting}
The \textit{model counting}~\cite{SATHandbook-ModelCounting} problem asks to find the number of the satisfying assignments of a Boolean formula $\pf$.
Model-counting algorithms can be classified into two categories:
\textit{exact} algorithms and \textit{approximate} algorithms.
The former computes the exact number of the satisfying assignments of a formula;
the latter computes upper or lower bounds of the number of the satisfying assignments with a confidence level.

\subsection{Weighted model counting}
The \textit{weighted model counting} problem asks to compute the weight of a formula $\pf$ given a weighting function $\wt:\vf{\pf}\mapsto[0,1]$.
The weight of a positive literal $x$ (resp. a negative literal $\lnot x$) is defined to be $\wt(x)$ (resp. $1-\wt(x)$).
The weight of an assignment equals the product of the weights of its individual literals.
The weight of a formula is the summation of the weights of its satisfying assignments.