\section{Propositional logic}
\label{sect:propositional-logic}

We denote Boolean constants \false and \true by symbols $\bot$ and $\top$, respectively.
In arithmetic expressions, $\bot$ is interpreted as integer $0$ and $\top$ as integer $1$.
A variable $x$ that takes values from the Boolean domain $\booldom=\{\bot,\top\}$ is called a Boolean variable.
A \textit{literal} is a variable itself (a \textit{positive} literal) or the negation of a variable (a \textit{negative} literal).
For a literal $l$, let $\vl{l}$ denote the variable of $l$.
Boolean connectives $\lnot, \lor, \land, \limply, \equiv$ are used under their conventional semantics.
Over a finite set $V$ of Boolean variables,
we define a \textit{well-formed formula} $\pf$ with the following Backus-Naur-form (BNF) grammar:
\begin{align}
    \pf ::= x\in V | \lnot\pf | (\pf\lor\pf) | (\pf\land\pf) | (\pf\limply\pf) | (\pf\equiv\pf).
\end{align}
Given a well-formed formula $\pf$, let $\vf{\pf}$ denote the set of Boolean variables appearing in $\pf$.
In the following, a variable is Boolean if not otherwise specified.
We shall consider well-formed formulas only and refer to them as \textit{Boolean formulas}.

\subsection{Conjunctive and disjunctive normal form}
Among various representations of a Boolean formula,
we are particularly interested in normal-form representations because their simplicity allows efficient analyses.

A Boolean formula is in \textit{conjunctive normal form} (CNF) if it is a conjunction of \textit{clauses},
where a clause is a disjunction of literals.
A Boolean formula is in \textit{disjunctive normal form} (DNF) if it is a disjunction of \textit{cubes},
where a cube is a conjunction of literals.
A variable $x$ is said to be \textit{pure} in a formula if its appearances in the formula are all positive literals or negative literals.
We alternatively treat a clause or a cube as a set of literals,
and a CNF (resp. DNF) formula as a set of clauses (resp. cubes).
In the rest of the dissertation, a Boolean formula is assumed to be given in CNF if not otherwise specified.

\subsection{Boolean satisfiability}
An \textit{assignment} $\as$ over a variable set $V$ is a mapping from $V$ to $\booldom$.
We denote the set of all assignments over $V$ by $\av{V}$.
Given a Boolean formula $\pf$,
an assignment $\as$ over $\vf{\pf}$ is called a \textit{complete} assignment for $\pf$.
If $\as$ is over a proper subset of $\vf{\pf}$, it is called a \textit{partial} assignment.
The resultant formula of $\pf$ induced by an assignment $\as$ over a variable set $V$,
denoted as $\pcf{\pf}{\as}$,
is obtained via substituting the occurrences of every $x\in V$ in $\pf$ with its assigned value $\as(x)$.
Such substitution is called \textit{cofactoring} $\pf$ with $\as$.
If $V=\{x\}$, we write $\pcf{\pf}{x}$ (resp. $\ncf{\pf}{x}$) to denote the resultant formula of $\pf$ under an assignment that maps $x$ to $\top$ (resp. $\bot$),
and call this formula the \textit{positive} (resp. \textit{negative}) \textit{cofactor} of $\pf$ with respect to variable $x$.

A complete assignment $\as$ \textit{satisfies} $\pf$, denoted as $\as\models\pf$, if $\pcf{\pf}{\as}=\top$.
Such complete assignment $\as$ is called a \textit{satisfying complete assignment} for $\pf$.
On the other hand, if $\pcf{\pf}{\as}=\bot$, $\as$ is called an \textit{unsatisfying complete assignment}.
Similarly, a partial assignment $\as^+$ over $X\subset\vf{\pf}$ is called a \textit{satisfying} (resp. an \textit{unsatisfying}) \textit{partial assignment} for $\pf$
if for some (resp. every) assignment $\mu$ over $\vf{\pf}\setminus X$,
$\pf$ valuates to $\top$ (resp. $\bot$) under the complete assignment that combines $\as$ and $\mu$.
We alternatively represent an assignment $\as$ for $\pf$ as a cube.
A cube is called a \textit{minterm} of formula $\pf$ when it corresponds to a complete assignment over $\vf{\pf}$.
Given two Boolean formulas $\pf_1$ and $\pf_2$ over a same set $V$ of variables,
we write $\pf_1\models\pf_2$ if the following condition holds:
$\forall\as\in\av{V}.\as\models\pf_1\limply\as\models\pf_2$.


A Boolean formula $\pf$ is \textit{satisfiable} if it has a satisfying complete assignment.
Otherwise, $\pf$ is \textit{unsatisfiable}.
A Boolean formula $\pf$ is a \textit{tautology} if the following condition holds:
$\forall\as\in\av{\vf{\pf}}.\as\models\pf$.
The Boolean satisfiability problem asks to decide whether a Boolean formula is satisfiable or not.
It is a well-known NP-complete~\cite{Cook1971} problem.
We write $\sat{\pf}$ (resp. $\unsat{\pf}$) to indicate $\pf$ is satisfiable (resp. unsatisfiable).
A satisfying complete assignment of $\pf$ is also called a \textit{model} of $\pf$, which is denoted by $\model{\pf}$.

An $n$-variable \textit{Boolean function} is a mapping from $\booldom^n$ to $\booldom$.
Note that a Boolean formula $\pf$ induces a Boolean function with a domain $\av{\vf{\pf}}$.