\section{Propositional logic}
\label{sect:propositional-logic}

Boolean values \true and \false are represented by symbols $\top$ and $\bot$, respectively;
they are also treated as $1$ and $0$, respectively, in arithmetic computation.
Boolean connectives $\lnot, \lor, \land, \Rightarrow, \equiv$ are interpreted in their conventional semantics.
Given a set $V$ of variables,
an \textit{assignment} $\alpha$ is a mapping from each variable $x\in V$ to $\mathbb{B}=\{\top,\bot\}$,
and we denote the set of all assignments over $V$ by $\mathcal{A}(V)$.
An assignment $\alpha$ \textit{satisfies} a Boolean formula $\phi$ over a set $V$ of variables if $\phi$ yields $\top$ after substituting all occurrences of every variable $x\in V$ with its assigned value $\alpha(x)$ and simplifying $\phi$ under the semantics of Boolean connectives.
A Boolean formula $\phi$ over a set $V$ of variables is a \textit{tautology} if every assignment $\alpha\in \mathcal{A}(V)$ satisfies $\phi$.

We represent Boolean values \textsc{true} and \textsc{false} by symbols $\top$ and $\bot$, respectively.
In the sequel, a variable $x$ is assumed in the Boolean domain $\mathbb{B} = \{\top, \bot\}$.
A \emph{literal} is a variable (called a \emph{positive literal}) or the negation of a variable (called a \emph{negative literal}).
For a literal $l$, let $\mathtt{var}(l)$ denote the variable of $l$.
Boolean connectives $\neg, \vee, \wedge, \Rightarrow, \equiv$ are interpreted in their conventional meanings.

Let $\mathbb{B}=\{\top, \bot\}$, where $\top$ and $\bot$ denote logic \textsc{true} and \textsc{false}.
A \emph{literal} is a Boolean variable or its negation.
A \emph{clause} (resp. \emph{cube}) is a disjunction (resp. conjunction) of literals. A \emph{conjunctive normal form} (CNF) formula is a conjunction of clauses.
A Boolean formula $\phi$ over variables $X = \{x_1, \ldots, x_n\}$ induces a Boolean function mapping from $\mathbb{B}^n$ to $\mathbb{B}$.
The set of Boolean variables appearing in a Boolean formula $\phi$ is denoted as $\mathtt{vars}(\phi)$.
An \emph{assignment} $\tau$ on variables $X \subseteq \mathtt{vars}(\phi)$ of formula $\phi$ is a mapping from $X$ to $\mathbb{B}$.
An assignment $\tau$ is called a \emph{complete assignment} on $\mathtt{vars}(\phi)$ if $X=\mathtt{vars}(\phi)$; otherwise, it is called a \emph{partial assignment} on $\mathtt{vars}(\phi)$.
The formula of $\phi$ induced under the assignment $\tau$ on $X \subseteq \mathtt{vars}(\phi)$ is obtained by substituting every appearance of $x \in X$ in $\phi$ by $\tau(x)$, and is denoted as $\phi|_\tau$.
A complete assignment $\tau$ is called a \emph{satisfying} (resp. an \emph{unsatisfying}) \emph{(complete) assignment} for $\phi$ if $\phi|_\tau=\top$ (resp. $\phi|_\tau=\bot$).
Similarly, a partial assignment $\tau^+$ on $X \subset \mathtt{vars}(\phi)$ for $\phi$ is called a \emph{satisfying} (resp. an \emph{unsatisfying}) \emph{(partial) assignment} on $\mathtt{vars}(\phi)$ if for some (resp. every) assignment $\mu$ on $\mathtt{vars}(\phi)\setminus X$, $\phi$ valuates to $\top$ (resp. $\bot$) under the complete assignment obtained by combining $\tau$ and $\mu$.
In the sequel, we alternatively represent an assignment $\tau$ for $\phi$ as a cube.
A cube is called a \emph{minterm} when it corresponds to a complete assignment with respect to a specified set of variables.
A Boolean formula $\phi$ is called \emph{satisfiable} if there exists a satisfying complete assignment for $\phi$. We write $\mathtt{SAT}(\phi)=\top$ to denote $\phi$ is satisfiable.
A satisfying assignment of $\phi$ is also called a \emph{model} of $\phi$, denoted by $\phi.\mathrm{model}$.
On the other hand, if $\phi$ has no satisfying assignment, it is unsatisfiable and written as $\mathtt{SAT}(\phi)=\bot$.
%Therefore, a satisfying (resp. an unsatisfying) complete assignment for $\phi$ is also called a satisfying (resp. an unsatisfying) minterm of $\phi$, while a satisfying (resp. an unsatisfying) partial assignment is denoted as a satisfying (resp. an unsatisfying) cube, or a SAT (resp. an UNSAT) cube of $\phi$.

\subsection{Conjunctive normal form}
A \emph{clause} is a disjunction of literals.
A propositional Boolean formula $\phi$ is in \emph{Conjunctive Normal Form} (CNF) if $\phi$ is a conjunction of clauses.
A variable $x$ is said to be \emph{pure} in a formula if its appearances in the formula are all in the \emph{positive phase} $x$ or in the \emph{negative phase} $\neg x$.
A \emph{cube} is a conjunction of literals.
In the sequel, we assume propositional Boolean formulas are in CNF.

\subsection{Satisfiability}
A Boolean formula $\phi$ over a set of variables $X=\{x_1, \ldots, x_n\}$ defines a unique Boolean function $\mathbb{B}^n \rightarrow \mathbb{B}$.
Let $\mathtt{vars}(\phi)$ denote the set of variables appearing in a Boolean formula $\phi$.
An \emph{assignment} $\tau$ over a set of variables $X \subseteq \mathtt{vars}(\phi)$ for a formula $\phi$ is a mapping $\tau: X \rightarrow \mathbb{B}$.
An assignment $\tau: X \rightarrow \mathbb{B}$ is a \emph{complete assignment} for formula $\phi$ if $X=\mathtt{vars}(\phi)$; otherwise, i.e., $X \subset \mathtt{vars}(\phi)$, it is a \emph{partial assignment}.
Given a Boolean formula $\phi$ and an assignment $\tau$ over $\mathtt{vars}(\phi)$, the \emph{cofactor} of $\phi$ under $\tau$, denoted by $\phi|_{\tau}$, is derived by substituting every occurrence of each variable $x \in \mathtt{vars}(\phi)$ in $\phi$ by $\tau(x)$.
If $\phi|_{\tau}=\top$, we call $\tau$ a satisfying assignment of $\phi$.
The satisfiability problem of a Boolean formula $\phi$ asks whether or not $\phi$ has a satisfying assignment.
We write $\mathtt{SAT}(\phi)=\top$ to denote that $\phi$ is satisfiable.
A satisfying assignment of $\phi$ is also called a \emph{model} of $\phi$.
%denoted by $\phi.\mathrm{model}$.
On the other hand, if $\phi$ has no satisfying assignment, it is unsatisfiable and written as $\mathtt{SAT}(\phi)=\bot$.
Given two Boolean formulas $\phi$ and $\psi$, we write $\phi \models \psi$ if every satisfying assignment for $\phi$ also satisfies $\psi$.
In the sequel, we alternatively represent an assignment $\tau$ as a cube, a clause $C$ as a set of literals, and a CNF formula as a set of clauses.

\subsection{Minterm generalization}
Consider a CNF formula $\phi(X,Y)$, where $X$ and $Y$ are two disjoint sets of Boolean variables.
Given an assignment $\tau$ on $X$, if $\phi(X,Y)|_\tau$ is satisfiable (resp. unsatisfiable), $\tau$ is called a SAT (resp. an UNSAT) minterm of $\phi$ on $X$.
The generalization process of a SAT or an UNSAT minterm $\tau$ aims at expanding it to a cube $\tau^+$, while maintaining the satisfiability of $\phi(X,Y)|_{\tau^+}$ the same as $\phi(X,Y)|_\tau$.
\begin{example}\label{ex:assign}
    Consider formula $\phi(x_1, x_2, y_1, y_2)=x_1 \wedge (\neg x_2 \vee y_1 \vee y_2)$. The assignment $\tau = x_1 x_2$ on $X$, i.e., $\tau(x_1)=\top, \tau(x_2)=\top$, is a satisfying assignment, or a SAT minterm, of $\phi$ on $X$ as $\phi|_\tau$ is satisfiable by assignment $\mu = y_1y_2$.
    On the other hand, the partial assignment $\tau^+ = \neg x_1$, i.e., $\tau^+(x_1)=\bot$, is an unsatisfying partial assignment, or an UNSAT cube, of $\phi$ as $\phi|_{\tau^+}$ is unsatisfiable.
\end{example}

\subsubsection{Minimum Satisfying Assignment}
For a CNF formula $\phi(X,Y)$, let $\tau$ be a SAT minterm on $X$ and let $\mu$ be a satisfying assignment for $\phi(X,Y)|_\tau$ on $Y$. To generalize $\tau$ into a cube, one can find a subset of literals from $\tau$ and $\mu$ that are able to satisfy all clauses in $\phi$ while the number of literals taken from $\tau$ is as few as possible. If some literals in $\tau$ are irrelevant to the satisfiability, they can be dropped from $\tau$, thus expanding $\tau$ to a SAT cube $\tau^+$. If in the SAT cube $\tau^+$, the number of literals taking from $\tau$ is minimized, $\tau^+$ is called the \textit{minimum satisfying assignment}. The process of finding the minimum satisfying assignment is also known as finding the \textit{minimum hitting set}.

\subsubsection{Minimum Conflicting Assignment}
Given an UNSAT minterm $\tau$ of a CNF formula $\phi$,
modern SAT solvers, such as \minisat~\cite{Een2003Solver,Een2003Incremental},
are able to analyze the reason of unsatisfiability,
which is represented as a conjunction of literals in $\tau$ causing the conflict.
If some literals in $\tau$ are irrelevant to the conflict, they are dropped from $\tau$, thus expanding $\tau$ to an UNSAT cube $\tau^+$.
If the number of literals in the UNSAT cube $\tau^+$ is minimized, $\tau^+$ is called the \textit{minimum conflicting assignment}.
The process of finding the minimum conflicting assignment is also known as finding the \textit{minimum UNSAT core}.


\subsection{Binary decision diagram}