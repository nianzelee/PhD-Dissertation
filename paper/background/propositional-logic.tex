\section{Propositional logic}
\label{sect:propositional-logic}

We denote Boolean constants \false and \true by symbols $\bot$ and $\top$, respectively.
In arithmetic expressions, $\bot$ is interpreted as integer $0$ and $\top$ as integer $1$.
A variable $x$ that takes values from the Boolean domain $\booldom=\{\bot,\top\}$ is called a Boolean variable.
A \textit{literal} is a variable itself (called a \textit{positive} literal) or the negation of a variable (called a \textit{negative} literal).
For a literal $l$, let $\vl{l}$ denote the variable of $l$.
Boolean connectives $\lnot, \lor, \land, \limply, \equiv$ are used under their conventional semantics.
Over a finite set $V=\{x_1,\ldots,x_n\}$ of Boolean variables,
we define a \textit{well-formed formula} $\pf$ with the following Backus-Naur-form (BNF) grammar:
\begin{align}
    \pf ::= x\in V | \lnot\pf | (\pf\lor\pf) | (\pf\land\pf) | (\pf\limply\pf) | (\pf\equiv\pf).
\end{align}
Given a well-formed formula $\pf$, let $\vf{\pf}$ denote the set of Boolean variables appearing in $\pf$.
In the following, a variable is Boolean if not otherwise specified.
We shall consider well-formed formulas only and refer to them as \textit{Boolean formulas}.

\subsection{Conjunctive and disjunctive normal form}
Among various representations of a Boolean formula,
we are particularly interested in normal-form representations because their simplicity allows efficient analyses.

A Boolean formula is in \textit{conjunctive normal form} (CNF) if it is a conjunction of \textit{clauses},
where a clause is a disjunction of literals.
A Boolean formula is in \textit{disjunctive normal form} (DNF) if it is a disjunction of \textit{cubes},
where a cube is a conjunction of literals.
A variable $x$ is said to be \textit{pure} in a formula if its appearances in the formula are all positive literals or negative literals.
We alternatively treat a clause or a cube as a set of literals,
and a CNF (resp. DNF) formula as a set of clauses (resp. cubes).
In the rest of the dissertation, a Boolean formula is assumed to be given in CNF if not otherwise specified.

\subsection{Boolean satisfiability}
An \textit{assignment} $\as$ over a variable set $V$ is a mapping from $V$ to $\booldom$.
We denote the set of all assignments over $V$ by $\av{V}$.
Given a Boolean formula $\pf$, an assignment $\as$ over $\vf{\pf}$ \textit{satisfies} $\pf$,
denoted as $\as\models\pf$,
if substituting the occurrences of every $x\in\vf{\pf}$ with its assigned value $\as(x)$ yields $\top$
after simplifying $\pf$ under the semantics of Boolean connectives.
Formula $\pf$ is \textit{satisfiable} if $\exists\as\in\av{\vf{\pf}}.\as\models\pf$.
Otherwise, $\pf$ is \textit{unsatisfiable}.
A Boolean formula $\pf$ is a \textit{tautology} if $\forall\as\in\av{\vf{\pf}}.\as\models\pf$.
The satisfiability problem of Boolean formulas in CNF is NP-complete~\cite{Cook1971}.

\subsection{Binary decision diagram}