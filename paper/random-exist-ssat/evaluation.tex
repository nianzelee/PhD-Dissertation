\section{Evaluation}
\label{sect:ressat-evaluation}

We evaluated the proposed~\cref{alg:ressat} against
the state-of-the-art DPLL-based SSAT solver \dcssat~\cite{Majercik2005}
over three families of random-exist quantified SSAT formulas.
The proposed algorithm is implemented in the \texttt{C++} language inside the \abc~\cite{ABC} environment.
The SAT solver \minisat-2.2~\cite{Een2003Solver}
and the model counter \cachet~\cite{Sang2004}
are used as underlying computational engines.
Our prototyping implementation\footnote{Available at: \url{\ssatabcurl}} is named \ressat.
A bare version of \ressat without minterm generalization is called \ressatb in the experiments.
The solver \dcssat,
which is also implemented in the \texttt{C++} language,
was kindly provided by its author Majercik.

\subsection{Benchmark set}
Three sets of random-exist quantified SSAT formulas\footnote{Available at: \url{\ssatbenchmarkurl}},
including random $k$-CNF formulas,
planning formulas,
and probabilistic equivalence checking formulas,
were used in the evaluation.

\subsubsection{Random $k$-CNF formulas}
The random $k$-CNF formulas are generated using the CNF generator \cnfgen~\cite{Lauria2017CNFgen}.
They were generated as follows.
Let $k$ be the number of literals in a clause,
$n$ be the number of variables,
and $m$ be the number of clauses.
A collection of 700 CNF formulas is generated with the following parameter settings.
Let $k$ range from 3 to 9,
$n$ equal 10, 20, 30, 40, and 50,
and the clauses-to-variables ratio $\frac{m}{n}$ range from $k-1$ to $k+2$.
For each combination of parameters,
five formulas were sampled.
To convert the generated formulas to random-exist quantified SSAT formulas,
half of the variables in a formula are randomly quantified with probability $0.5$,
and the rest of the variables are existentially quantified.

\subsubsection{Planning formulas}
We use the \textit{strategic-company} problem~\cite{Cadoli1997} as an example
to evaluate the performance of the SSAT solvers over planning formulas.
We briefly describe the problem as follows.
Suppose a businessman owns $n$ companies that produce $m$ different kinds of products.
A company is \textit{strategic} if it is in a minimal set of companies that together produce all kinds of products.
The information about a company being strategic is valuable to the businessman.
Suppose the businessman considers selling out some companies upon a financial crisis,
but still hopes to produce every kind of products.
The businessman would prefer selling out a non-strategic company.
The problem becomes more complicated if the \textit{controlling relations} are taken into account.
If a company is \textit{controlled} by some other companies,
the company can be sold out only if some of its controlling companies is also sold out.
The problem to decide whether a company is strategic can be encoded as a forall-exist quantified QBF~\cite{Faber2005,Leone2006}.

We modify the QBFs encoding the strategic-company problem
to their SSAT variants by replacing the universal quantifiers in the original QBFs
with randomized ones with probabilities $0.5$.
These QBFs are taken from \texttt{QBFLIB}~\cite{Narizzano2006}.
The satisfying probability reflects the likelihood for a company to be strategic.
The QBFs that we experimented with have the following parameter settings:
$n$ equals 5, 10, 15, $\ldots$, 75, $m=3n$,
and the number of controlling relations equals 4, 9, 14, and 19.

\subsection{Experimental Setup}
Our experiments were performed on a machine with
one 2.2\,GHz CPU (Intel Xeon Silver 4210) with 40~processing units and 134616\,MB of RAM.
The operating system was Ubuntu~20.04 (64~bit),
using Linux~5.4.
The programs were compiled with \texttt{g++ 9.3.0}.
Each SSAT-solving task was limited to a CPU core,
a CPU time of \SI{15}{min},
and a memory usage of \SI{8}{GB}.
To achieve reliable benchmarking,
we used a benchmarking framework \benchexec\footnote{Available at: \url{\benchexecurl}}~\cite{Benchmarking-STTT}.
%%% TODO: fix the evaluation commit
%and \experimentRevision of \cpachecker for evaluation.

\subsection{Results}

\subsubsection{Random $k$-CNF formulas}

\begin{figure*}[t]
    \centering
    % This file is part of BenchExec, a framework for reliable benchmarking:
% https://github.com/sosy-lab/benchexec
%
% SPDX-FileCopyrightText: 2007-2020 Dirk Beyer <https://www.sosy-lab.org>
%
% SPDX-License-Identifier: Apache-2.0

% LaTeX code for a quantile plot
% Copy the tikzpicture environment to your own document
% and make sure the siunitx and pgfplots package are loaded,
% possibly with the options suggested here.
\documentclass{standalone}
\usepackage[
    group-digits=integer, group-minimum-digits=4, % group digits by thousands
    free-standing-units, unit-optional-argument, % easier input of numbers with units
]{siunitx}
\usepackage{pgfplots}
\pgfplotsset{
    compat=1.9,
    log ticks with fixed point, % no scientific notation in plots
    table/col sep=tab, % only tabs are column separators
    unbounded coords=jump, % better have skips in a plot than appear to be interpolating
    filter discard warning=false, % Don't complain about empty cells
}
\SendSettingsToPgf % use siunitx formatting settings in PGF, too

\begin{document}

\begin{tikzpicture}
    \begin{semilogyaxis}[
            % Which column to be taken from each file
            /pgfplots/table/y index=3,
            /pgfplots/table/header=false,
            % axis labels
            xlabel=n-th fastest result,
            ylabel=CPU time (\second),
            % axis ranges
            xmin=0,
            ymin=0.001,
            ymax=1000,
            mark repeat=100,
            % legend
            legend entries={\dcssat,\ressatb,\ressat},
            every axis legend/.append style={at={(1,0)}, anchor=south east, outer xsep=5pt, outer ysep=5pt,},
        ]
        \addplot+ table {random-exist-ssat/evaluation/csv/dcssat.default.Random.quantile.csv};
        \addplot+ table {random-exist-ssat/evaluation/csv/ressat.bare-Cachet.Random.quantile.csv};
        \addplot+ table {random-exist-ssat/evaluation/csv/ressat.minimize-Cachet.Random.quantile.csv};
    \end{semilogyaxis}
\end{tikzpicture}

\end{document}

    \caption{The quantile plot of random $k$-CNF formulas}
    \label{fig:ressat-random-quantile}
\end{figure*}

\Cref{fig:ressat-random-quantile} shows the quantile plot of the SSAT instances
derived from the random $k$-CNF formulas.

\subsubsection{Strategic-company formulas}

\begin{figure*}[t]
    \centering
    % This file is part of BenchExec, a framework for reliable benchmarking:
% https://github.com/sosy-lab/benchexec
%
% SPDX-FileCopyrightText: 2007-2020 Dirk Beyer <https://www.sosy-lab.org>
%
% SPDX-License-Identifier: Apache-2.0

% LaTeX code for a quantile plot
% Copy the tikzpicture environment to your own document
% and make sure the siunitx and pgfplots package are loaded,
% possibly with the options suggested here.
\documentclass{standalone}
\usepackage[
    group-digits=integer, group-minimum-digits=4, % group digits by thousands
    free-standing-units, unit-optional-argument, % easier input of numbers with units
]{siunitx}
\usepackage{pgfplots}
\pgfplotsset{
    compat=1.9,
    log ticks with fixed point, % no scientific notation in plots
    table/col sep=tab, % only tabs are column separators
    unbounded coords=jump, % better have skips in a plot than appear to be interpolating
    filter discard warning=false, % Don't complain about empty cells
}
\SendSettingsToPgf % use siunitx formatting settings in PGF, too

\begin{document}

\begin{tikzpicture}
    \begin{semilogyaxis}[
            % Which column to be taken from each file
            /pgfplots/table/y index=3,
            /pgfplots/table/header=false,
            % axis labels
            xlabel=n-th fastest result,
            ylabel=CPU time (\second),
            % axis ranges
            xmin=0,
            ymin=0.001,
            ymax=1000,
            mark repeat=10,
            % legend
            legend entries={\dcssat,\ressatb,\ressat},
            every axis legend/.append style={at={(1,0)}, anchor=south east, outer xsep=5pt, outer ysep=5pt,},
        ]
        \addplot+ table {random-exist-ssat/evaluation/csv/dcssat.default.Strategic.quantile.csv};
        \addplot+ table {random-exist-ssat/evaluation/csv/ressat.bare-Cachet.Strategic.quantile.csv};
        \addplot+ table {random-exist-ssat/evaluation/csv/ressat.minimize-Cachet.Strategic.quantile.csv};
    \end{semilogyaxis}
\end{tikzpicture}

\end{document}

    \caption{The quantile plot of strategic-company formulas}
    \label{fig:ressat-strategic-quantile}
\end{figure*}

\Cref{fig:ressat-strategic-quantile} shows the quantile plot of the SSAT instances
derived from the strategic-company QBFs.