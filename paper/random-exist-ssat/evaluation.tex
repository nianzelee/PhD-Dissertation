\newcommand{\nrandom}{\num{700}}
\newcommand{\napplication}{\num{120}}
\newcommand{\nstrategic}{\num{60}}
\newcommand{\npec}{\num{60}}

\section{Evaluation}
\label{sect:ressat-evaluation}

We evaluated the proposed~\cref{alg:ressat} against
the state-of-the-art DPLL-based SSAT solver \dcssat~\cite{Majercik2005}
over three families of random-exist quantified SSAT formulas.
The proposed algorithm is implemented in the \texttt{C++} language inside the \abc~\cite{ABC} environment.
The SAT solver \minisat-2.2~\cite{Een2003Solver}
and the model counter \cachet~\cite{Sang2004}
are used as underlying computational engines.
Our prototyping implementation\footnote{Available at: \url{\ssatabcurl}} is named \ressat.
A bare version of \ressat without minterm generalization is called \ressatb in the experiments.
We used \ssatABCRevision for the evaluation.
The solver \dcssat,
which is also implemented in the \texttt{C++} language,
was kindly provided by its author Majercik.

\subsection{Benchmark set}
We evaluated the SSAT solvers with both random and application formulas.
These formulas are hosted in a publicly available repository\footnote{Available at: \url{\ssatbenchmarkurl}}.
We used \ssatBenchRevision in the experiments.

\subsubsection{Random $k$-CNF formulas}
The random $k$-CNF formulas are generated using the CNF generator \cnfgen~\cite{Lauria2017CNFgen}.
Let $k$ be the number of literals in a clause,
$n$ be the number of variables of a formula,
and $m$ be the number of clauses of a formula.
The CNF formulas were generated with the following parameter settings.
Let $k$ range from 3 to 9,
$n$ equal 10, 20, 30, 40, and 50,
and the clauses-to-variables ratio $\frac{m}{n}$ range from $k-1$ to $k+2$.
For each combination of parameters,
five formulas were sampled.
As a result,~\nrandom~CNF formulas were generated.
To convert the generated formulas to random-exist quantified SSAT formulas,
half of the variables in a formula are randomly quantified with probability $0.5$,
and the rest of the variables are existentially quantified.

\subsubsection{Application formulas}
There are two families of application formulas.
The first family consists of formulas that encode a planning problem
\textit{Strategic-Company}~\cite{Cadoli1997}.
We briefly describe the problem as follows.
Suppose a businessman owns $n$ companies that produce $m$ different kinds of products.
A company is \textit{strategic} if it is in a minimal set of companies that together produce all kinds of products.
The information about a company being strategic is valuable to the businessman.
Suppose the businessman considers selling out some companies upon a financial crisis,
but still hopes to produce every kind of products.
The businessman would prefer selling out a non-strategic company.
The problem becomes more complicated if the \textit{controlling relations} are taken into account.
If a company is \textit{controlled} by some other companies,
the company can be sold out only if some of its controlling companies is also sold out.
The problem to decide whether a company is strategic can be encoded as a forall-exist quantified QBF~\cite{Faber2005,Leone2006}.

We modify the QBFs encoding the strategic-company problem
to their SSAT variants by replacing the universal quantifiers in the original QBFs
with randomized ones with probabilities $0.5$.
These QBFs are taken from \texttt{QBFLIB}~\cite{Narizzano2006}.
The satisfying probability reflects the likelihood for a company to be strategic.
The QBFs that we experimented with have the following parameter settings:
$n$ equals 5, 10, 15, $\ldots$, 75, $m=3n$,
and the number of controlling relations equals 4, 9, 14, and 19.
In total, there are~\nstrategic~formulas in the \textit{Strategic-Company} family.

The second family consists of formulas that encode
the average-case analysis of equivalence checking for probabilistic designs.
These PEC formulas are borrowed from~\cref{chap:prob-design-eval} and
their creation is briefly explained as follows.
For more details, please refer to~\cref{sect:prob-evaluation}.
A circuit with probabilistic errors is generated by
randomly assigning gates in its faultless counterpart to be erroneous.
Two parameters are specified to control the generation of probabilistic circuits.
The error rate $\er$ controls the probability of the occurrence of an error at a logic gate.
The defect rate $\dr$ controls the ratio of the number of erroneous gates to the total number of gates in a design.
These SSAT formulas are derived from the ISCAS\,'85~\cite{ISCAS85-benchmark}
and EPFL~\cite{EPFL-benchmark} benchmark suites with $\er=0.125$ and $\dr=0.01,0.1$.
In total, there are~\npec~formulas in the \textit{PEC} family.

\subsection{Experimental setup}
Our experiments were performed on a machine with~\machineSpec.
The operating system was~\osInfo.
The programs were compiled with~\compiler.
Each SSAT-solving task was limited to a CPU core,
a CPU time of~\timelimit,
and a memory usage of~\memlimit.
To achieve reliable benchmarking,
we used a benchmarking framework \benchexec\footnote{Available at: \url{\benchexecurl}}~\cite{Benchmarking-STTT},
and assumed~\measurement.

\subsection{Results}

\subsubsection{Random $k$-CNF formulas}

\cref{fig:ressat-quantile-random} shows the quantile plots regarding CPU time and memory usage
of the SSAT instances derived from the random $k$-CNF formulas.
A data point $(x,y)$ in a quantile plot indicates that
there are $x$ formulas processed by the respective algorithm within a resource constraint of $y$.
In~\cref{fig:ressat-quantile-cputime-random},
we observe that \ressat not only solved more formulas than \dcssat,
but was also more efficient.
Moreover, the minterm-generalization technique is crucial for the performance of \ressat,
as can be seen from the huge gap between \ressat and \ressatb.
On the other hand,
\cref{fig:ressat-quantile-memory-random} shows that the memory usage of \dcssat is about two orders of magnitude greater than that of \ressat for large formulas.
This can be attributed to the subformula caching of \dcssat.
Instead, \ressat stores information as SAT or UNSAT cubes,
which are more compact than subformulas.

\begin{figure*}[hp]
    \centering
    \subfloat[CPU time]{
        \includegraphics{random-exist-ssat/evaluation/plots/quantile-cputime-Random.pdf}
        \label{fig:ressat-quantile-cputime-random}
    }\\
    \subfloat[Memory usage]{
        \includegraphics{random-exist-ssat/evaluation/plots/quantile-memory-Random.pdf}
        \label{fig:ressat-quantile-memory-random}
    }
    \caption{Quantile plots of random $k$-CNF formulas}
    \label{fig:ressat-quantile-random}
\end{figure*}

\subsubsection{Application formulas}


% The following definition defines a command for each value.
% The command name is the concatenation of the first six arguments.
% To override this definition, define \StoreBenchExecResult with \newcommand before including this file.
% Arguments: benchmark name, run-set name, category, status, column name, statistic, value
\providecommand\StoreBenchExecResult[7]{\expandafter\newcommand\csname#1#2#3#4#5#6\endcsname{#7}}%
\StoreBenchExecResult{DcssatEr}{DefaultApplication}{Total}{}{Count}{}{160}%
\StoreBenchExecResult{DcssatEr}{DefaultApplication}{Total}{}{Cputime}{}{71834.861030650}%
\StoreBenchExecResult{DcssatEr}{DefaultApplication}{Total}{}{Cputime}{Avg}{448.9678814415625}%
\StoreBenchExecResult{DcssatEr}{DefaultApplication}{Total}{}{Cputime}{Median}{330.638654461}%
\StoreBenchExecResult{DcssatEr}{DefaultApplication}{Total}{}{Cputime}{Min}{0.005558095}%
\StoreBenchExecResult{DcssatEr}{DefaultApplication}{Total}{}{Cputime}{Max}{901.501094716}%
\StoreBenchExecResult{DcssatEr}{DefaultApplication}{Total}{}{Cputime}{Stdev}{410.6288898779780039960556041}%
\StoreBenchExecResult{DcssatEr}{DefaultApplication}{Total}{}{Walltime}{}{71838.407134809531468968}%
\StoreBenchExecResult{DcssatEr}{DefaultApplication}{Total}{}{Walltime}{Avg}{448.99004459255957168105}%
\StoreBenchExecResult{DcssatEr}{DefaultApplication}{Total}{}{Walltime}{Median}{330.6678233835846}%
\StoreBenchExecResult{DcssatEr}{DefaultApplication}{Total}{}{Walltime}{Min}{0.008783593773841858}%
\StoreBenchExecResult{DcssatEr}{DefaultApplication}{Total}{}{Walltime}{Max}{901.5298540564254}%
\StoreBenchExecResult{DcssatEr}{DefaultApplication}{Total}{}{Walltime}{Stdev}{410.6444476790752092161474212}%
\StoreBenchExecResult{DcssatEr}{DefaultApplication}{Error}{}{Count}{}{89}%
\StoreBenchExecResult{DcssatEr}{DefaultApplication}{Error}{}{Cputime}{}{68148.156984843}%
\StoreBenchExecResult{DcssatEr}{DefaultApplication}{Error}{}{Cputime}{Avg}{765.7096290431797752808988764}%
\StoreBenchExecResult{DcssatEr}{DefaultApplication}{Error}{}{Cputime}{Median}{901.000404586}%
\StoreBenchExecResult{DcssatEr}{DefaultApplication}{Error}{}{Cputime}{Min}{0.00690421}%
\StoreBenchExecResult{DcssatEr}{DefaultApplication}{Error}{}{Cputime}{Max}{901.501094716}%
\StoreBenchExecResult{DcssatEr}{DefaultApplication}{Error}{}{Cputime}{Stdev}{242.3007420382804066182760935}%
\StoreBenchExecResult{DcssatEr}{DefaultApplication}{Error}{}{Walltime}{}{68151.210960471071555125}%
\StoreBenchExecResult{DcssatEr}{DefaultApplication}{Error}{}{Walltime}{Avg}{765.7439433760794556755617978}%
\StoreBenchExecResult{DcssatEr}{DefaultApplication}{Error}{}{Walltime}{Median}{901.0384520078078}%
\StoreBenchExecResult{DcssatEr}{DefaultApplication}{Error}{}{Walltime}{Min}{0.014586311765015125}%
\StoreBenchExecResult{DcssatEr}{DefaultApplication}{Error}{}{Walltime}{Max}{901.5298540564254}%
\StoreBenchExecResult{DcssatEr}{DefaultApplication}{Error}{}{Walltime}{Stdev}{242.3076097998281048370405144}%
\StoreBenchExecResult{DcssatEr}{DefaultApplication}{Error}{Error}{Count}{}{1}%
\StoreBenchExecResult{DcssatEr}{DefaultApplication}{Error}{Error}{Cputime}{}{0.00690421}%
\StoreBenchExecResult{DcssatEr}{DefaultApplication}{Error}{Error}{Cputime}{Avg}{0.00690421}%
\StoreBenchExecResult{DcssatEr}{DefaultApplication}{Error}{Error}{Cputime}{Median}{0.00690421}%
\StoreBenchExecResult{DcssatEr}{DefaultApplication}{Error}{Error}{Cputime}{Min}{0.00690421}%
\StoreBenchExecResult{DcssatEr}{DefaultApplication}{Error}{Error}{Cputime}{Max}{0.00690421}%
\StoreBenchExecResult{DcssatEr}{DefaultApplication}{Error}{Error}{Cputime}{Stdev}{0.00000000000000}%
\StoreBenchExecResult{DcssatEr}{DefaultApplication}{Error}{Error}{Walltime}{}{0.014586311765015125}%
\StoreBenchExecResult{DcssatEr}{DefaultApplication}{Error}{Error}{Walltime}{Avg}{0.014586311765015125}%
\StoreBenchExecResult{DcssatEr}{DefaultApplication}{Error}{Error}{Walltime}{Median}{0.014586311765015125}%
\StoreBenchExecResult{DcssatEr}{DefaultApplication}{Error}{Error}{Walltime}{Min}{0.014586311765015125}%
\StoreBenchExecResult{DcssatEr}{DefaultApplication}{Error}{Error}{Walltime}{Max}{0.014586311765015125}%
\StoreBenchExecResult{DcssatEr}{DefaultApplication}{Error}{Error}{Walltime}{Stdev}{0.000000000000000000}%
\StoreBenchExecResult{DcssatEr}{DefaultApplication}{Error}{OutOfMemory}{Count}{}{25}%
\StoreBenchExecResult{DcssatEr}{DefaultApplication}{Error}{OutOfMemory}{Cputime}{}{11381.691933651}%
\StoreBenchExecResult{DcssatEr}{DefaultApplication}{Error}{OutOfMemory}{Cputime}{Avg}{455.26767734604}%
\StoreBenchExecResult{DcssatEr}{DefaultApplication}{Error}{OutOfMemory}{Cputime}{Median}{395.236226995}%
\StoreBenchExecResult{DcssatEr}{DefaultApplication}{Error}{OutOfMemory}{Cputime}{Min}{187.455423821}%
\StoreBenchExecResult{DcssatEr}{DefaultApplication}{Error}{OutOfMemory}{Cputime}{Max}{782.882748459}%
\StoreBenchExecResult{DcssatEr}{DefaultApplication}{Error}{OutOfMemory}{Cputime}{Stdev}{207.4075182889810651207325628}%
\StoreBenchExecResult{DcssatEr}{DefaultApplication}{Error}{OutOfMemory}{Walltime}{}{11382.41090900916614}%
\StoreBenchExecResult{DcssatEr}{DefaultApplication}{Error}{OutOfMemory}{Walltime}{Avg}{455.2964363603666456}%
\StoreBenchExecResult{DcssatEr}{DefaultApplication}{Error}{OutOfMemory}{Walltime}{Median}{395.26761069893837}%
\StoreBenchExecResult{DcssatEr}{DefaultApplication}{Error}{OutOfMemory}{Walltime}{Min}{187.46373896300793}%
\StoreBenchExecResult{DcssatEr}{DefaultApplication}{Error}{OutOfMemory}{Walltime}{Max}{782.9895220380276}%
\StoreBenchExecResult{DcssatEr}{DefaultApplication}{Error}{OutOfMemory}{Walltime}{Stdev}{207.4195162042323621035277689}%
\StoreBenchExecResult{DcssatEr}{DefaultApplication}{Error}{Timeout}{Count}{}{63}%
\StoreBenchExecResult{DcssatEr}{DefaultApplication}{Error}{Timeout}{Cputime}{}{56766.458146982}%
\StoreBenchExecResult{DcssatEr}{DefaultApplication}{Error}{Timeout}{Cputime}{Avg}{901.0548912219365079365079365}%
\StoreBenchExecResult{DcssatEr}{DefaultApplication}{Error}{Timeout}{Cputime}{Median}{901.024737657}%
\StoreBenchExecResult{DcssatEr}{DefaultApplication}{Error}{Timeout}{Cputime}{Min}{900.923230944}%
\StoreBenchExecResult{DcssatEr}{DefaultApplication}{Error}{Timeout}{Cputime}{Max}{901.501094716}%
\StoreBenchExecResult{DcssatEr}{DefaultApplication}{Error}{Timeout}{Cputime}{Stdev}{0.1000676663868040309241283849}%
\StoreBenchExecResult{DcssatEr}{DefaultApplication}{Error}{Timeout}{Walltime}{}{56768.7854651501404}%
\StoreBenchExecResult{DcssatEr}{DefaultApplication}{Error}{Timeout}{Walltime}{Avg}{901.0918327801609587301587302}%
\StoreBenchExecResult{DcssatEr}{DefaultApplication}{Error}{Timeout}{Walltime}{Median}{901.0557705014944}%
\StoreBenchExecResult{DcssatEr}{DefaultApplication}{Error}{Timeout}{Walltime}{Min}{901.0144192399457}%
\StoreBenchExecResult{DcssatEr}{DefaultApplication}{Error}{Timeout}{Walltime}{Max}{901.5298540564254}%
\StoreBenchExecResult{DcssatEr}{DefaultApplication}{Error}{Timeout}{Walltime}{Stdev}{0.09690864417508489761207137976}%
\StoreBenchExecResult{DcssatEr}{DefaultApplication}{Missing}{}{Count}{}{71}%
\StoreBenchExecResult{DcssatEr}{DefaultApplication}{Missing}{}{Cputime}{}{3686.704045807}%
\StoreBenchExecResult{DcssatEr}{DefaultApplication}{Missing}{}{Cputime}{Avg}{51.92540909587323943661971831}%
\StoreBenchExecResult{DcssatEr}{DefaultApplication}{Missing}{}{Cputime}{Median}{0.401376788}%
\StoreBenchExecResult{DcssatEr}{DefaultApplication}{Missing}{}{Cputime}{Min}{0.005558095}%
\StoreBenchExecResult{DcssatEr}{DefaultApplication}{Missing}{}{Cputime}{Max}{886.698473883}%
\StoreBenchExecResult{DcssatEr}{DefaultApplication}{Missing}{}{Cputime}{Stdev}{151.6030105157439752550361934}%
\StoreBenchExecResult{DcssatEr}{DefaultApplication}{Missing}{}{Walltime}{}{3687.196174338459913843}%
\StoreBenchExecResult{DcssatEr}{DefaultApplication}{Missing}{}{Walltime}{Avg}{51.93234048364028047666197183}%
\StoreBenchExecResult{DcssatEr}{DefaultApplication}{Missing}{}{Walltime}{Median}{0.402801631949842}%
\StoreBenchExecResult{DcssatEr}{DefaultApplication}{Missing}{}{Walltime}{Min}{0.008783593773841858}%
\StoreBenchExecResult{DcssatEr}{DefaultApplication}{Missing}{}{Walltime}{Max}{886.725481309928}%
\StoreBenchExecResult{DcssatEr}{DefaultApplication}{Missing}{}{Walltime}{Stdev}{151.6124987101443727971910575}%
\StoreBenchExecResult{DcssatEr}{DefaultApplication}{Missing}{Done}{Count}{}{71}%
\StoreBenchExecResult{DcssatEr}{DefaultApplication}{Missing}{Done}{Cputime}{}{3686.704045807}%
\StoreBenchExecResult{DcssatEr}{DefaultApplication}{Missing}{Done}{Cputime}{Avg}{51.92540909587323943661971831}%
\StoreBenchExecResult{DcssatEr}{DefaultApplication}{Missing}{Done}{Cputime}{Median}{0.401376788}%
\StoreBenchExecResult{DcssatEr}{DefaultApplication}{Missing}{Done}{Cputime}{Min}{0.005558095}%
\StoreBenchExecResult{DcssatEr}{DefaultApplication}{Missing}{Done}{Cputime}{Max}{886.698473883}%
\StoreBenchExecResult{DcssatEr}{DefaultApplication}{Missing}{Done}{Cputime}{Stdev}{151.6030105157439752550361934}%
\StoreBenchExecResult{DcssatEr}{DefaultApplication}{Missing}{Done}{Walltime}{}{3687.196174338459913843}%
\StoreBenchExecResult{DcssatEr}{DefaultApplication}{Missing}{Done}{Walltime}{Avg}{51.93234048364028047666197183}%
\StoreBenchExecResult{DcssatEr}{DefaultApplication}{Missing}{Done}{Walltime}{Median}{0.402801631949842}%
\StoreBenchExecResult{DcssatEr}{DefaultApplication}{Missing}{Done}{Walltime}{Min}{0.008783593773841858}%
\StoreBenchExecResult{DcssatEr}{DefaultApplication}{Missing}{Done}{Walltime}{Max}{886.725481309928}%
\StoreBenchExecResult{DcssatEr}{DefaultApplication}{Missing}{Done}{Walltime}{Stdev}{151.6124987101443727971910575}%
% The following definition defines a command for each value.
% The command name is the concatenation of the first six arguments.
% To override this definition, define \StoreBenchExecResult with \newcommand before including this file.
% Arguments: benchmark name, run-set name, category, status, column name, statistic, value
\providecommand\StoreBenchExecResult[7]{\expandafter\newcommand\csname#1#2#3#4#5#6\endcsname{#7}}%
\StoreBenchExecResult{Erssat}{DefaultBddApplication}{Total}{}{Count}{}{160}%
\StoreBenchExecResult{Erssat}{DefaultBddApplication}{Total}{}{Cputime}{}{95433.696587174}%
\StoreBenchExecResult{Erssat}{DefaultBddApplication}{Total}{}{Cputime}{Avg}{596.4606036698375}%
\StoreBenchExecResult{Erssat}{DefaultBddApplication}{Total}{}{Cputime}{Median}{900.9960360215}%
\StoreBenchExecResult{Erssat}{DefaultBddApplication}{Total}{}{Cputime}{Min}{0.040599129}%
\StoreBenchExecResult{Erssat}{DefaultBddApplication}{Total}{}{Cputime}{Max}{901.683691398}%
\StoreBenchExecResult{Erssat}{DefaultBddApplication}{Total}{}{Cputime}{Stdev}{409.8293182112904061356360349}%
\StoreBenchExecResult{Erssat}{DefaultBddApplication}{Total}{}{Walltime}{}{95439.408425660803482168}%
\StoreBenchExecResult{Erssat}{DefaultBddApplication}{Total}{}{Walltime}{Avg}{596.49630266038002176355}%
\StoreBenchExecResult{Erssat}{DefaultBddApplication}{Total}{}{Walltime}{Median}{901.0262540052645}%
\StoreBenchExecResult{Erssat}{DefaultBddApplication}{Total}{}{Walltime}{Min}{0.04257557261735201}%
\StoreBenchExecResult{Erssat}{DefaultBddApplication}{Total}{}{Walltime}{Max}{901.7038613380864}%
\StoreBenchExecResult{Erssat}{DefaultBddApplication}{Total}{}{Walltime}{Stdev}{409.8494678989227486890783493}%
\StoreBenchExecResult{Erssat}{DefaultBddApplication}{Error}{}{Count}{}{102}%
\StoreBenchExecResult{Erssat}{DefaultBddApplication}{Error}{}{Cputime}{}{90734.921256090}%
\StoreBenchExecResult{Erssat}{DefaultBddApplication}{Error}{}{Cputime}{Avg}{889.5580515302941176470588235}%
\StoreBenchExecResult{Erssat}{DefaultBddApplication}{Error}{}{Cputime}{Median}{901.022195085}%
\StoreBenchExecResult{Erssat}{DefaultBddApplication}{Error}{}{Cputime}{Min}{312.383213832}%
\StoreBenchExecResult{Erssat}{DefaultBddApplication}{Error}{}{Cputime}{Max}{901.683691398}%
\StoreBenchExecResult{Erssat}{DefaultBddApplication}{Error}{}{Cputime}{Stdev}{81.59444369319235672039935244}%
\StoreBenchExecResult{Erssat}{DefaultBddApplication}{Error}{}{Walltime}{}{90740.38185383751947}%
\StoreBenchExecResult{Erssat}{DefaultBddApplication}{Error}{}{Walltime}{Avg}{889.6115868023286222549019608}%
\StoreBenchExecResult{Erssat}{DefaultBddApplication}{Error}{}{Walltime}{Median}{901.13097573677075}%
\StoreBenchExecResult{Erssat}{DefaultBddApplication}{Error}{}{Walltime}{Min}{312.63300833757967}%
\StoreBenchExecResult{Erssat}{DefaultBddApplication}{Error}{}{Walltime}{Max}{901.7038613380864}%
\StoreBenchExecResult{Erssat}{DefaultBddApplication}{Error}{}{Walltime}{Stdev}{81.56961084934820997395047966}%
\StoreBenchExecResult{Erssat}{DefaultBddApplication}{Error}{SegmentationFault}{Count}{}{2}%
\StoreBenchExecResult{Erssat}{DefaultBddApplication}{Error}{SegmentationFault}{Cputime}{}{625.197866485}%
\StoreBenchExecResult{Erssat}{DefaultBddApplication}{Error}{SegmentationFault}{Cputime}{Avg}{312.5989332425}%
\StoreBenchExecResult{Erssat}{DefaultBddApplication}{Error}{SegmentationFault}{Cputime}{Median}{312.5989332425}%
\StoreBenchExecResult{Erssat}{DefaultBddApplication}{Error}{SegmentationFault}{Cputime}{Min}{312.383213832}%
\StoreBenchExecResult{Erssat}{DefaultBddApplication}{Error}{SegmentationFault}{Cputime}{Max}{312.814652653}%
\StoreBenchExecResult{Erssat}{DefaultBddApplication}{Error}{SegmentationFault}{Cputime}{Stdev}{0.21571941050000}%
\StoreBenchExecResult{Erssat}{DefaultBddApplication}{Error}{SegmentationFault}{Walltime}{}{625.65604413859547}%
\StoreBenchExecResult{Erssat}{DefaultBddApplication}{Error}{SegmentationFault}{Walltime}{Avg}{312.828022069297735}%
\StoreBenchExecResult{Erssat}{DefaultBddApplication}{Error}{SegmentationFault}{Walltime}{Median}{312.828022069297735}%
\StoreBenchExecResult{Erssat}{DefaultBddApplication}{Error}{SegmentationFault}{Walltime}{Min}{312.63300833757967}%
\StoreBenchExecResult{Erssat}{DefaultBddApplication}{Error}{SegmentationFault}{Walltime}{Max}{313.0230358010158}%
\StoreBenchExecResult{Erssat}{DefaultBddApplication}{Error}{SegmentationFault}{Walltime}{Stdev}{0.1950137317180650000000000000}%
\StoreBenchExecResult{Erssat}{DefaultBddApplication}{Error}{Timeout}{Count}{}{100}%
\StoreBenchExecResult{Erssat}{DefaultBddApplication}{Error}{Timeout}{Cputime}{}{90109.723389605}%
\StoreBenchExecResult{Erssat}{DefaultBddApplication}{Error}{Timeout}{Cputime}{Avg}{901.09723389605}%
\StoreBenchExecResult{Erssat}{DefaultBddApplication}{Error}{Timeout}{Cputime}{Median}{901.0225239675}%
\StoreBenchExecResult{Erssat}{DefaultBddApplication}{Error}{Timeout}{Cputime}{Min}{900.896737756}%
\StoreBenchExecResult{Erssat}{DefaultBddApplication}{Error}{Timeout}{Cputime}{Max}{901.683691398}%
\StoreBenchExecResult{Erssat}{DefaultBddApplication}{Error}{Timeout}{Cputime}{Stdev}{0.1271320476503559861839442074}%
\StoreBenchExecResult{Erssat}{DefaultBddApplication}{Error}{Timeout}{Walltime}{}{90114.7258096989240}%
\StoreBenchExecResult{Erssat}{DefaultBddApplication}{Error}{Timeout}{Walltime}{Avg}{901.14725809698924}%
\StoreBenchExecResult{Erssat}{DefaultBddApplication}{Error}{Timeout}{Walltime}{Median}{901.1404981263913}%
\StoreBenchExecResult{Erssat}{DefaultBddApplication}{Error}{Timeout}{Walltime}{Min}{901.0066848965362}%
\StoreBenchExecResult{Erssat}{DefaultBddApplication}{Error}{Timeout}{Walltime}{Max}{901.7038613380864}%
\StoreBenchExecResult{Erssat}{DefaultBddApplication}{Error}{Timeout}{Walltime}{Stdev}{0.1239300123881479746725328914}%
\StoreBenchExecResult{Erssat}{DefaultBddApplication}{Missing}{}{Count}{}{58}%
\StoreBenchExecResult{Erssat}{DefaultBddApplication}{Missing}{}{Cputime}{}{4698.775331084}%
\StoreBenchExecResult{Erssat}{DefaultBddApplication}{Missing}{}{Cputime}{Avg}{81.01336777731034482758620690}%
\StoreBenchExecResult{Erssat}{DefaultBddApplication}{Missing}{}{Cputime}{Median}{1.5293811165}%
\StoreBenchExecResult{Erssat}{DefaultBddApplication}{Missing}{}{Cputime}{Min}{0.040599129}%
\StoreBenchExecResult{Erssat}{DefaultBddApplication}{Missing}{}{Cputime}{Max}{864.799801094}%
\StoreBenchExecResult{Erssat}{DefaultBddApplication}{Missing}{}{Cputime}{Stdev}{186.7289887709010217860957077}%
\StoreBenchExecResult{Erssat}{DefaultBddApplication}{Missing}{}{Walltime}{}{4699.026571823284012168}%
\StoreBenchExecResult{Erssat}{DefaultBddApplication}{Missing}{}{Walltime}{Avg}{81.01769951419455193393103448}%
\StoreBenchExecResult{Erssat}{DefaultBddApplication}{Missing}{}{Walltime}{Median}{1.53087391564622525}%
\StoreBenchExecResult{Erssat}{DefaultBddApplication}{Missing}{}{Walltime}{Min}{0.04257557261735201}%
\StoreBenchExecResult{Erssat}{DefaultBddApplication}{Missing}{}{Walltime}{Max}{864.8248462686315}%
\StoreBenchExecResult{Erssat}{DefaultBddApplication}{Missing}{}{Walltime}{Stdev}{186.7342438000275984958325587}%
\StoreBenchExecResult{Erssat}{DefaultBddApplication}{Missing}{Done}{Count}{}{58}%
\StoreBenchExecResult{Erssat}{DefaultBddApplication}{Missing}{Done}{Cputime}{}{4698.775331084}%
\StoreBenchExecResult{Erssat}{DefaultBddApplication}{Missing}{Done}{Cputime}{Avg}{81.01336777731034482758620690}%
\StoreBenchExecResult{Erssat}{DefaultBddApplication}{Missing}{Done}{Cputime}{Median}{1.5293811165}%
\StoreBenchExecResult{Erssat}{DefaultBddApplication}{Missing}{Done}{Cputime}{Min}{0.040599129}%
\StoreBenchExecResult{Erssat}{DefaultBddApplication}{Missing}{Done}{Cputime}{Max}{864.799801094}%
\StoreBenchExecResult{Erssat}{DefaultBddApplication}{Missing}{Done}{Cputime}{Stdev}{186.7289887709010217860957077}%
\StoreBenchExecResult{Erssat}{DefaultBddApplication}{Missing}{Done}{Walltime}{}{4699.026571823284012168}%
\StoreBenchExecResult{Erssat}{DefaultBddApplication}{Missing}{Done}{Walltime}{Avg}{81.01769951419455193393103448}%
\StoreBenchExecResult{Erssat}{DefaultBddApplication}{Missing}{Done}{Walltime}{Median}{1.53087391564622525}%
\StoreBenchExecResult{Erssat}{DefaultBddApplication}{Missing}{Done}{Walltime}{Min}{0.04257557261735201}%
\StoreBenchExecResult{Erssat}{DefaultBddApplication}{Missing}{Done}{Walltime}{Max}{864.8248462686315}%
\StoreBenchExecResult{Erssat}{DefaultBddApplication}{Missing}{Done}{Walltime}{Stdev}{186.7342438000275984958325587}%
% The following definition defines a command for each value.
% The command name is the concatenation of the first six arguments.
% To override this definition, define \StoreBenchExecResult with \newcommand before including this file.
% Arguments: benchmark name, run-set name, category, status, column name, statistic, value
\providecommand\StoreBenchExecResult[7]{\expandafter\newcommand\csname#1#2#3#4#5#6\endcsname{#7}}%
\StoreBenchExecResult{Erssat}{BareBddApplication}{Total}{}{Count}{}{160}%
\StoreBenchExecResult{Erssat}{BareBddApplication}{Total}{}{Cputime}{}{87131.618210014}%
\StoreBenchExecResult{Erssat}{BareBddApplication}{Total}{}{Cputime}{Avg}{544.5726138125875}%
\StoreBenchExecResult{Erssat}{BareBddApplication}{Total}{}{Cputime}{Median}{900.9923350185}%
\StoreBenchExecResult{Erssat}{BareBddApplication}{Total}{}{Cputime}{Min}{0.043293474}%
\StoreBenchExecResult{Erssat}{BareBddApplication}{Total}{}{Cputime}{Max}{901.674077293}%
\StoreBenchExecResult{Erssat}{BareBddApplication}{Total}{}{Cputime}{Stdev}{429.8298373702258205108455494}%
\StoreBenchExecResult{Erssat}{BareBddApplication}{Total}{}{Walltime}{}{87137.005280911922004714}%
\StoreBenchExecResult{Erssat}{BareBddApplication}{Total}{}{Walltime}{Avg}{544.6062830056995125294625}%
\StoreBenchExecResult{Erssat}{BareBddApplication}{Total}{}{Walltime}{Median}{901.0144369280897}%
\StoreBenchExecResult{Erssat}{BareBddApplication}{Total}{}{Walltime}{Min}{0.0445504579693079}%
\StoreBenchExecResult{Erssat}{BareBddApplication}{Total}{}{Walltime}{Max}{901.6921367570758}%
\StoreBenchExecResult{Erssat}{BareBddApplication}{Total}{}{Walltime}{Stdev}{429.8456916725929113683152270}%
\StoreBenchExecResult{Erssat}{BareBddApplication}{Error}{}{Count}{}{98}%
\StoreBenchExecResult{Erssat}{BareBddApplication}{Error}{}{Cputime}{}{84706.308552562}%
\StoreBenchExecResult{Erssat}{BareBddApplication}{Error}{}{Cputime}{Avg}{864.3500872710408163265306122}%
\StoreBenchExecResult{Erssat}{BareBddApplication}{Error}{}{Cputime}{Median}{901.013068872}%
\StoreBenchExecResult{Erssat}{BareBddApplication}{Error}{}{Cputime}{Min}{0.761268082}%
\StoreBenchExecResult{Erssat}{BareBddApplication}{Error}{}{Cputime}{Max}{901.674077293}%
\StoreBenchExecResult{Erssat}{BareBddApplication}{Error}{}{Cputime}{Stdev}{178.0642449299897837836263060}%
\StoreBenchExecResult{Erssat}{BareBddApplication}{Error}{}{Walltime}{}{84711.5119971176604861}%
\StoreBenchExecResult{Erssat}{BareBddApplication}{Error}{}{Walltime}{Avg}{864.4031836440577600622448980}%
\StoreBenchExecResult{Erssat}{BareBddApplication}{Error}{}{Walltime}{Median}{901.1240388415754}%
\StoreBenchExecResult{Erssat}{BareBddApplication}{Error}{}{Walltime}{Min}{0.9539869418367743}%
\StoreBenchExecResult{Erssat}{BareBddApplication}{Error}{}{Walltime}{Max}{901.6921367570758}%
\StoreBenchExecResult{Erssat}{BareBddApplication}{Error}{}{Walltime}{Stdev}{178.0357486274157922587281994}%
\StoreBenchExecResult{Erssat}{BareBddApplication}{Error}{SegmentationFault}{Count}{}{4}%
\StoreBenchExecResult{Erssat}{BareBddApplication}{Error}{SegmentationFault}{Cputime}{}{4.607598963}%
\StoreBenchExecResult{Erssat}{BareBddApplication}{Error}{SegmentationFault}{Cputime}{Avg}{1.15189974075}%
\StoreBenchExecResult{Erssat}{BareBddApplication}{Error}{SegmentationFault}{Cputime}{Median}{1.099821245}%
\StoreBenchExecResult{Erssat}{BareBddApplication}{Error}{SegmentationFault}{Cputime}{Min}{0.761268082}%
\StoreBenchExecResult{Erssat}{BareBddApplication}{Error}{SegmentationFault}{Cputime}{Max}{1.646688391}%
\StoreBenchExecResult{Erssat}{BareBddApplication}{Error}{SegmentationFault}{Cputime}{Stdev}{0.3937375208343058270030168665}%
\StoreBenchExecResult{Erssat}{BareBddApplication}{Error}{SegmentationFault}{Walltime}{}{5.3725353293120861}%
\StoreBenchExecResult{Erssat}{BareBddApplication}{Error}{SegmentationFault}{Walltime}{Avg}{1.343133832328021525}%
\StoreBenchExecResult{Erssat}{BareBddApplication}{Error}{SegmentationFault}{Walltime}{Median}{1.2907491861842573}%
\StoreBenchExecResult{Erssat}{BareBddApplication}{Error}{SegmentationFault}{Walltime}{Min}{0.9539869418367743}%
\StoreBenchExecResult{Erssat}{BareBddApplication}{Error}{SegmentationFault}{Walltime}{Max}{1.8370500151067972}%
\StoreBenchExecResult{Erssat}{BareBddApplication}{Error}{SegmentationFault}{Walltime}{Stdev}{0.3923080483709211947621329156}%
\StoreBenchExecResult{Erssat}{BareBddApplication}{Error}{Timeout}{Count}{}{94}%
\StoreBenchExecResult{Erssat}{BareBddApplication}{Error}{Timeout}{Cputime}{}{84701.700953599}%
\StoreBenchExecResult{Erssat}{BareBddApplication}{Error}{Timeout}{Cputime}{Avg}{901.0819250382872340425531915}%
\StoreBenchExecResult{Erssat}{BareBddApplication}{Error}{Timeout}{Cputime}{Median}{901.015790674}%
\StoreBenchExecResult{Erssat}{BareBddApplication}{Error}{Timeout}{Cputime}{Min}{900.890174616}%
\StoreBenchExecResult{Erssat}{BareBddApplication}{Error}{Timeout}{Cputime}{Max}{901.674077293}%
\StoreBenchExecResult{Erssat}{BareBddApplication}{Error}{Timeout}{Cputime}{Stdev}{0.1173521311118864200217232826}%
\StoreBenchExecResult{Erssat}{BareBddApplication}{Error}{Timeout}{Walltime}{}{84706.1394617883484}%
\StoreBenchExecResult{Erssat}{BareBddApplication}{Error}{Timeout}{Walltime}{Avg}{901.1291432105143446808510638}%
\StoreBenchExecResult{Erssat}{BareBddApplication}{Error}{Timeout}{Walltime}{Median}{901.1370040657930}%
\StoreBenchExecResult{Erssat}{BareBddApplication}{Error}{Timeout}{Walltime}{Min}{901.0070397015661}%
\StoreBenchExecResult{Erssat}{BareBddApplication}{Error}{Timeout}{Walltime}{Max}{901.6921367570758}%
\StoreBenchExecResult{Erssat}{BareBddApplication}{Error}{Timeout}{Walltime}{Stdev}{0.1164837688602781020101066434}%
\StoreBenchExecResult{Erssat}{BareBddApplication}{Missing}{}{Count}{}{62}%
\StoreBenchExecResult{Erssat}{BareBddApplication}{Missing}{}{Cputime}{}{2425.309657452}%
\StoreBenchExecResult{Erssat}{BareBddApplication}{Missing}{}{Cputime}{Avg}{39.11789770083870967741935484}%
\StoreBenchExecResult{Erssat}{BareBddApplication}{Missing}{}{Cputime}{Median}{1.846133201}%
\StoreBenchExecResult{Erssat}{BareBddApplication}{Missing}{}{Cputime}{Min}{0.043293474}%
\StoreBenchExecResult{Erssat}{BareBddApplication}{Missing}{}{Cputime}{Max}{495.99237704}%
\StoreBenchExecResult{Erssat}{BareBddApplication}{Missing}{}{Cputime}{Stdev}{97.71843544911616823158405099}%
\StoreBenchExecResult{Erssat}{BareBddApplication}{Missing}{}{Walltime}{}{2425.493283794261518614}%
\StoreBenchExecResult{Erssat}{BareBddApplication}{Missing}{}{Walltime}{Avg}{39.12085941603647610667741935}%
\StoreBenchExecResult{Erssat}{BareBddApplication}{Missing}{}{Walltime}{Median}{1.847707605920732}%
\StoreBenchExecResult{Erssat}{BareBddApplication}{Missing}{}{Walltime}{Min}{0.0445504579693079}%
\StoreBenchExecResult{Erssat}{BareBddApplication}{Missing}{}{Walltime}{Max}{496.00512275565416}%
\StoreBenchExecResult{Erssat}{BareBddApplication}{Missing}{}{Walltime}{Stdev}{97.72114502500470110460819337}%
\StoreBenchExecResult{Erssat}{BareBddApplication}{Missing}{Done}{Count}{}{62}%
\StoreBenchExecResult{Erssat}{BareBddApplication}{Missing}{Done}{Cputime}{}{2425.309657452}%
\StoreBenchExecResult{Erssat}{BareBddApplication}{Missing}{Done}{Cputime}{Avg}{39.11789770083870967741935484}%
\StoreBenchExecResult{Erssat}{BareBddApplication}{Missing}{Done}{Cputime}{Median}{1.846133201}%
\StoreBenchExecResult{Erssat}{BareBddApplication}{Missing}{Done}{Cputime}{Min}{0.043293474}%
\StoreBenchExecResult{Erssat}{BareBddApplication}{Missing}{Done}{Cputime}{Max}{495.99237704}%
\StoreBenchExecResult{Erssat}{BareBddApplication}{Missing}{Done}{Cputime}{Stdev}{97.71843544911616823158405099}%
\StoreBenchExecResult{Erssat}{BareBddApplication}{Missing}{Done}{Walltime}{}{2425.493283794261518614}%
\StoreBenchExecResult{Erssat}{BareBddApplication}{Missing}{Done}{Walltime}{Avg}{39.12085941603647610667741935}%
\StoreBenchExecResult{Erssat}{BareBddApplication}{Missing}{Done}{Walltime}{Median}{1.847707605920732}%
\StoreBenchExecResult{Erssat}{BareBddApplication}{Missing}{Done}{Walltime}{Min}{0.0445504579693079}%
\StoreBenchExecResult{Erssat}{BareBddApplication}{Missing}{Done}{Walltime}{Max}{496.00512275565416}%
\StoreBenchExecResult{Erssat}{BareBddApplication}{Missing}{Done}{Walltime}{Stdev}{97.72114502500470110460819337}%
\ifdefined\DcssatErDefaultApplicationTotalCount\else\edef\DcssatErDefaultApplicationTotalCount{0}\fi
\ifdefined\DcssatErDefaultApplicationCorrectCount\else\edef\DcssatErDefaultApplicationCorrectCount{0}\fi
\ifdefined\DcssatErDefaultApplicationCorrectTrueCount\else\edef\DcssatErDefaultApplicationCorrectTrueCount{0}\fi
\ifdefined\DcssatErDefaultApplicationCorrectFalseCount\else\edef\DcssatErDefaultApplicationCorrectFalseCount{0}\fi
\ifdefined\DcssatErDefaultApplicationWrongTrueCount\else\edef\DcssatErDefaultApplicationWrongTrueCount{0}\fi
\ifdefined\DcssatErDefaultApplicationWrongFalseCount\else\edef\DcssatErDefaultApplicationWrongFalseCount{0}\fi
\ifdefined\DcssatErDefaultApplicationErrorTimeoutCount\else\edef\DcssatErDefaultApplicationErrorTimeoutCount{0}\fi
\ifdefined\DcssatErDefaultApplicationErrorOutOfMemoryCount\else\edef\DcssatErDefaultApplicationErrorOutOfMemoryCount{0}\fi
\ifdefined\DcssatErDefaultApplicationCorrectCputime\else\edef\DcssatErDefaultApplicationCorrectCputime{0}\fi
\ifdefined\DcssatErDefaultApplicationCorrectCputimeAvg\else\edef\DcssatErDefaultApplicationCorrectCputimeAvg{None}\fi
\ifdefined\DcssatErDefaultApplicationCorrectWalltime\else\edef\DcssatErDefaultApplicationCorrectWalltime{0}\fi
\ifdefined\DcssatErDefaultApplicationCorrectWalltimeAvg\else\edef\DcssatErDefaultApplicationCorrectWalltimeAvg{None}\fi
\ifdefined\ErssatDefaultBddApplicationTotalCount\else\edef\ErssatDefaultBddApplicationTotalCount{0}\fi
\ifdefined\ErssatDefaultBddApplicationCorrectCount\else\edef\ErssatDefaultBddApplicationCorrectCount{0}\fi
\ifdefined\ErssatDefaultBddApplicationCorrectTrueCount\else\edef\ErssatDefaultBddApplicationCorrectTrueCount{0}\fi
\ifdefined\ErssatDefaultBddApplicationCorrectFalseCount\else\edef\ErssatDefaultBddApplicationCorrectFalseCount{0}\fi
\ifdefined\ErssatDefaultBddApplicationWrongTrueCount\else\edef\ErssatDefaultBddApplicationWrongTrueCount{0}\fi
\ifdefined\ErssatDefaultBddApplicationWrongFalseCount\else\edef\ErssatDefaultBddApplicationWrongFalseCount{0}\fi
\ifdefined\ErssatDefaultBddApplicationErrorTimeoutCount\else\edef\ErssatDefaultBddApplicationErrorTimeoutCount{0}\fi
\ifdefined\ErssatDefaultBddApplicationErrorOutOfMemoryCount\else\edef\ErssatDefaultBddApplicationErrorOutOfMemoryCount{0}\fi
\ifdefined\ErssatDefaultBddApplicationCorrectCputime\else\edef\ErssatDefaultBddApplicationCorrectCputime{0}\fi
\ifdefined\ErssatDefaultBddApplicationCorrectCputimeAvg\else\edef\ErssatDefaultBddApplicationCorrectCputimeAvg{None}\fi
\ifdefined\ErssatDefaultBddApplicationCorrectWalltime\else\edef\ErssatDefaultBddApplicationCorrectWalltime{0}\fi
\ifdefined\ErssatDefaultBddApplicationCorrectWalltimeAvg\else\edef\ErssatDefaultBddApplicationCorrectWalltimeAvg{None}\fi
\ifdefined\ErssatBareBddApplicationTotalCount\else\edef\ErssatBareBddApplicationTotalCount{0}\fi
\ifdefined\ErssatBareBddApplicationCorrectCount\else\edef\ErssatBareBddApplicationCorrectCount{0}\fi
\ifdefined\ErssatBareBddApplicationCorrectTrueCount\else\edef\ErssatBareBddApplicationCorrectTrueCount{0}\fi
\ifdefined\ErssatBareBddApplicationCorrectFalseCount\else\edef\ErssatBareBddApplicationCorrectFalseCount{0}\fi
\ifdefined\ErssatBareBddApplicationWrongTrueCount\else\edef\ErssatBareBddApplicationWrongTrueCount{0}\fi
\ifdefined\ErssatBareBddApplicationWrongFalseCount\else\edef\ErssatBareBddApplicationWrongFalseCount{0}\fi
\ifdefined\ErssatBareBddApplicationErrorTimeoutCount\else\edef\ErssatBareBddApplicationErrorTimeoutCount{0}\fi
\ifdefined\ErssatBareBddApplicationErrorOutOfMemoryCount\else\edef\ErssatBareBddApplicationErrorOutOfMemoryCount{0}\fi
\ifdefined\ErssatBareBddApplicationCorrectCputime\else\edef\ErssatBareBddApplicationCorrectCputime{0}\fi
\ifdefined\ErssatBareBddApplicationCorrectCputimeAvg\else\edef\ErssatBareBddApplicationCorrectCputimeAvg{None}\fi
\ifdefined\ErssatBareBddApplicationCorrectWalltime\else\edef\ErssatBareBddApplicationCorrectWalltime{0}\fi
\ifdefined\ErssatBareBddApplicationCorrectWalltimeAvg\else\edef\ErssatBareBddApplicationCorrectWalltimeAvg{None}\fi
\edef\DcssatErDefaultApplicationErrorOtherInconclusiveCount{\the\numexpr \DcssatErDefaultApplicationTotalCount - \DcssatErDefaultApplicationMissingCount - \DcssatErDefaultApplicationErrorTimeoutCount - \DcssatErDefaultApplicationErrorOutOfMemoryCount \relax}
\edef\ErssatDefaultBddApplicationErrorOtherInconclusiveCount{\the\numexpr \ErssatDefaultBddApplicationTotalCount - \ErssatDefaultBddApplicationMissingCount - \ErssatDefaultBddApplicationErrorTimeoutCount - \ErssatDefaultBddApplicationErrorOutOfMemoryCount \relax}
\edef\ErssatBareBddApplicationErrorOtherInconclusiveCount{\the\numexpr \ErssatBareBddApplicationTotalCount - \ErssatBareBddApplicationMissingCount - \ErssatBareBddApplicationErrorTimeoutCount - \ErssatBareBddApplicationErrorOutOfMemoryCount \relax}
% Commands for application formulas of ER-SSAT
\newcommand{\dcssatToiletA}{44}
\newcommand{\dcssatconformant}{1}
\newcommand{\dcssatcastle}{21}
\newcommand{\dcssatMaxCount}{2}
\newcommand{\dcssatMPEC}{3}
\newcommand{\erssatbToiletA}{45}
\newcommand{\erssatbconformant}{1}
\newcommand{\erssatbcastle}{14}
\newcommand{\erssatbMaxCount}{1}
\newcommand{\erssatbMPEC}{1}
\newcommand{\erssatToiletA}{38}
\newcommand{\erssatconformant}{2}
\newcommand{\erssatcastle}{13}
\newcommand{\erssatMaxCount}{3}
\newcommand{\erssatMPEC}{2}

\begin{table}[ht]
    \centering
    \caption{Summary of the results for~\nstrategic~strategic-company formulas}
    \label{tbl:random-exist-ssat-strategic}
    \begin{tabular}{l|ccc}
        \toprule
        Algorithm          & {\dcssat}                                                       & {\ressat} & {\ressatb} \\
        \midrule
        Solved formulas    & \num{\DcssatReDefaultStrategicMissingCount}
                           & \num{\RessatMinimizeCachetStrategicMissingCount}
                           & \num{\RessatBareCachetStrategicMissingCount}                                             \\
        Timeouts           & \num{\DcssatReDefaultStrategicErrorTimeoutCount}
                           & \num{\RessatMinimizeCachetStrategicErrorTimeoutCount}
                           & \num{\RessatBareCachetStrategicErrorTimeoutCount}                                        \\
        Out of memory      & \num{\DcssatReDefaultStrategicErrorOutOfMemoryCount}
                           & \num{\RessatMinimizeCachetStrategicErrorOutOfMemoryCount}
                           & \num{\RessatBareCachetStrategicErrorOutOfMemoryCount}                                    \\
        Other inconclusive & \num{\DcssatReDefaultStrategicErrorOtherInconclusiveCount}
                           & \num{\RessatMinimizeCachetStrategicErrorOtherInconclusiveCount}
                           & \num{\RessatBareCachetStrategicErrorOtherInconclusiveCount}                              \\
        \bottomrule
    \end{tabular}
\end{table}

The solving results for the \textit{Strategic-Company} family are summarized in~\cref{tbl:random-exist-ssat-strategic}.
Observe that \ressat successfully solved all formulas in this family,
while \dcssat and \ressatb only solved \num{28} and \num{12} formulas, respectively.
The results show that \ressat also works well over structured instances from AI applications.

\cref{fig:ressat-quantile-strategic} shows the quantile plots for the \textit{Strategic-Company} family.
The quantile plots show that \ressat is not only effective but also efficient
in terms of CPU time and memory usage.
The performance gap between \ressat and \ressatb again confirms the importance of the minterm-generalization technique.
\cref{fig:ressat-scatter-strategic} shows the scatter plots,
which indicate the great improvement for the run-time efficiency of \ressat.

\begin{figure*}[hp]
    \centering
    \subfloat[CPU time]{
        \includegraphics{random-exist-ssat/evaluation/plots/quantile-cputime-Strategic.pdf}
        \label{fig:ressat-quantile-cputime-strategic}
    }\\
    \subfloat[Memory usage]{
        \includegraphics{random-exist-ssat/evaluation/plots/quantile-memory-Strategic.pdf}
        \label{fig:ressat-quantile-memory-strategic}
    }
    \caption{Quantile plots of strategic-company formulas}
    \label{fig:ressat-quantile-strategic}
\end{figure*}

\begin{figure*}[hp]
    \centering
    \subfloat[\ressatb]{
        \includegraphics{random-exist-ssat/evaluation/plots/scatter-ressat.pdf}
        \label{fig:ressat-scatter-cputime-strategic}
    }\\
    \subfloat[\dcssat]{
        \includegraphics{random-exist-ssat/evaluation/plots/scatter-dcssat.pdf}
        \label{fig:dcssat-scatter-cputime-strategic}
    }
    \caption{Run-time scatter plots of strategic-company formulas with \ressat in y-axis and compared approaches in x-axis}
    \label{fig:ressat-scatter-strategic}
\end{figure*}

The solving results for the \textit{PEC} family are summarized in~\cref{tbl:random-exist-ssat-pec}.
Note that all the evaluated SSAT solvers failed to exactly solve most formulas in this family.
This phenomenon is similar to a concluding remark from~\cref{chap:prob-design-eval}
that states CNF-based SSAT solvers do not scale nicely when the circuit size grows.
Nevertheless, recall that the proposed~\cref{alg:ressat} is able to derive upper and lower bounds
of the satisfying probability even if it could not exactly solve a large instance.
We will investigate the approximation ability of \ressat in the following.

\begin{table}[ht]
    \centering
    \caption{Summary of the results for~\npec~PEC formulas}
    \label{tbl:random-exist-ssat-pec}
    \begin{tabular}{l|ccc}
        \toprule
        Algorithm          & {\dcssat}                                                 & {\ressat} & {\ressatb} \\
        \midrule
        Solved formulas    & \num{\DcssatReDefaultPecMissingCount}
                           & \num{\RessatMinimizeCachetPecMissingCount}
                           & \num{\RessatBareCachetPecMissingCount}                                             \\
        Timeouts           & \num{\DcssatReDefaultPecErrorTimeoutCount}
                           & \num{\RessatMinimizeCachetPecErrorTimeoutCount}
                           & \num{\RessatBareCachetPecErrorTimeoutCount}                                        \\
        Out of memory      & \num{\DcssatReDefaultPecErrorOutOfMemoryCount}
                           & \num{\RessatMinimizeCachetPecErrorOutOfMemoryCount}
                           & \num{\RessatBareCachetPecErrorOutOfMemoryCount}                                    \\
        Other inconclusive & \num{\DcssatReDefaultPecErrorOtherInconclusiveCount}
                           & \num{\RessatMinimizeCachetPecErrorOtherInconclusiveCount}
                           & \num{\RessatBareCachetPecErrorOtherInconclusiveCount}                              \\
        \bottomrule
    \end{tabular}
\end{table}

\Cref{tbl:random-exist-ssat-pec-0.01,tbl:random-exist-ssat-pec-0.10}
show the results of solving PEC formulas with $\dr$ equal to $0.01$ and $0.1$, respectively.
As \benchexec sends a timeout signal to \ressat after~\timelimit,
we allow an additional~\SI{100}{sec} for \ressat to compute the weights of the collected cubes.
In our evaluation, the numbers of the SAT cubes are often much greater than those of the UNSAT cubes.
In order to successfully deliver useful information upon timeout,
\ressat will first invoke \cachet over the UNSAT cubes to calculate an upper bound.
After \cachet finishes this weight-computation query,
\ressat will call it again over the SAT cubes to obtain a lower bound.
Unfortunately, due to the huge numbers of SAT cubes,
\cachet failed to complete the queries within the additional~\SI{100}{sec}.
As a result, the following tables only report upper bounds (UB) when \ressat suffered from timeouts.
Nevertheless, this is only a technical limitation in our evaluation,
and it should not be misunderstood as the proposed~\cref{alg:ressat} cannot derive lower bounds.

\begin{table}[ht]
    \centering
    \scriptsize
    \caption{Results of solving the PEC formulas ($\dr=0.01$)}
    \label{tbl:random-exist-ssat-pec-0.01}
    \pgfplotstabletypeset[
        every head row/.style={before row={\toprule
                        & \multicolumn{4}{c}{\dcssat} & \multicolumn{6}{c}{\ressat} & \multicolumn{6}{c}{\ressatb}\\},after row=\midrule},
        every last row/.style={after row=\bottomrule},
        empty cells with={--},
        formula column/.list={0},
        time column/.list={1,3,6},
        prob column/.list={2,4,7},
        ubound column/.list={5,8}
    ]
    {random-exist-ssat/evaluation/csv/parsed-PEC-0.01.csv}
\end{table}

\begin{table}[ht]
    \centering
    \scriptsize
    \caption{Results of solving the PEC formulas ($\dr=0.1$)}
    \label{tbl:random-exist-ssat-pec-0.10}
    \pgfplotstabletypeset[
        every head row/.style={before row={\toprule
                        & \multicolumn{4}{c}{\dcssat} & \multicolumn{6}{c}{\ressat} & \multicolumn{6}{c}{\ressatb}\\},after row=\midrule},
        every last row/.style={after row=\bottomrule},
        empty cells with={--},
        formula column/.list={0},
        time column/.list={1,3,6},
        prob column/.list={2,4,7},
        ubound column/.list={5,8}
    ]
    {random-exist-ssat/evaluation/csv/parsed-PEC-0.10.csv}
\end{table}

From these tables, we observe the following phenomenons.
First, \ressat was able to exactly solve a formula or derive a very tight upper bound
when \dcssat solved the formula exactly.
Second, \ressat was able to compute useful upper bounds on larger formulas which \dcssat failed to solve.
Compared to~\cref{tbl:prob-design-eval-pec-0.01,tbl:prob-design-eval-pec-0.10},
these upper bounds are often close to the exact probabilities obtained by the BDD-based SSAT solver.
Third, the proposed minterm-generalization technique helps to tighten the obtained upper bounds.

The above results on the random and application formulas suggest that:
\begin{itemize}
    \item The proposed solver \ressat outperforms \dcssat in terms of both CPU time and memory consumption on random and strategic-company formulas.
    \item The proposed solver \ressat is able to derive non-trivial bounds of satisfying probability,
          while \dcssat suffered from timeouts on most of the PEC formulas.
    \item The minterm-generalization technique is of vital importance to improve the performance of \ressat.
\end{itemize}