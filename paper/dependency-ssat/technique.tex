\section{Lifting SSAT to NEXPTIME-completeness}
\label{sect:dssat-technique}

\subsection{Formulation}
In the following, we extend DQBF to its stochastic variant,
named \textit{dependency stochastic Boolean satisfiability} (DSSAT).

A DSSAT formula $\Qf$ over $V=\{x_1,\ldots,x_n,y_1,\ldots,y_m\}$ is of the form:
\begin{align}
    \label{eq:dssat}
    \random{p_1}x_1,\ldots,\random{p_n}x_n,\exists y_1(\dep{y_1}),\ldots,\exists y_m(\dep{y_m}).\pf(x_1,\ldots,x_n,y_1,\ldots,y_m),
\end{align}
where each $\dep{y_j}\subseteq\{x_1,\ldots,x_n\}$ denotes the set of variables that variable $y_j$ depends on,
and Boolean formula $\pf$ over $V$ is quantifier-free.
We denote the set $\{x_1,\ldots,x_n\}$ (resp. $\{y_1,\ldots,y_m\}$) of randomly (resp. existentially) quantified variables of $\Qf$ by $\rvs$ (resp. $\evs$).

Given a DSSAT formula $\Qf$ and a set of Skolem functions
$\skf=\{f_j:\av{\dep{y_j}}\mapsto\booldom\ |\ j=1,\ldots,m\}$,
the satisfying probability $\spb{\pcf{\Qf}{\skf}}$ of $\Qf$ with respect to $\skf$ is defined as follows:
\begin{align}\label{eq:dssat-probability}
    \spb{\pcf{\Qf}{\skf}}=\spb{\random{p_1}x_1,\ldots,\random{p_n}x_n.\pcf{\pf}{\skf}},
\end{align}
where $\pcf{\pf}{\skf}$ denotes the formula obtained by substituting existentially quantified variables in $\pf$ with their respective Skolem functions, as defined in~\cref{sect:dssat-dqbf}.
In other words, the satisfying probability of $\Qf$ with respect to $\skf$ equals
the satisfying probability of the SSAT formula $\random{p_1}x_1,\ldots,\random{p_n}x_n.\pcf{\pf}{\skf}$.

\begin{example}
    Given a DSSAT formula $\Qf=\random{0.5}x_1,\random{0.5}x_2,\exists y_1(\{x_1\}),\exists y_2(\{x_2\}).\pf$
    and $\pf=(x_1\lor\lnot y_1)(\lnot x_1\lor y_1)(\lnot x_1\lor\lnot x_2\lor y_2)(x_1\lor\lnot y_2)(x_2\lor\lnot y_2)$,
    the satisfying probability of $\Qf$ with respect to $\skf=\{f_1(x_1)=x_1,f_2(x_2)=x_2\}$
    equals $\spb{\random{0.5} x_1,\random{0.5} x_2.\pcf{\pf}{\skf}}=\spb{\random{0.5} x_1,\random{0.5} x_2.(x_1\lor\lnot x_2)}=0.75$.
\end{example}

The \textit{decision version} of DSSAT is stated as follows.
Given a DSSAT formula $\Qf$ and a threshold $\theta\in[0,1]$,
decide whether there exists a set $\skf$ of Skolem functions such that $\spb{\pcf{\Qf}{\skf}}\geq\theta$.
On the other hand, the \textit{optimization version} asks to find a set of Skolem functions to maximize the satisfying probability of $\Qf$.

The formulation of SSAT can be extended by incorporating universal quantifiers,
resulting in a unified framework named \textit{extended} SSAT (XSSAT)~\cite{SATHandbook-SSAT},
which subsumes both QBF and SSAT.
In the extended SSAT,
besides the four rules discussed in~\cref{sect:background-ssat} to calculate the satisfying probability of an SSAT formula $\Qf$,
the following rule is added:
$\spb{\Qf}=\min\{\spb{\ncf{\Qf}{x}},\spb{\pcf{\Qf}{x}}\}$, if $x$ is universally quantified.

Similarly, an \textit{extended} DSSAT (XDSSAT) formula $\Qf$ over a set of variables
$\{x_1,\ldots,x_n,y_1,\ldots,y_m,z_1,\ldots,z_l\}$ is of the form:
\begin{align}\label{eq:dssat-xdssat}
    Q_1 v_1,\ldots,Q_{n+l} v_{n+l},\exists y_1(\dep{y_1}),\ldots,\exists y_m(\dep{y_m}).\pf,
\end{align}
where $Q_i v_i$ equals either $\random{p_k} x_k$ or $\forall z_k$ for some $k$ with $v_i \neq v_j$ for $i \neq j$,
and each $\dep{y_j}\subseteq\{x_1,\ldots,x_n,z_1,\ldots,z_l\}$ denotes the set of randomly and universally quantified variables that variable $y_j$ depends on.

The satisfying probability of an XDSSAT formula $\Qf$ with respect to a set of Skolem functions
$\skf=\{f_j:\av{\dep{y_j}}\mapsto\booldom\ |\ j=1,\ldots,m\}$,
denoted by $\spb{\pcf{\Qf}{\skf}}$,
equals the satisfying probability of the XSSAT formula $Q_1 v_1,\ldots,Q_{n+l} v_{n+l}.\pcf{\pf}{\skf}$.
This satisfiability definition subsumes those for DQBF (\cref{eq:dssat-dqbf-satisfiable}) and DSSAT (\cref{eq:dssat-probability}),
where the variables preceding the existential quantifiers in the prefixes are solely universally or randomly quantified.

Note that in the above extension the Henkin-type quantifiers are only defined for the existential variables.
Although the extended formulation increases practical expressive succinctness,
the computational complexity is not changed as to be shown in the next section.

\subsection{Complexity proof}
In the following, we show that the decision version of DSSAT is NEXPTIME-complete.

\begin{theorem}
    The decision version of DSSAT is NEXPTIME-complete.
\end{theorem}
\begin{proof}
    To show that DSSAT is NEXPTIME-complete,
    we have to show that it belongs to the NEXPTIME complexity class and that it is NEXPTIME-hard.

    First, to see why DSSAT belongs to the NEXPTIME complexity class,
    observe that a Skolem function for an existentially quantified variable can be guessed and constructed in nondeterministic exponential time with respect to the number of randomly quantified variables.
    Given the guessed Skolem functions,
    the evaluation of the matrix,
    summation of weights of satisfying assignments,
    and comparison against the threshold $\theta$ can also be performed in exponential time.
    Overall, the whole procedure is done in nondeterministic exponential time with respect to the input size,
    and hence DSSAT belongs to the NEXPTIME complexity class.

    Second, to see why DSSAT is NEXPTIME-hard,
    we reduce the NEXPTIME-complete problem DQBF to DSSAT as follows.
    Given an arbitrary DQBF:
    \begin{align*}
        \Qf_Q=\forall x_1,\ldots,\forall x_n,\exists y_1(\dep{y_1}),\ldots,\exists y_m(\dep{y_m}).\pf,
    \end{align*}
    we construct a DSSAT formula:
    \begin{align*}
        \Qf_S=\random{0.5} x_1,\ldots,\random{0.5} x_n,\exists y_1(\dep{y_1}),\ldots,\exists y_m(\dep{y_m}).\pf
    \end{align*}
    by changing every universal quantifier to a randomized quantifier with probability $0.5$.
    The reduction can be done in polynomial time with respect to the size of $\Qf_Q$.
    We will show that $\Qf_Q$ is satisfiable if and only if there exists a set $\skf$ of Skolem functions such that $\spb{\pcf{\Qf_S}{\skf}}\geq 1$.

    The ``only if'' direction:
    As $\Qf_Q$ is satisfiable,
    there exists a set $\skf$ of Skolem functions such that
    substituting the existentially quantified variables with the respective Skolem functions
    makes $\pf$ a tautology over variables $\{x_1,\ldots,x_n\}$.
    Therefore, $\spb{\pcf{\Qf_S}{\skf}}=\spb{\random{0.5}x_1,\ldots,\random{0.5}x_n.\top}=1\geq 1$.

    The ``if'' direction:
    As there exists a set $\skf$ of Skolem functions such that $\spb{\pcf{\Qf_S}{\skf}}\geq 1$,
    after substituting the existentially quantified variables with the corresponding Skolem functions,
    each assignment $\as\in\av{\{x_1,\ldots,x_n\}}$ must satisfy $\pf$,
    i.e., $\pf$ becomes a tautology over variables $\{x_1,\ldots,x_n\}$.
    Otherwise, the satisfying probability $\spb{\pcf{\Qf_S}{\skf}}$ must be less than $1$
    as the weight $2^{-n}$ of some unsatisfying assignment is missing from the summation.
    Therefore, $\Qf_Q$ is satisfiable due to the existence of the set $\skf$.
\end{proof}

When DSSAT is extended with universal quantifiers,
its complexity remains in the NEXPTIME complexity class
as the fifth rule of the satisfying probability calculation does not incur any complexity overhead.
Therefore the following corollary is immediate.
\begin{corollary}
    The decision problem of XDSSAT is NEXPTIME-complete.
\end{corollary}
