\section{Applications of DSSAT}
\label{sect:dssat-application}

In this section, we demonstrate two applications of DSSAT.

\subsection{Analyzing probabilistic/approximate partial design}
After formulating DSSAT and proving its NEXPTIME-completeness,
we show its application to the analysis of probabilistic design and approximate design.
Specifically, we consider the probabilistic version of the \textit{topologically constrained logic synthesis problem}~\cite{Sinha2002,Balabanov2014},
or equivalently the \textit{partial design problem}~\cite{Gitina2013}.

In the \textit{(deterministic) partial design problem},
we are given a specification function $G(X)$ over primary input variables $X$ and
a \textit{partial design} $C_F$ with black boxes to be synthesized.
The Boolean functions induced at the primary outputs of $C_F$ can be described by $F(X,T)$,
where $T$ corresponds to the variables of the black box outputs.
Each black box output $t_i$ is specified with its input variables (i.e., dependency set) $\dep{i}\subseteq X \cup Y$ in $C_F$,
where $Y$ represents the variables for intermediate gates in $C_F$ referred to by the black boxes.
The partial design problem aims at deriving the black box functions $\{h_1(\dep{1}),\ldots,h_{|T|}(\dep{|T|})\}$
such that substituting $t_i$ with $h_i$ in $C_F$ makes the resultant circuit function equal $G(X)$.
The above partial design problem can be encoded as a DQBF problem;
moreover, the partial equivalence checking problem is shown to be NEXPTIME-complete~\cite{Gitina2013}.

Specifically, the DQBF that encodes the partial equivalence checking problem is of the form:
\begin{align}
    \label{eq:dssat-dqbf-partial-design}
    \forall X,\forall Y,\exists T(D).(Y \equiv E(X)) \limply (F(X,T)\equiv G(X)),
\end{align}
where $D$ consists of $(\dep{1},\ldots,\dep{|T|})$,
$E$ corresponds to the defining functions of $Y$ in $C_F$,
and the operator ``$\equiv$'' denotes element-wise equivalence between its two operands.

The above partial design problem can be extended to its probabilistic variant,
which is illustrated by the circuit shown in \textbf{TODO}.
The \textit{probabilistic partial design problem} is the same as the deterministic partial design problem except that
$C_F$ is a distilled probabilistic design~\cite{LeeTC18ProbDesign} with black boxes,
whose functions at the primary outputs can be described by $F(X,Z,T)$,
where $Z$ represents the variables for the auxiliary inputs that trigger errors in $C_F$
(including the errors of the black boxes) and
$T$ corresponds to the variables of the black box outputs.
Each black box output $t_i$ is specified with its input variables (i.e., dependency set)
$\dep{i} \subseteq X \cup Y$ in $C_F$.
When $t_i$ is substituted with $h_i$ in $C_F$,
the function of the resultant circuit is required to be sufficiently close to the specification with respect to some expected probability.

\begin{theorem}
    The probabilistic partial design problem is NEXPTIME-complete.
\end{theorem}
\begin{proof}
    To show that the probabilistic partial design problem is in the NEXPTIME complexity class,
    we note that the black box functions can be guessed and validated in time exponential to the number of black box inputs.

    To show completeness in the NEXPTIME complexity class,
    we reduce the known NEXPTIME-complete DSSAT problem to the probabilistic partial design problem,
    similar to the construction used in the previous work~\cite{Gitina2013}.
    Given a DSSAT instance,
    it can be reduced to a probabilistic partial design instance in polynomial time as follows.
    Without loss of generality,
    consider the DSSAT formula~\cref{eq:dssat}.
    We create a probabilistic partial design instance by letting the specification $G$ be a tautology and
    letting $C_F$ be a probabilistic design with black boxes,
    which involves primary inputs $x_1,\ldots,x_n$ and black box outputs $y_1,\ldots,y_m$ to compute the matrix $\pf$.
    The driving inputs of the black box output $y_j$ is specified by the dependency set $\dep{y_j}$ in~\cref{eq:dssat},
    and the probability for primary input $x_i$ to evaluate to $\top$ is set to $p_i$.
    The original DSSAT formula is satisfiable with respect to a target satisfying probability $\theta$ if and only if
    there exist implementations of the black boxes such that the resultant circuit composed with the black box implementations behaves like a tautology with respect to the required expectation $\theta$.
\end{proof}

On the other hand,
the probabilistic partial design problem can be encoded with the following XDSSAT formula:
\begin{align}
    \label{eq:dssat-partial-design}
    \random{} X,\random{} Z,\forall Y,\exists T(D).(Y \equiv E(X)) \limply (F(X,Z,T)\equiv G(X)),
\end{align}
where the primary input variables are randomly quantified with probability $p_{x_i}$ of $x_i \in X$ to reflect their weights,
and the error-triggering auxiliary input variables $Z$ are randomly quantified according to the pre-specified error rates of the erroneous gates in $C_F$.
Notice that the above formula takes advantage of the extension with universal quantifiers as discussed previously.

In approximate design,
a circuit implementation may deviate from its specification by a certain extent.
The amount of deviation can be characterized in a way similar to the error probability calculation in probabilistic design.
For approximate partial design,
the equivalence checking problem can be expressed by the XDSSAT formula:
\begin{align}
    \label{eq:dssat-approximate-design}
    \random{} X,\forall Y,\exists T(D).(Y \equiv E(X)) \limply (F(X,T)\equiv G(X)),
\end{align}
which differs from~\cref{eq:dssat-partial-design} only in requiring no auxiliary inputs.
The probabilities of the randomly quantified primary input variables are determined by the approximation criteria in measuring the deviation.
For example, when all the input assignments are of equal weight,
the probabilities of the primary inputs are all set to 0.5.

We note that as the engineering change order (ECO) problem~\cite{JiangDATE20ECOSurvey} heavily relies on partial design equivalence checking,
the above DSSAT formulations provide fundamental bases for ECOs of probabilistic and approximate designs.