\section{Solving probabilistic property evaluation}
\label{sect:prob-solutions}

We propose different solutions to the MPPE and PPE problems.
For the former, we resort to SSAT solving.
For the latter, in addition to SSAT solving,
we present solutions based on signal probability calculation,
weighted model counting,
and probabilistic model checking.

\subsection{Solving MPPE and PPE via SSAT}
Note that the entire miter SPBN can be directly converted to a CNF formula by Tseitin transformation~\cite{Tseitin1983}
since all vertices except for those in $X,Z,W$ are error-free after the standardization.
In the following,
let $\pf_M$ be the CNF formula converted from the miter SPBN $G_M$,
which contains vertex sets
$X=\{x_1,\ldots,x_n\}$,
$Y=\{y_1,\ldots,y_m\}$,
$Z=\{z_1,\ldots,z_l\}$, and
$W=\{w_1,\ldots,w_q\}$ as shown in~\cref{fig:prob-spbn-miter}.
Observe that $X \cup Z \cup W$ is a base set for $\pf_M$.
Given a parameter assignment $\pi$ to $X$,
we define the corresponding weighting function $\wt:X\cup Z\cup W\mapsto[0,1]$ for $\pf_M$ as
$\wt(x_i)=\pi(x_i)$,
$\wt(y_j)=p_{y_j}$, and
$\wt(w_k)=p_{w_k}$ for all
$x_i \in X$,
$y_j \in Y$, and
$w_k \in W$.
The weight assignment $\wt$ will be used throughout our discussion.

\begin{theorem}
    \label{thm:prob-ppe-ssat}
    The probabilistic property evaluation (PPE) of $\pf_M$ under a parameter assignment $\pi$ can be expressed
    by the following SSAT formula $\Qf_\mathrm{PPE}(\pi)$:
    \begin{align}
        \label{eq:prob-ppe-ssat}
        \random{\pi(x_1)}x_1,\ldots,\random{\pi(x_n)}x_n,
        \random{p_{z_1}}z_1,\ldots,\random{p_{z_l}}z_l,
        \random{p_{w_1}}w_1,\ldots,\random{p_{w_q}}w_q,
        \exists y_1,\ldots,\exists y_m.
        \pf_M.
    \end{align}
\end{theorem}
\begin{proof}
    Let $A$ be the event $\pf_M=\top$ and $\Lambda=\av{X \cup Z \cup W}$.
    By the law of total probability,
    $\Pr[A]=\sum\limits_{\as\in\Lambda}\Pr[\as]\Pr[A\mid\as]$,
    where $\Pr[\as]=\wt(\as)$ ($\wt$ is the weighting function defined previously) and
    $\Pr[A\mid\as]$ is the conditional probability of event $A$ under the assignment $\as$.
    Notice that $\Pr[A\mid\as]=\pcf{\pf_M}{\as}$ since
    $X \cup Z \cup W$ is a base set for $\pf_M$.
    As a result,
    $\Pr[A]=\sum\limits_{\as\in\Lambda}\wt(\as)\pcf{\pf_M}{\as}$,
    which equals the satisfying probability of the SSAT formula $\Qf_\mathrm{PPE}(\pi)$.
\end{proof}

\begin{theorem}
    \label{thm:prob-mppe-ssat}
    The maximum probabilistic property evaluation (MPPE) of $\pf_M$ can be expressed
    by the following SSAT formula $\Qf_\mathrm{MPPE}$:
    \begin{align}
        \label{eq:prob-mppe-ssat}
        \exists x_1,\ldots,\exists x_n,
        \random{p_{z_1}}z_1,\ldots,\random{p_{z_l}}z_l,
        \random{p_{w_1}}w_1,\ldots,\random{p_{w_q}}w_q,
        \exists y_1,\ldots,\exists y_m.
        \pf_M.
    \end{align}
\end{theorem}
\begin{proof}
    By the same argument in the proof of~\cref{thm:prob-ppe-ssat},
    given an assignment $\as_X$ over $X$,
    the SSAT formula:
    \begin{align*}
        \Qf_\mathrm{MPPE}(\as_X)=
        \random{p_{z_1}}{z_1},\ldots,\random{p_{z_l}}{z_l},
        \random{p_{w_1}}{w_1},\ldots,\random{p_{w_q}}{w_q},
        \exists y_1,\ldots,\exists y_m.
        \pcf{\pf_M}{\as_X}
    \end{align*}
    computes the satisfying probability of the miter under the assignment $\as_X$.
    According to the SSAT semantics,
    the outermost existential quantification of primary inputs $X$
    ensures to find an optimum assignment $\as_X^*$
    such that the satisfying probability of $\Qf_\mathrm{MPPE}(\as_X^*)$ is maximized.
    Hence the SSAT formula $\Qf_\mathrm{MPPE}$ computes the maximum satisfying probability of the miter.
\end{proof}

Note that the only difference between $\Qf_\mathrm{MPPE}$ and $\Qf_\mathrm{PPE}$
lies in the quantification for the primary inputs $X$.
Although SSAT provides a convenient language for expressing both MPPE and PPE problems,
its solvers to date remain immature to handle formulas of practical sizes in our considered application.
One of the main inefficiencies can be attributed to representing $\pf_M$ in CNF,
which results in the additional quantification of the intermediate circuit variables $Y=\{y_1,\ldots,y_m\}$.
It motivates the development of a new SSAT solver as we present below.

\subsubsection{BDD-based SSAT solving}

\begin{algorithm}[t]
    \caption{BDD-based SSAT solving: \texttt{BddSsatSolve}}
    \label{alg:bddssat}
    \begin{algorithmic}[1]
        \REQUIRE $\Qf=Q_1 v_1,\ldots,Q_n v_n.\pf$
        \ENSURE $\spb{\Qf}$
        \STATE $N := \texttt{BuildReducedOrderedBdd}(\pf,(v_1,\ldots,v_n))$\label{code:bddssat-build-bdd}
        \STATE $Q := Q_1 v_1,\ldots,Q_n v_n$
        \RETURN $\texttt{BddSsatRecur}(N,Q)$\label{code:bddssat-recursive}
    \end{algorithmic}
\end{algorithm}

\begin{algorithm}[t]
    \caption{The recursive step of \texttt{BddSsatSolve}: \texttt{BddSsatRecur}}
    \label{alg:bddssat-recursive}
    \begin{algorithmic}[1]
        \REQUIRE An ROBDD node $N$ and a prefix $Q$
        \ENSURE $\spb{N=\top}$ under $Q$
        \IF{($N$ is a terminal node)}\label{code:bddssat-recursive-constant-start}
        \RETURN $\nodesp{N}$\label{code:bddssat-recursive-constant-end}
        \ENDIF
        \IF{($\nodevisit{N}=\false$)}
        \IF{($Q(\nodevar{N})=\random{p}$)}
        \STATE $\nodesp{N}:=(1-p)\cdot\texttt{BddSsatRecur}(\nodeelse{N},Q)+p\cdot\texttt{BddSsatRecur}(\nodethen{N},Q)$
        \label{code:bddssat-recursive-random}
        \ELSE
        \STATE $\nodesp{N}:=\max\{\texttt{BddSsatRecur}(\nodeelse{N},Q),\texttt{BddSsatRecur}(\nodethen{N},Q)\}$
        \label{code:bddssat-recursive-exist}
        \ENDIF
        \STATE $\nodevisit{N} := \true$
        \ENDIF
        \RETURN $\nodesp{N}$
    \end{algorithmic}
\end{algorithm}

We propose a BDD-based solver to enhance the scalability of SSAT solving.
The procedure \texttt{BddSsatSolve} is outlined in~\cref{alg:bddssat},
which takes as input an SSAT formula $\Qf=Q_1 v_1,\ldots,Q_n v_n.\pf$.
At~\cref{code:bddssat-build-bdd},
a \textit{reduced ordered BDD} (ROBDD) of $\pf$ is built with a variable ordering following the quantification order.
At~\cref{code:bddssat-recursive},
a recursive procedure \texttt{BddSsatRecur},
sketched in~\cref{alg:bddssat-recursive},
is called to calculate the satisfying probability of $\Qf$.
In the pseudo code, for an ROBDD node $N$,
$\nodethen{N}$ and $\nodeelse{N}$ denote its $\mathrm{then}$- and $\mathrm{else}$-child, respectively;
$\nodeval{N}$ equals 0 (resp. 1) if $N$ is a 0-terminal (resp. 1-terminal) node;
$\nodevisit{N}$ is a flag initialized to \false and records whether $N$ has been processed;
$\nodevar{N}$ and $\nodesp{N}$ denote the control variable and the satisfying probability of node $N$, respectively.
In~\cref{alg:bddssat-recursive},
\crefrange{code:bddssat-recursive-constant-start}{code:bddssat-recursive-constant-end} implement the first and second computation rules of SSAT in~\cref{sect:background-ssat};
\Cref{code:bddssat-recursive-random} and \cref{code:bddssat-recursive-exist} implement the third and fourth rules corresponding to the random and existential quantification of the variable, respectively.

Note that \texttt{BddSsatSolve} runs in time linear to the number of BDD nodes (as each node is processed only once).
Therefore the computation complexity is dominated by constructing the ROBDD of $\pf$.
Note also that if the outermost variables in the quantification order are existentially quantified, e.g.,
those in $\Qf_\mathrm{MPPE}$ of~\cref{thm:prob-mppe-ssat},
the corresponding assignments to the existentially quantified variables
to maximize the satisfying probability of $\pf$ can be obtained.
Specifically,
the assignment to an outermost existential variable $\nodevar{N}$ can be derived by recording which of
\texttt{BddSsatRecur}($\nodeelse{N},Q$) and
\texttt{BddSsatRecur}($\nodethen{N},Q$)
contributes to the maximum probability in~\cref{code:bddssat-recursive-exist} of~\cref{alg:bddssat-recursive}.

\subsubsection{Signal Probability with BDD-based SSAT}
We exploit the developed BDD-based SSAT solver to compute signal probabilities as defined in~\cref{def:prob-signal-prob,def:prob-signal-prob-max}.
Given an SPBN $G=(V,E)$,
the satisfying probability of any $v \in V$ can be obtained by first encoding the problem into an SSAT instance
and then solving the SSAT formula by \texttt{BddSsatSolve}.

Notice that formula $\pf$ needs not be represented in CNF for \texttt{BddSsatSolve}.
In the application of circuit verification,
formula $\pf$ can be directly input to \texttt{BddSsatSolve} as a circuit.
Without Tseitin transformation from circuit to CNF formula,
the algorithm avoids introducing extra variables.
Consequently, the innermost existential quantification $\exists y_1,\ldots,\exists y_m$
in~\cref{eq:prob-ppe-ssat} of~\cref{thm:prob-ppe-ssat}
and~\cref{eq:prob-mppe-ssat} of~\cref{thm:prob-mppe-ssat} is removed.
The utilization of circuit structures makes the calculation of signal probability
(and therefore the computation of PPE and MPPE) on an SPBN more efficient and scalable
using the proposed BDD-based SSAT solver.
Empirical evidence suggests our proposed SSAT solver outperforms and is much scalable than other CNF-based SSAT solvers.

We note that the signal probability calculation via BDD is used for power estimation of integrated circuits~\cite{Najm1994}.
As we formulate PPE as a signal probability problem on SPBN,
any prior method for signal probability calculation can be used to solve PPE.
However, it is worth emphasizing that prior methods for signal probability calculation,
such as Monte Carlo simulation, cannot solve MPPE.
The BDD-based method has its unique value over other previous endeavors for signal probability calculation
because of its generality of solving both PPE and MPPE.
On the other hand,
since we established the connections of MPPE and PPE to SSAT formulations,
SSAT solvers can also be applied to calculate the maximum signal probability as well as signal probability.

BDD is a well-studied data structure,
but known for its memory limitation.
Techniques have been highly developed in the 1990s to extend its scalability.
Practical experience suggests that BDD-based computation remains competitive to other formulations
due to the fact that other tools,
such as SSAT and model counting,
are still in their early development.
In our experiments over \texttt{ISCAS} and \texttt{ITC} benchmark suits,
the BDD-based approach stands as the most scalable one over other formulations to be discussed.

\subsubsection{Generalization to dependent random variables}
The proposed BDD-based SSAT solver is advantageous over other SSAT solvers for its ability to handle randomly quantified variables whose probability distributions are mutually dependent.
Note that according to the SSAT syntax,
there is no support to describe the joint behavior among randomly quantified variables.
That is, every randomly quantified variable acts independently of each other.
This assumption limits the expressiveness of SSAT.
In this paper, we combine our BDD-based SSAT solver with previous methods~\cite{Marculescu1998,Miskov-Zivanov2006}
to represent joint probability distribution of random variables,
and propose a novel SSAT solver that is capable of expressing mutual dependence among randomly quantified variables.
A joint probability distribution of random variables is represented as an algebraic decision diagram (ADD)~\cite{Marculescu1998,Miskov-Zivanov2006}.
After such an ADD representing joint probability distribution is constructed,
the original BDD (which is also an ADD) of the Boolean formula $\pf$ is conjoined with the ADD.
Finally, the value of the SSAT formula can be calculated by traversing the merged ADD similar to the prior independent counterpart.

Following the above strategy,
now we explain how to extend the proposed PPE and MPPE framework to approach PBNs whose random variables are mutually dependent.
After the distillation operation,
the mutually dependent random variables on erroneous vertices are converted to correlated AIs.
Given the joint probability distribution of random variables,
an ADD is built to represent the mutual dependence among AIs.
Similarly, if PIs are correlated,
another ADD can be built to describe their correlation.
After multiplying the ADDs with the BDD of the circuit under evaluation,
the average (PPE) or the maximum (MPPE) violating probability can be computed through traversing the product ADD.
Therefore, the proposed PPE framework is generalized to dependent PBNs,
whose random variables have mutual dependent probability distribution.