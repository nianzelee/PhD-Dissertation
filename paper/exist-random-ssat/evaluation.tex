\renewcommand{\nrandom}{\num{700}}
\renewcommand{\napplication}{\num{212}}
\newcommand{\ntoilet}{\num{77}}
\newcommand{\nmaxcount}{\num{26}}
\newcommand{\nsandcastle}{\num{25}}
\newcommand{\nconformant}{\num{24}}
\newcommand{\nmpec}{\num{60}}

\section{Evaluation}
\label{sect:erssat-evaluation}

We evaluated the proposed~\cref{alg:erssat} against
the state-of-the-art DPLL-based SSAT solver \dcssat~\cite{Majercik2005}
over both random $k$-CNF and application formulas.
The proposed algorithm is implemented in the \texttt{C++} language inside the \abc~\cite{ABC} environment.
The SAT solver \minisat-2.2~\cite{Een2003Solver} is used to answer satisfiability queries.
For weighted model counting,
we tried \cachet~\cite{Sang2004,Sang2005ModelCounting},
but the overall performance was not satisfactory.
Instead, we resorted to a well-developed BDD package \cudd~\cite{CUDD}.
Weight computation of a formula is fulfilled via a classic approach~\cite{Darwiche2002KnowledgeCompilation} that traverses the BDD of a formula and computes the satisfying probabilities of the BDD nodes.
Our prototyping implementation\footnote{Available at: \url{\ssatabcurl}} is named \erssat.
A bare version of \erssat without the clause-strengthening heuristics is called \erssatb in the experiments.
We used \ssatABCRevision in the experiments.

\subsection{Benchmark set}
The SSAT instances in the evaluation are hosted
in a publicly available database\footnote{Available at: \url{\ssatbenchmarkurl}}.
We used \ssatBenchRevision in the experiments.

\subsubsection{Random $k$-CNF formulas}
We generated random $k$-CNF formulas by \cnfgen~\cite{Lauria2017CNFgen}.
A collection of~\nrandom~formulas were generated with $k$,
i.e., the number of literals in a clause,
taking values from $\{3,4,5,6,7,8,9\}$,
the number of variables taking values from $\{10,20,30,40,50\}$,
and clause-to-variable ratio taking values from $\{k-1,k,k+1,k+2\}$.
Five formulas were sampled for each parameter combination.
To convert the propositional formulas into E-MAJSAT formulas,
the first half of the variables are existentially quantified,
and the rest are randomly quantified with probability $0.37$.

\subsubsection{Application formulas}
\begin{table}[ht]
    \centering
    \caption{The families of the application formulas}
    \label{tbl:exist-random-ssat-families}
    \begin{tabular}{c|c|c}
        Family               & Description                                            & Number       \\
        \hline
        \textit{Toilet-A}    & Adapted from exist-forall QBFs~\cite{Narizzano2006}    & \ntoilet     \\
        \textit{Conformant}  & Adapted from exist-forall QBFs~\cite{Narizzano2006}    & \nconformant \\
        \textit{Sand-Castle} & A probabilistic planning problem~\cite{Majercik1998}   & \nsandcastle \\
        \textit{Max-Count}   & Adapted from maximum model counting~\cite{Fremont2017} & \nmaxcount   \\
        \textit{MPEC}        & Maximum probabilistic equivalence checking             & \nmpec       \\
    \end{tabular}
\end{table}

We collected five families of application formulas for evaluation.
Their descriptions and the numbers of the instances in each family are summarized in~\cref{tbl:exist-random-ssat-families}.
The first two families,
\textit{Toilet-A} and \textit{Conformant},
were adapted from exist-forall-exist QBFs~\cite{Narizzano2006}.
We converted the QBFs into exist-random-exist quantified SSAT formulas
by replacing their universal quantifiers with randomized ones with probabilities $0.5$.
The third family \textit{Sand-Castle} is a probabilistic conformant planning domain.
The problem can be encoded as E-MAJSAT formulas~\cite{Majercik1998}.
The family \textit{Max-Count} models the problems of maximum satisfiability, quantitative information flow, and program synthesis with maximum model counting~\cite{Fremont2017}.
We represented the maximum model counting instances as E-MAJSAT formulas.
The last family \textit{MPEC} consists of formulas that analyze the maximum probability of a probabilistic circuit to produce erroneous outputs,
as discussed in~\cref{chap:prob-design-eval}.

\subsection{Experimental setup}
Our experiments were performed on a machine with~\machineSpec.
The operating system was~\osInfo.
The programs were compiled with~\compiler.
Each SSAT-solving task was limited to a CPU core,
a CPU time of~\timelimit,
and a memory usage of~\memlimit.
To achieve reliable benchmarking,
we used a benchmarking framework \benchexec\footnote{Available at: \url{\benchexecurl}}~\cite{Benchmarking-STTT},
and assumed~\measurement.

\subsection{Results}

\subsubsection{Random $k$-CNF formulas}

\begin{figure*}[hp]
    \centering
    \subfloat[CPU time]{
        \includegraphics{exist-random-ssat/evaluation/plots/quantile-cputime-Random.pdf}
        \label{fig:erssat-quantile-cputime-random}
    }\\
    \subfloat[Memory usage]{
        \includegraphics{exist-random-ssat/evaluation/plots/quantile-memory-Random.pdf}
        \label{fig:erssat-quantile-memory-random}
    }
    \caption{Quantile plots of random $k$-CNF formulas}
    \label{fig:erssat-quantile-random}
\end{figure*}

\cref{fig:erssat-quantile-random} shows the quantile plots regarding CPU time and memory usage
of the SSAT instances derived from the random $k$-CNF formulas.
A data point $(x,y)$ in a quantile plot indicates that
there are $x$ formulas processed by the respective algorithm within a resource constraint of $y$.
In~\cref{fig:erssat-quantile-cputime-random},
we observe that \erssat solved a similar amount of formulas as \dcssat did.
Moreover, the clause-strengthening heuristics improve the performance of \erssat a lot,
as can be seen from the huge difference between \erssat and \erssatb.
On the other hand,
\cref{fig:erssat-quantile-memory-random} shows that \dcssat used much more memory than \erssat.
This can be attributed to the subformula caching of \dcssat.
Instead, \erssat only builds BDDs for cofactored formulas, which confined its memory footprint.

\subsubsection{Application formulas}


% The following definition defines a command for each value.
% The command name is the concatenation of the first six arguments.
% To override this definition, define \StoreBenchExecResult with \newcommand before including this file.
% Arguments: benchmark name, run-set name, category, status, column name, statistic, value
\providecommand\StoreBenchExecResult[7]{\expandafter\newcommand\csname#1#2#3#4#5#6\endcsname{#7}}%
\StoreBenchExecResult{DcssatEr}{DefaultApplication}{Total}{}{Count}{}{160}%
\StoreBenchExecResult{DcssatEr}{DefaultApplication}{Total}{}{Cputime}{}{71834.861030650}%
\StoreBenchExecResult{DcssatEr}{DefaultApplication}{Total}{}{Cputime}{Avg}{448.9678814415625}%
\StoreBenchExecResult{DcssatEr}{DefaultApplication}{Total}{}{Cputime}{Median}{330.638654461}%
\StoreBenchExecResult{DcssatEr}{DefaultApplication}{Total}{}{Cputime}{Min}{0.005558095}%
\StoreBenchExecResult{DcssatEr}{DefaultApplication}{Total}{}{Cputime}{Max}{901.501094716}%
\StoreBenchExecResult{DcssatEr}{DefaultApplication}{Total}{}{Cputime}{Stdev}{410.6288898779780039960556041}%
\StoreBenchExecResult{DcssatEr}{DefaultApplication}{Total}{}{Walltime}{}{71838.407134809531468968}%
\StoreBenchExecResult{DcssatEr}{DefaultApplication}{Total}{}{Walltime}{Avg}{448.99004459255957168105}%
\StoreBenchExecResult{DcssatEr}{DefaultApplication}{Total}{}{Walltime}{Median}{330.6678233835846}%
\StoreBenchExecResult{DcssatEr}{DefaultApplication}{Total}{}{Walltime}{Min}{0.008783593773841858}%
\StoreBenchExecResult{DcssatEr}{DefaultApplication}{Total}{}{Walltime}{Max}{901.5298540564254}%
\StoreBenchExecResult{DcssatEr}{DefaultApplication}{Total}{}{Walltime}{Stdev}{410.6444476790752092161474212}%
\StoreBenchExecResult{DcssatEr}{DefaultApplication}{Error}{}{Count}{}{89}%
\StoreBenchExecResult{DcssatEr}{DefaultApplication}{Error}{}{Cputime}{}{68148.156984843}%
\StoreBenchExecResult{DcssatEr}{DefaultApplication}{Error}{}{Cputime}{Avg}{765.7096290431797752808988764}%
\StoreBenchExecResult{DcssatEr}{DefaultApplication}{Error}{}{Cputime}{Median}{901.000404586}%
\StoreBenchExecResult{DcssatEr}{DefaultApplication}{Error}{}{Cputime}{Min}{0.00690421}%
\StoreBenchExecResult{DcssatEr}{DefaultApplication}{Error}{}{Cputime}{Max}{901.501094716}%
\StoreBenchExecResult{DcssatEr}{DefaultApplication}{Error}{}{Cputime}{Stdev}{242.3007420382804066182760935}%
\StoreBenchExecResult{DcssatEr}{DefaultApplication}{Error}{}{Walltime}{}{68151.210960471071555125}%
\StoreBenchExecResult{DcssatEr}{DefaultApplication}{Error}{}{Walltime}{Avg}{765.7439433760794556755617978}%
\StoreBenchExecResult{DcssatEr}{DefaultApplication}{Error}{}{Walltime}{Median}{901.0384520078078}%
\StoreBenchExecResult{DcssatEr}{DefaultApplication}{Error}{}{Walltime}{Min}{0.014586311765015125}%
\StoreBenchExecResult{DcssatEr}{DefaultApplication}{Error}{}{Walltime}{Max}{901.5298540564254}%
\StoreBenchExecResult{DcssatEr}{DefaultApplication}{Error}{}{Walltime}{Stdev}{242.3076097998281048370405144}%
\StoreBenchExecResult{DcssatEr}{DefaultApplication}{Error}{Error}{Count}{}{1}%
\StoreBenchExecResult{DcssatEr}{DefaultApplication}{Error}{Error}{Cputime}{}{0.00690421}%
\StoreBenchExecResult{DcssatEr}{DefaultApplication}{Error}{Error}{Cputime}{Avg}{0.00690421}%
\StoreBenchExecResult{DcssatEr}{DefaultApplication}{Error}{Error}{Cputime}{Median}{0.00690421}%
\StoreBenchExecResult{DcssatEr}{DefaultApplication}{Error}{Error}{Cputime}{Min}{0.00690421}%
\StoreBenchExecResult{DcssatEr}{DefaultApplication}{Error}{Error}{Cputime}{Max}{0.00690421}%
\StoreBenchExecResult{DcssatEr}{DefaultApplication}{Error}{Error}{Cputime}{Stdev}{0.00000000000000}%
\StoreBenchExecResult{DcssatEr}{DefaultApplication}{Error}{Error}{Walltime}{}{0.014586311765015125}%
\StoreBenchExecResult{DcssatEr}{DefaultApplication}{Error}{Error}{Walltime}{Avg}{0.014586311765015125}%
\StoreBenchExecResult{DcssatEr}{DefaultApplication}{Error}{Error}{Walltime}{Median}{0.014586311765015125}%
\StoreBenchExecResult{DcssatEr}{DefaultApplication}{Error}{Error}{Walltime}{Min}{0.014586311765015125}%
\StoreBenchExecResult{DcssatEr}{DefaultApplication}{Error}{Error}{Walltime}{Max}{0.014586311765015125}%
\StoreBenchExecResult{DcssatEr}{DefaultApplication}{Error}{Error}{Walltime}{Stdev}{0.000000000000000000}%
\StoreBenchExecResult{DcssatEr}{DefaultApplication}{Error}{OutOfMemory}{Count}{}{25}%
\StoreBenchExecResult{DcssatEr}{DefaultApplication}{Error}{OutOfMemory}{Cputime}{}{11381.691933651}%
\StoreBenchExecResult{DcssatEr}{DefaultApplication}{Error}{OutOfMemory}{Cputime}{Avg}{455.26767734604}%
\StoreBenchExecResult{DcssatEr}{DefaultApplication}{Error}{OutOfMemory}{Cputime}{Median}{395.236226995}%
\StoreBenchExecResult{DcssatEr}{DefaultApplication}{Error}{OutOfMemory}{Cputime}{Min}{187.455423821}%
\StoreBenchExecResult{DcssatEr}{DefaultApplication}{Error}{OutOfMemory}{Cputime}{Max}{782.882748459}%
\StoreBenchExecResult{DcssatEr}{DefaultApplication}{Error}{OutOfMemory}{Cputime}{Stdev}{207.4075182889810651207325628}%
\StoreBenchExecResult{DcssatEr}{DefaultApplication}{Error}{OutOfMemory}{Walltime}{}{11382.41090900916614}%
\StoreBenchExecResult{DcssatEr}{DefaultApplication}{Error}{OutOfMemory}{Walltime}{Avg}{455.2964363603666456}%
\StoreBenchExecResult{DcssatEr}{DefaultApplication}{Error}{OutOfMemory}{Walltime}{Median}{395.26761069893837}%
\StoreBenchExecResult{DcssatEr}{DefaultApplication}{Error}{OutOfMemory}{Walltime}{Min}{187.46373896300793}%
\StoreBenchExecResult{DcssatEr}{DefaultApplication}{Error}{OutOfMemory}{Walltime}{Max}{782.9895220380276}%
\StoreBenchExecResult{DcssatEr}{DefaultApplication}{Error}{OutOfMemory}{Walltime}{Stdev}{207.4195162042323621035277689}%
\StoreBenchExecResult{DcssatEr}{DefaultApplication}{Error}{Timeout}{Count}{}{63}%
\StoreBenchExecResult{DcssatEr}{DefaultApplication}{Error}{Timeout}{Cputime}{}{56766.458146982}%
\StoreBenchExecResult{DcssatEr}{DefaultApplication}{Error}{Timeout}{Cputime}{Avg}{901.0548912219365079365079365}%
\StoreBenchExecResult{DcssatEr}{DefaultApplication}{Error}{Timeout}{Cputime}{Median}{901.024737657}%
\StoreBenchExecResult{DcssatEr}{DefaultApplication}{Error}{Timeout}{Cputime}{Min}{900.923230944}%
\StoreBenchExecResult{DcssatEr}{DefaultApplication}{Error}{Timeout}{Cputime}{Max}{901.501094716}%
\StoreBenchExecResult{DcssatEr}{DefaultApplication}{Error}{Timeout}{Cputime}{Stdev}{0.1000676663868040309241283849}%
\StoreBenchExecResult{DcssatEr}{DefaultApplication}{Error}{Timeout}{Walltime}{}{56768.7854651501404}%
\StoreBenchExecResult{DcssatEr}{DefaultApplication}{Error}{Timeout}{Walltime}{Avg}{901.0918327801609587301587302}%
\StoreBenchExecResult{DcssatEr}{DefaultApplication}{Error}{Timeout}{Walltime}{Median}{901.0557705014944}%
\StoreBenchExecResult{DcssatEr}{DefaultApplication}{Error}{Timeout}{Walltime}{Min}{901.0144192399457}%
\StoreBenchExecResult{DcssatEr}{DefaultApplication}{Error}{Timeout}{Walltime}{Max}{901.5298540564254}%
\StoreBenchExecResult{DcssatEr}{DefaultApplication}{Error}{Timeout}{Walltime}{Stdev}{0.09690864417508489761207137976}%
\StoreBenchExecResult{DcssatEr}{DefaultApplication}{Missing}{}{Count}{}{71}%
\StoreBenchExecResult{DcssatEr}{DefaultApplication}{Missing}{}{Cputime}{}{3686.704045807}%
\StoreBenchExecResult{DcssatEr}{DefaultApplication}{Missing}{}{Cputime}{Avg}{51.92540909587323943661971831}%
\StoreBenchExecResult{DcssatEr}{DefaultApplication}{Missing}{}{Cputime}{Median}{0.401376788}%
\StoreBenchExecResult{DcssatEr}{DefaultApplication}{Missing}{}{Cputime}{Min}{0.005558095}%
\StoreBenchExecResult{DcssatEr}{DefaultApplication}{Missing}{}{Cputime}{Max}{886.698473883}%
\StoreBenchExecResult{DcssatEr}{DefaultApplication}{Missing}{}{Cputime}{Stdev}{151.6030105157439752550361934}%
\StoreBenchExecResult{DcssatEr}{DefaultApplication}{Missing}{}{Walltime}{}{3687.196174338459913843}%
\StoreBenchExecResult{DcssatEr}{DefaultApplication}{Missing}{}{Walltime}{Avg}{51.93234048364028047666197183}%
\StoreBenchExecResult{DcssatEr}{DefaultApplication}{Missing}{}{Walltime}{Median}{0.402801631949842}%
\StoreBenchExecResult{DcssatEr}{DefaultApplication}{Missing}{}{Walltime}{Min}{0.008783593773841858}%
\StoreBenchExecResult{DcssatEr}{DefaultApplication}{Missing}{}{Walltime}{Max}{886.725481309928}%
\StoreBenchExecResult{DcssatEr}{DefaultApplication}{Missing}{}{Walltime}{Stdev}{151.6124987101443727971910575}%
\StoreBenchExecResult{DcssatEr}{DefaultApplication}{Missing}{Done}{Count}{}{71}%
\StoreBenchExecResult{DcssatEr}{DefaultApplication}{Missing}{Done}{Cputime}{}{3686.704045807}%
\StoreBenchExecResult{DcssatEr}{DefaultApplication}{Missing}{Done}{Cputime}{Avg}{51.92540909587323943661971831}%
\StoreBenchExecResult{DcssatEr}{DefaultApplication}{Missing}{Done}{Cputime}{Median}{0.401376788}%
\StoreBenchExecResult{DcssatEr}{DefaultApplication}{Missing}{Done}{Cputime}{Min}{0.005558095}%
\StoreBenchExecResult{DcssatEr}{DefaultApplication}{Missing}{Done}{Cputime}{Max}{886.698473883}%
\StoreBenchExecResult{DcssatEr}{DefaultApplication}{Missing}{Done}{Cputime}{Stdev}{151.6030105157439752550361934}%
\StoreBenchExecResult{DcssatEr}{DefaultApplication}{Missing}{Done}{Walltime}{}{3687.196174338459913843}%
\StoreBenchExecResult{DcssatEr}{DefaultApplication}{Missing}{Done}{Walltime}{Avg}{51.93234048364028047666197183}%
\StoreBenchExecResult{DcssatEr}{DefaultApplication}{Missing}{Done}{Walltime}{Median}{0.402801631949842}%
\StoreBenchExecResult{DcssatEr}{DefaultApplication}{Missing}{Done}{Walltime}{Min}{0.008783593773841858}%
\StoreBenchExecResult{DcssatEr}{DefaultApplication}{Missing}{Done}{Walltime}{Max}{886.725481309928}%
\StoreBenchExecResult{DcssatEr}{DefaultApplication}{Missing}{Done}{Walltime}{Stdev}{151.6124987101443727971910575}%
% The following definition defines a command for each value.
% The command name is the concatenation of the first six arguments.
% To override this definition, define \StoreBenchExecResult with \newcommand before including this file.
% Arguments: benchmark name, run-set name, category, status, column name, statistic, value
\providecommand\StoreBenchExecResult[7]{\expandafter\newcommand\csname#1#2#3#4#5#6\endcsname{#7}}%
\StoreBenchExecResult{Erssat}{DefaultBddApplication}{Total}{}{Count}{}{160}%
\StoreBenchExecResult{Erssat}{DefaultBddApplication}{Total}{}{Cputime}{}{95433.696587174}%
\StoreBenchExecResult{Erssat}{DefaultBddApplication}{Total}{}{Cputime}{Avg}{596.4606036698375}%
\StoreBenchExecResult{Erssat}{DefaultBddApplication}{Total}{}{Cputime}{Median}{900.9960360215}%
\StoreBenchExecResult{Erssat}{DefaultBddApplication}{Total}{}{Cputime}{Min}{0.040599129}%
\StoreBenchExecResult{Erssat}{DefaultBddApplication}{Total}{}{Cputime}{Max}{901.683691398}%
\StoreBenchExecResult{Erssat}{DefaultBddApplication}{Total}{}{Cputime}{Stdev}{409.8293182112904061356360349}%
\StoreBenchExecResult{Erssat}{DefaultBddApplication}{Total}{}{Walltime}{}{95439.408425660803482168}%
\StoreBenchExecResult{Erssat}{DefaultBddApplication}{Total}{}{Walltime}{Avg}{596.49630266038002176355}%
\StoreBenchExecResult{Erssat}{DefaultBddApplication}{Total}{}{Walltime}{Median}{901.0262540052645}%
\StoreBenchExecResult{Erssat}{DefaultBddApplication}{Total}{}{Walltime}{Min}{0.04257557261735201}%
\StoreBenchExecResult{Erssat}{DefaultBddApplication}{Total}{}{Walltime}{Max}{901.7038613380864}%
\StoreBenchExecResult{Erssat}{DefaultBddApplication}{Total}{}{Walltime}{Stdev}{409.8494678989227486890783493}%
\StoreBenchExecResult{Erssat}{DefaultBddApplication}{Error}{}{Count}{}{102}%
\StoreBenchExecResult{Erssat}{DefaultBddApplication}{Error}{}{Cputime}{}{90734.921256090}%
\StoreBenchExecResult{Erssat}{DefaultBddApplication}{Error}{}{Cputime}{Avg}{889.5580515302941176470588235}%
\StoreBenchExecResult{Erssat}{DefaultBddApplication}{Error}{}{Cputime}{Median}{901.022195085}%
\StoreBenchExecResult{Erssat}{DefaultBddApplication}{Error}{}{Cputime}{Min}{312.383213832}%
\StoreBenchExecResult{Erssat}{DefaultBddApplication}{Error}{}{Cputime}{Max}{901.683691398}%
\StoreBenchExecResult{Erssat}{DefaultBddApplication}{Error}{}{Cputime}{Stdev}{81.59444369319235672039935244}%
\StoreBenchExecResult{Erssat}{DefaultBddApplication}{Error}{}{Walltime}{}{90740.38185383751947}%
\StoreBenchExecResult{Erssat}{DefaultBddApplication}{Error}{}{Walltime}{Avg}{889.6115868023286222549019608}%
\StoreBenchExecResult{Erssat}{DefaultBddApplication}{Error}{}{Walltime}{Median}{901.13097573677075}%
\StoreBenchExecResult{Erssat}{DefaultBddApplication}{Error}{}{Walltime}{Min}{312.63300833757967}%
\StoreBenchExecResult{Erssat}{DefaultBddApplication}{Error}{}{Walltime}{Max}{901.7038613380864}%
\StoreBenchExecResult{Erssat}{DefaultBddApplication}{Error}{}{Walltime}{Stdev}{81.56961084934820997395047966}%
\StoreBenchExecResult{Erssat}{DefaultBddApplication}{Error}{SegmentationFault}{Count}{}{2}%
\StoreBenchExecResult{Erssat}{DefaultBddApplication}{Error}{SegmentationFault}{Cputime}{}{625.197866485}%
\StoreBenchExecResult{Erssat}{DefaultBddApplication}{Error}{SegmentationFault}{Cputime}{Avg}{312.5989332425}%
\StoreBenchExecResult{Erssat}{DefaultBddApplication}{Error}{SegmentationFault}{Cputime}{Median}{312.5989332425}%
\StoreBenchExecResult{Erssat}{DefaultBddApplication}{Error}{SegmentationFault}{Cputime}{Min}{312.383213832}%
\StoreBenchExecResult{Erssat}{DefaultBddApplication}{Error}{SegmentationFault}{Cputime}{Max}{312.814652653}%
\StoreBenchExecResult{Erssat}{DefaultBddApplication}{Error}{SegmentationFault}{Cputime}{Stdev}{0.21571941050000}%
\StoreBenchExecResult{Erssat}{DefaultBddApplication}{Error}{SegmentationFault}{Walltime}{}{625.65604413859547}%
\StoreBenchExecResult{Erssat}{DefaultBddApplication}{Error}{SegmentationFault}{Walltime}{Avg}{312.828022069297735}%
\StoreBenchExecResult{Erssat}{DefaultBddApplication}{Error}{SegmentationFault}{Walltime}{Median}{312.828022069297735}%
\StoreBenchExecResult{Erssat}{DefaultBddApplication}{Error}{SegmentationFault}{Walltime}{Min}{312.63300833757967}%
\StoreBenchExecResult{Erssat}{DefaultBddApplication}{Error}{SegmentationFault}{Walltime}{Max}{313.0230358010158}%
\StoreBenchExecResult{Erssat}{DefaultBddApplication}{Error}{SegmentationFault}{Walltime}{Stdev}{0.1950137317180650000000000000}%
\StoreBenchExecResult{Erssat}{DefaultBddApplication}{Error}{Timeout}{Count}{}{100}%
\StoreBenchExecResult{Erssat}{DefaultBddApplication}{Error}{Timeout}{Cputime}{}{90109.723389605}%
\StoreBenchExecResult{Erssat}{DefaultBddApplication}{Error}{Timeout}{Cputime}{Avg}{901.09723389605}%
\StoreBenchExecResult{Erssat}{DefaultBddApplication}{Error}{Timeout}{Cputime}{Median}{901.0225239675}%
\StoreBenchExecResult{Erssat}{DefaultBddApplication}{Error}{Timeout}{Cputime}{Min}{900.896737756}%
\StoreBenchExecResult{Erssat}{DefaultBddApplication}{Error}{Timeout}{Cputime}{Max}{901.683691398}%
\StoreBenchExecResult{Erssat}{DefaultBddApplication}{Error}{Timeout}{Cputime}{Stdev}{0.1271320476503559861839442074}%
\StoreBenchExecResult{Erssat}{DefaultBddApplication}{Error}{Timeout}{Walltime}{}{90114.7258096989240}%
\StoreBenchExecResult{Erssat}{DefaultBddApplication}{Error}{Timeout}{Walltime}{Avg}{901.14725809698924}%
\StoreBenchExecResult{Erssat}{DefaultBddApplication}{Error}{Timeout}{Walltime}{Median}{901.1404981263913}%
\StoreBenchExecResult{Erssat}{DefaultBddApplication}{Error}{Timeout}{Walltime}{Min}{901.0066848965362}%
\StoreBenchExecResult{Erssat}{DefaultBddApplication}{Error}{Timeout}{Walltime}{Max}{901.7038613380864}%
\StoreBenchExecResult{Erssat}{DefaultBddApplication}{Error}{Timeout}{Walltime}{Stdev}{0.1239300123881479746725328914}%
\StoreBenchExecResult{Erssat}{DefaultBddApplication}{Missing}{}{Count}{}{58}%
\StoreBenchExecResult{Erssat}{DefaultBddApplication}{Missing}{}{Cputime}{}{4698.775331084}%
\StoreBenchExecResult{Erssat}{DefaultBddApplication}{Missing}{}{Cputime}{Avg}{81.01336777731034482758620690}%
\StoreBenchExecResult{Erssat}{DefaultBddApplication}{Missing}{}{Cputime}{Median}{1.5293811165}%
\StoreBenchExecResult{Erssat}{DefaultBddApplication}{Missing}{}{Cputime}{Min}{0.040599129}%
\StoreBenchExecResult{Erssat}{DefaultBddApplication}{Missing}{}{Cputime}{Max}{864.799801094}%
\StoreBenchExecResult{Erssat}{DefaultBddApplication}{Missing}{}{Cputime}{Stdev}{186.7289887709010217860957077}%
\StoreBenchExecResult{Erssat}{DefaultBddApplication}{Missing}{}{Walltime}{}{4699.026571823284012168}%
\StoreBenchExecResult{Erssat}{DefaultBddApplication}{Missing}{}{Walltime}{Avg}{81.01769951419455193393103448}%
\StoreBenchExecResult{Erssat}{DefaultBddApplication}{Missing}{}{Walltime}{Median}{1.53087391564622525}%
\StoreBenchExecResult{Erssat}{DefaultBddApplication}{Missing}{}{Walltime}{Min}{0.04257557261735201}%
\StoreBenchExecResult{Erssat}{DefaultBddApplication}{Missing}{}{Walltime}{Max}{864.8248462686315}%
\StoreBenchExecResult{Erssat}{DefaultBddApplication}{Missing}{}{Walltime}{Stdev}{186.7342438000275984958325587}%
\StoreBenchExecResult{Erssat}{DefaultBddApplication}{Missing}{Done}{Count}{}{58}%
\StoreBenchExecResult{Erssat}{DefaultBddApplication}{Missing}{Done}{Cputime}{}{4698.775331084}%
\StoreBenchExecResult{Erssat}{DefaultBddApplication}{Missing}{Done}{Cputime}{Avg}{81.01336777731034482758620690}%
\StoreBenchExecResult{Erssat}{DefaultBddApplication}{Missing}{Done}{Cputime}{Median}{1.5293811165}%
\StoreBenchExecResult{Erssat}{DefaultBddApplication}{Missing}{Done}{Cputime}{Min}{0.040599129}%
\StoreBenchExecResult{Erssat}{DefaultBddApplication}{Missing}{Done}{Cputime}{Max}{864.799801094}%
\StoreBenchExecResult{Erssat}{DefaultBddApplication}{Missing}{Done}{Cputime}{Stdev}{186.7289887709010217860957077}%
\StoreBenchExecResult{Erssat}{DefaultBddApplication}{Missing}{Done}{Walltime}{}{4699.026571823284012168}%
\StoreBenchExecResult{Erssat}{DefaultBddApplication}{Missing}{Done}{Walltime}{Avg}{81.01769951419455193393103448}%
\StoreBenchExecResult{Erssat}{DefaultBddApplication}{Missing}{Done}{Walltime}{Median}{1.53087391564622525}%
\StoreBenchExecResult{Erssat}{DefaultBddApplication}{Missing}{Done}{Walltime}{Min}{0.04257557261735201}%
\StoreBenchExecResult{Erssat}{DefaultBddApplication}{Missing}{Done}{Walltime}{Max}{864.8248462686315}%
\StoreBenchExecResult{Erssat}{DefaultBddApplication}{Missing}{Done}{Walltime}{Stdev}{186.7342438000275984958325587}%
% The following definition defines a command for each value.
% The command name is the concatenation of the first six arguments.
% To override this definition, define \StoreBenchExecResult with \newcommand before including this file.
% Arguments: benchmark name, run-set name, category, status, column name, statistic, value
\providecommand\StoreBenchExecResult[7]{\expandafter\newcommand\csname#1#2#3#4#5#6\endcsname{#7}}%
\StoreBenchExecResult{Erssat}{BareBddApplication}{Total}{}{Count}{}{160}%
\StoreBenchExecResult{Erssat}{BareBddApplication}{Total}{}{Cputime}{}{87131.618210014}%
\StoreBenchExecResult{Erssat}{BareBddApplication}{Total}{}{Cputime}{Avg}{544.5726138125875}%
\StoreBenchExecResult{Erssat}{BareBddApplication}{Total}{}{Cputime}{Median}{900.9923350185}%
\StoreBenchExecResult{Erssat}{BareBddApplication}{Total}{}{Cputime}{Min}{0.043293474}%
\StoreBenchExecResult{Erssat}{BareBddApplication}{Total}{}{Cputime}{Max}{901.674077293}%
\StoreBenchExecResult{Erssat}{BareBddApplication}{Total}{}{Cputime}{Stdev}{429.8298373702258205108455494}%
\StoreBenchExecResult{Erssat}{BareBddApplication}{Total}{}{Walltime}{}{87137.005280911922004714}%
\StoreBenchExecResult{Erssat}{BareBddApplication}{Total}{}{Walltime}{Avg}{544.6062830056995125294625}%
\StoreBenchExecResult{Erssat}{BareBddApplication}{Total}{}{Walltime}{Median}{901.0144369280897}%
\StoreBenchExecResult{Erssat}{BareBddApplication}{Total}{}{Walltime}{Min}{0.0445504579693079}%
\StoreBenchExecResult{Erssat}{BareBddApplication}{Total}{}{Walltime}{Max}{901.6921367570758}%
\StoreBenchExecResult{Erssat}{BareBddApplication}{Total}{}{Walltime}{Stdev}{429.8456916725929113683152270}%
\StoreBenchExecResult{Erssat}{BareBddApplication}{Error}{}{Count}{}{98}%
\StoreBenchExecResult{Erssat}{BareBddApplication}{Error}{}{Cputime}{}{84706.308552562}%
\StoreBenchExecResult{Erssat}{BareBddApplication}{Error}{}{Cputime}{Avg}{864.3500872710408163265306122}%
\StoreBenchExecResult{Erssat}{BareBddApplication}{Error}{}{Cputime}{Median}{901.013068872}%
\StoreBenchExecResult{Erssat}{BareBddApplication}{Error}{}{Cputime}{Min}{0.761268082}%
\StoreBenchExecResult{Erssat}{BareBddApplication}{Error}{}{Cputime}{Max}{901.674077293}%
\StoreBenchExecResult{Erssat}{BareBddApplication}{Error}{}{Cputime}{Stdev}{178.0642449299897837836263060}%
\StoreBenchExecResult{Erssat}{BareBddApplication}{Error}{}{Walltime}{}{84711.5119971176604861}%
\StoreBenchExecResult{Erssat}{BareBddApplication}{Error}{}{Walltime}{Avg}{864.4031836440577600622448980}%
\StoreBenchExecResult{Erssat}{BareBddApplication}{Error}{}{Walltime}{Median}{901.1240388415754}%
\StoreBenchExecResult{Erssat}{BareBddApplication}{Error}{}{Walltime}{Min}{0.9539869418367743}%
\StoreBenchExecResult{Erssat}{BareBddApplication}{Error}{}{Walltime}{Max}{901.6921367570758}%
\StoreBenchExecResult{Erssat}{BareBddApplication}{Error}{}{Walltime}{Stdev}{178.0357486274157922587281994}%
\StoreBenchExecResult{Erssat}{BareBddApplication}{Error}{SegmentationFault}{Count}{}{4}%
\StoreBenchExecResult{Erssat}{BareBddApplication}{Error}{SegmentationFault}{Cputime}{}{4.607598963}%
\StoreBenchExecResult{Erssat}{BareBddApplication}{Error}{SegmentationFault}{Cputime}{Avg}{1.15189974075}%
\StoreBenchExecResult{Erssat}{BareBddApplication}{Error}{SegmentationFault}{Cputime}{Median}{1.099821245}%
\StoreBenchExecResult{Erssat}{BareBddApplication}{Error}{SegmentationFault}{Cputime}{Min}{0.761268082}%
\StoreBenchExecResult{Erssat}{BareBddApplication}{Error}{SegmentationFault}{Cputime}{Max}{1.646688391}%
\StoreBenchExecResult{Erssat}{BareBddApplication}{Error}{SegmentationFault}{Cputime}{Stdev}{0.3937375208343058270030168665}%
\StoreBenchExecResult{Erssat}{BareBddApplication}{Error}{SegmentationFault}{Walltime}{}{5.3725353293120861}%
\StoreBenchExecResult{Erssat}{BareBddApplication}{Error}{SegmentationFault}{Walltime}{Avg}{1.343133832328021525}%
\StoreBenchExecResult{Erssat}{BareBddApplication}{Error}{SegmentationFault}{Walltime}{Median}{1.2907491861842573}%
\StoreBenchExecResult{Erssat}{BareBddApplication}{Error}{SegmentationFault}{Walltime}{Min}{0.9539869418367743}%
\StoreBenchExecResult{Erssat}{BareBddApplication}{Error}{SegmentationFault}{Walltime}{Max}{1.8370500151067972}%
\StoreBenchExecResult{Erssat}{BareBddApplication}{Error}{SegmentationFault}{Walltime}{Stdev}{0.3923080483709211947621329156}%
\StoreBenchExecResult{Erssat}{BareBddApplication}{Error}{Timeout}{Count}{}{94}%
\StoreBenchExecResult{Erssat}{BareBddApplication}{Error}{Timeout}{Cputime}{}{84701.700953599}%
\StoreBenchExecResult{Erssat}{BareBddApplication}{Error}{Timeout}{Cputime}{Avg}{901.0819250382872340425531915}%
\StoreBenchExecResult{Erssat}{BareBddApplication}{Error}{Timeout}{Cputime}{Median}{901.015790674}%
\StoreBenchExecResult{Erssat}{BareBddApplication}{Error}{Timeout}{Cputime}{Min}{900.890174616}%
\StoreBenchExecResult{Erssat}{BareBddApplication}{Error}{Timeout}{Cputime}{Max}{901.674077293}%
\StoreBenchExecResult{Erssat}{BareBddApplication}{Error}{Timeout}{Cputime}{Stdev}{0.1173521311118864200217232826}%
\StoreBenchExecResult{Erssat}{BareBddApplication}{Error}{Timeout}{Walltime}{}{84706.1394617883484}%
\StoreBenchExecResult{Erssat}{BareBddApplication}{Error}{Timeout}{Walltime}{Avg}{901.1291432105143446808510638}%
\StoreBenchExecResult{Erssat}{BareBddApplication}{Error}{Timeout}{Walltime}{Median}{901.1370040657930}%
\StoreBenchExecResult{Erssat}{BareBddApplication}{Error}{Timeout}{Walltime}{Min}{901.0070397015661}%
\StoreBenchExecResult{Erssat}{BareBddApplication}{Error}{Timeout}{Walltime}{Max}{901.6921367570758}%
\StoreBenchExecResult{Erssat}{BareBddApplication}{Error}{Timeout}{Walltime}{Stdev}{0.1164837688602781020101066434}%
\StoreBenchExecResult{Erssat}{BareBddApplication}{Missing}{}{Count}{}{62}%
\StoreBenchExecResult{Erssat}{BareBddApplication}{Missing}{}{Cputime}{}{2425.309657452}%
\StoreBenchExecResult{Erssat}{BareBddApplication}{Missing}{}{Cputime}{Avg}{39.11789770083870967741935484}%
\StoreBenchExecResult{Erssat}{BareBddApplication}{Missing}{}{Cputime}{Median}{1.846133201}%
\StoreBenchExecResult{Erssat}{BareBddApplication}{Missing}{}{Cputime}{Min}{0.043293474}%
\StoreBenchExecResult{Erssat}{BareBddApplication}{Missing}{}{Cputime}{Max}{495.99237704}%
\StoreBenchExecResult{Erssat}{BareBddApplication}{Missing}{}{Cputime}{Stdev}{97.71843544911616823158405099}%
\StoreBenchExecResult{Erssat}{BareBddApplication}{Missing}{}{Walltime}{}{2425.493283794261518614}%
\StoreBenchExecResult{Erssat}{BareBddApplication}{Missing}{}{Walltime}{Avg}{39.12085941603647610667741935}%
\StoreBenchExecResult{Erssat}{BareBddApplication}{Missing}{}{Walltime}{Median}{1.847707605920732}%
\StoreBenchExecResult{Erssat}{BareBddApplication}{Missing}{}{Walltime}{Min}{0.0445504579693079}%
\StoreBenchExecResult{Erssat}{BareBddApplication}{Missing}{}{Walltime}{Max}{496.00512275565416}%
\StoreBenchExecResult{Erssat}{BareBddApplication}{Missing}{}{Walltime}{Stdev}{97.72114502500470110460819337}%
\StoreBenchExecResult{Erssat}{BareBddApplication}{Missing}{Done}{Count}{}{62}%
\StoreBenchExecResult{Erssat}{BareBddApplication}{Missing}{Done}{Cputime}{}{2425.309657452}%
\StoreBenchExecResult{Erssat}{BareBddApplication}{Missing}{Done}{Cputime}{Avg}{39.11789770083870967741935484}%
\StoreBenchExecResult{Erssat}{BareBddApplication}{Missing}{Done}{Cputime}{Median}{1.846133201}%
\StoreBenchExecResult{Erssat}{BareBddApplication}{Missing}{Done}{Cputime}{Min}{0.043293474}%
\StoreBenchExecResult{Erssat}{BareBddApplication}{Missing}{Done}{Cputime}{Max}{495.99237704}%
\StoreBenchExecResult{Erssat}{BareBddApplication}{Missing}{Done}{Cputime}{Stdev}{97.71843544911616823158405099}%
\StoreBenchExecResult{Erssat}{BareBddApplication}{Missing}{Done}{Walltime}{}{2425.493283794261518614}%
\StoreBenchExecResult{Erssat}{BareBddApplication}{Missing}{Done}{Walltime}{Avg}{39.12085941603647610667741935}%
\StoreBenchExecResult{Erssat}{BareBddApplication}{Missing}{Done}{Walltime}{Median}{1.847707605920732}%
\StoreBenchExecResult{Erssat}{BareBddApplication}{Missing}{Done}{Walltime}{Min}{0.0445504579693079}%
\StoreBenchExecResult{Erssat}{BareBddApplication}{Missing}{Done}{Walltime}{Max}{496.00512275565416}%
\StoreBenchExecResult{Erssat}{BareBddApplication}{Missing}{Done}{Walltime}{Stdev}{97.72114502500470110460819337}%
\ifdefined\DcssatErDefaultApplicationTotalCount\else\edef\DcssatErDefaultApplicationTotalCount{0}\fi
\ifdefined\DcssatErDefaultApplicationCorrectCount\else\edef\DcssatErDefaultApplicationCorrectCount{0}\fi
\ifdefined\DcssatErDefaultApplicationCorrectTrueCount\else\edef\DcssatErDefaultApplicationCorrectTrueCount{0}\fi
\ifdefined\DcssatErDefaultApplicationCorrectFalseCount\else\edef\DcssatErDefaultApplicationCorrectFalseCount{0}\fi
\ifdefined\DcssatErDefaultApplicationWrongTrueCount\else\edef\DcssatErDefaultApplicationWrongTrueCount{0}\fi
\ifdefined\DcssatErDefaultApplicationWrongFalseCount\else\edef\DcssatErDefaultApplicationWrongFalseCount{0}\fi
\ifdefined\DcssatErDefaultApplicationErrorTimeoutCount\else\edef\DcssatErDefaultApplicationErrorTimeoutCount{0}\fi
\ifdefined\DcssatErDefaultApplicationErrorOutOfMemoryCount\else\edef\DcssatErDefaultApplicationErrorOutOfMemoryCount{0}\fi
\ifdefined\DcssatErDefaultApplicationCorrectCputime\else\edef\DcssatErDefaultApplicationCorrectCputime{0}\fi
\ifdefined\DcssatErDefaultApplicationCorrectCputimeAvg\else\edef\DcssatErDefaultApplicationCorrectCputimeAvg{None}\fi
\ifdefined\DcssatErDefaultApplicationCorrectWalltime\else\edef\DcssatErDefaultApplicationCorrectWalltime{0}\fi
\ifdefined\DcssatErDefaultApplicationCorrectWalltimeAvg\else\edef\DcssatErDefaultApplicationCorrectWalltimeAvg{None}\fi
\ifdefined\ErssatDefaultBddApplicationTotalCount\else\edef\ErssatDefaultBddApplicationTotalCount{0}\fi
\ifdefined\ErssatDefaultBddApplicationCorrectCount\else\edef\ErssatDefaultBddApplicationCorrectCount{0}\fi
\ifdefined\ErssatDefaultBddApplicationCorrectTrueCount\else\edef\ErssatDefaultBddApplicationCorrectTrueCount{0}\fi
\ifdefined\ErssatDefaultBddApplicationCorrectFalseCount\else\edef\ErssatDefaultBddApplicationCorrectFalseCount{0}\fi
\ifdefined\ErssatDefaultBddApplicationWrongTrueCount\else\edef\ErssatDefaultBddApplicationWrongTrueCount{0}\fi
\ifdefined\ErssatDefaultBddApplicationWrongFalseCount\else\edef\ErssatDefaultBddApplicationWrongFalseCount{0}\fi
\ifdefined\ErssatDefaultBddApplicationErrorTimeoutCount\else\edef\ErssatDefaultBddApplicationErrorTimeoutCount{0}\fi
\ifdefined\ErssatDefaultBddApplicationErrorOutOfMemoryCount\else\edef\ErssatDefaultBddApplicationErrorOutOfMemoryCount{0}\fi
\ifdefined\ErssatDefaultBddApplicationCorrectCputime\else\edef\ErssatDefaultBddApplicationCorrectCputime{0}\fi
\ifdefined\ErssatDefaultBddApplicationCorrectCputimeAvg\else\edef\ErssatDefaultBddApplicationCorrectCputimeAvg{None}\fi
\ifdefined\ErssatDefaultBddApplicationCorrectWalltime\else\edef\ErssatDefaultBddApplicationCorrectWalltime{0}\fi
\ifdefined\ErssatDefaultBddApplicationCorrectWalltimeAvg\else\edef\ErssatDefaultBddApplicationCorrectWalltimeAvg{None}\fi
\ifdefined\ErssatBareBddApplicationTotalCount\else\edef\ErssatBareBddApplicationTotalCount{0}\fi
\ifdefined\ErssatBareBddApplicationCorrectCount\else\edef\ErssatBareBddApplicationCorrectCount{0}\fi
\ifdefined\ErssatBareBddApplicationCorrectTrueCount\else\edef\ErssatBareBddApplicationCorrectTrueCount{0}\fi
\ifdefined\ErssatBareBddApplicationCorrectFalseCount\else\edef\ErssatBareBddApplicationCorrectFalseCount{0}\fi
\ifdefined\ErssatBareBddApplicationWrongTrueCount\else\edef\ErssatBareBddApplicationWrongTrueCount{0}\fi
\ifdefined\ErssatBareBddApplicationWrongFalseCount\else\edef\ErssatBareBddApplicationWrongFalseCount{0}\fi
\ifdefined\ErssatBareBddApplicationErrorTimeoutCount\else\edef\ErssatBareBddApplicationErrorTimeoutCount{0}\fi
\ifdefined\ErssatBareBddApplicationErrorOutOfMemoryCount\else\edef\ErssatBareBddApplicationErrorOutOfMemoryCount{0}\fi
\ifdefined\ErssatBareBddApplicationCorrectCputime\else\edef\ErssatBareBddApplicationCorrectCputime{0}\fi
\ifdefined\ErssatBareBddApplicationCorrectCputimeAvg\else\edef\ErssatBareBddApplicationCorrectCputimeAvg{None}\fi
\ifdefined\ErssatBareBddApplicationCorrectWalltime\else\edef\ErssatBareBddApplicationCorrectWalltime{0}\fi
\ifdefined\ErssatBareBddApplicationCorrectWalltimeAvg\else\edef\ErssatBareBddApplicationCorrectWalltimeAvg{None}\fi
\edef\DcssatErDefaultApplicationErrorOtherInconclusiveCount{\the\numexpr \DcssatErDefaultApplicationTotalCount - \DcssatErDefaultApplicationMissingCount - \DcssatErDefaultApplicationErrorTimeoutCount - \DcssatErDefaultApplicationErrorOutOfMemoryCount \relax}
\edef\ErssatDefaultBddApplicationErrorOtherInconclusiveCount{\the\numexpr \ErssatDefaultBddApplicationTotalCount - \ErssatDefaultBddApplicationMissingCount - \ErssatDefaultBddApplicationErrorTimeoutCount - \ErssatDefaultBddApplicationErrorOutOfMemoryCount \relax}
\edef\ErssatBareBddApplicationErrorOtherInconclusiveCount{\the\numexpr \ErssatBareBddApplicationTotalCount - \ErssatBareBddApplicationMissingCount - \ErssatBareBddApplicationErrorTimeoutCount - \ErssatBareBddApplicationErrorOutOfMemoryCount \relax}
% Commands for application formulas of ER-SSAT
\newcommand{\dcssatToiletA}{44}
\newcommand{\dcssatconformant}{1}
\newcommand{\dcssatcastle}{21}
\newcommand{\dcssatMaxCount}{2}
\newcommand{\dcssatMPEC}{3}
\newcommand{\erssatbToiletA}{45}
\newcommand{\erssatbconformant}{1}
\newcommand{\erssatbcastle}{14}
\newcommand{\erssatbMaxCount}{1}
\newcommand{\erssatbMPEC}{1}
\newcommand{\erssatToiletA}{38}
\newcommand{\erssatconformant}{2}
\newcommand{\erssatcastle}{13}
\newcommand{\erssatMaxCount}{3}
\newcommand{\erssatMPEC}{2}

\begin{table}[t]
    \centering
    \caption{Summary of the results for~\napplication~application formulas}
    \label{tbl:exist-random-ssat-application}
    \begin{tabular}{l|ccc}
        \toprule
        Algorithm                   & {\dcssat}                                                     & {\erssat} & {\erssatb} \\
        \midrule
        Solved formulas             & \num{\DcssatErDefaultApplicationMissingCount}
                                    & \num{\ErssatDefaultBddApplicationMissingCount}
                                    & \num{\ErssatBareBddApplicationMissingCount}                                            \\
        \qquad \textit{Toilet-A}    & \num{\dcssatToiletA}
                                    & \num{\erssatToiletA}
                                    & \num{\erssatbToiletA}                                                                  \\
        \qquad \textit{Conformant}  & \num{\dcssatconformant}
                                    & \num{\erssatconformant}
                                    & \num{\erssatbconformant}                                                               \\
        \qquad \textit{Sand-Castle} & \num{\dcssatcastle}
                                    & \num{\erssatcastle}
                                    & \num{\erssatbcastle}                                                                   \\
        \qquad \textit{Max-Count}   & \num{\dcssatMaxCount}
                                    & \num{\erssatMaxCount}
                                    & \num{\erssatbMaxCount}                                                                 \\
        \qquad \textit{MPEC}        & \num{\dcssatMPEC}
                                    & \num{\erssatMPEC}
                                    & \num{\erssatbMPEC}                                                                     \\
        Timeouts                    & \num{\DcssatErDefaultApplicationErrorTimeoutCount}
                                    & \num{\ErssatDefaultBddApplicationErrorTimeoutCount}
                                    & \num{\ErssatBareBddApplicationErrorTimeoutCount}                                       \\
        Out of memory               & \num{\DcssatErDefaultApplicationErrorOutOfMemoryCount}
                                    & \num{\ErssatDefaultBddApplicationErrorOutOfMemoryCount}
                                    & \num{\ErssatBareBddApplicationErrorOutOfMemoryCount}                                   \\
        Other inconclusive          & \num{\DcssatErDefaultApplicationErrorOtherInconclusiveCount}
                                    & \num{\ErssatDefaultBddApplicationErrorOtherInconclusiveCount}
                                    & \num{\ErssatBareBddApplicationErrorOtherInconclusiveCount}                             \\
        \bottomrule
    \end{tabular}
\end{table}

The solving results of the application formulas are summarized in~\cref{tbl:exist-random-ssat-application}.
For each compared approach,
the numbers of its exactly solved formulas,
timeouts, out of memory, and other inconclusive situations are reported.
To study the solving performance regarding different kinds of formulas,
we further report the numbers of exactly solved formulas per family.
Observe that \dcssat exactly solved the most formulas.
Its advantage mainly comes from family \textit{Sand-Castle},
where it solved \num{22} formulas,
but \erssat and \erssatb only solved \num{13} and \num{14} formulas, respectively.
We will analyze why the proposed clause-containment learning is not suitable for this family later.
To our surprise, the proposed clause-strengthening heuristics seem not very useful on the evaluated application formulas.
They even worsened the performance over formulas from the family \textit{Toilet-A}.
While \erssat and \erssatb suffered from more timeouts than \dcssat,
they did not run out of memory for any formula.
Instead, \dcssat tends to consume a lot of memory, because it memorizes many subformulas.

\begin{figure*}[hp]
    \centering
    \subfloat[CPU time]{
        \includegraphics{exist-random-ssat/evaluation/plots/quantile-cputime-Application.pdf}
        \label{fig:erssat-quantile-cputime-application}
    }\\
    \subfloat[Memory usage]{
        \includegraphics{exist-random-ssat/evaluation/plots/quantile-memory-Application.pdf}
        \label{fig:erssat-quantile-memory-application}
    }
    \caption{Quantile plots of application formulas}
    \label{fig:erssat-quantile-application}
\end{figure*}

\cref{fig:erssat-quantile-application} shows the quantile plots of the application SSAT formulas.
We can see that the clause-strengthening heuristics affected not only the effectiveness of \erssat but also its efficiency
from~\cref{fig:erssat-quantile-cputime-application}.
This phenomenon indicates that the additional effort spent to strengthen a learnt clause is not worthy.
The reason behind this phenomenon will be inspected in the following.

\begin{figure*}[hp]
    \centering
    \subfloat[\erssatb]{
        \includegraphics{exist-random-ssat/evaluation/plots/scatter-erssat.pdf}
        \label{fig:erssat-scatter-cputime-application}
    }\\
    \subfloat[\dcssat]{
        \includegraphics{exist-random-ssat/evaluation/plots/scatter-dcssat.pdf}
        \label{fig:dcssat-scatter-cputime-application}
    }
    \caption{Run-time scatter plots of application formulas with \erssat in y-axis and compared approaches in x-axis}
    \label{fig:erssat-scatter-application}
\end{figure*}

To further examine the suitability of the clause-strengthening heuristics,
we demonstrate the scatter plots with \erssat in y-axis and compared approaches in x-axis
in~\cref{fig:erssat-scatter-application}.
A data point $(x,y)$ in the plots indicates that there is a formula processed by both \erssat and a compared approach,
while \erssat took a CPU time of $y$~seconds and the other approach took a CPU time of $x$~seconds.
From~\cref{fig:erssat-scatter-cputime-application},
we find that the clause-strengthening heuristics did improve the solving of some formulas,
but more often they were an overhead to \erssatb.
\cref{fig:dcssat-scatter-cputime-application} also shows that
\dcssat was more efficient to exactly solve formulas than \erssat over the evaluated application formulas.

\subsubsection{Clause-containment learning over the \textit{Sand-Castle} problem}
As \erssat and \erssatb did not solve formulas from the probabilistic planning domain \textit{Sand-Castle} quite well,
we look into the problem and discuss our findings here.
The \textit{Sand-Castle} problem~\cite{Majercik1998} describes an agent who wants to build a sand castle on a beach.
The agent has two actions to choose from: digging a moat or erecting a castle.
A moat protects a castle from the water and increases the probability to successfully build a castle.
The agent must take a unique action at every stage.
Under the settings of conformant planning,
the agent must decide its strategy beforehand and does not have access to internal states
(whether a moat has been digged or a castle has been erected) during the execution.
Given a finite number of stages,
the problem asks to compute a strategy to maximize the chance of successfully building a castle at the last stage.
The agent's actions are encoded with existentially quantified variables,
and the nondeterminism in the state-transition mechanism is encoded with randomly quantified variables.

A formula that encodes the \textit{Sand-Castle} problem with $n$ stages has the form:
\begin{align}
    \pf=\bigwedge_{i=1}^n \pf_d^{(i)}\land\pf_e^{(i)},
\end{align}
where $\pf_d$ and $\pf_e$ are sets of clauses used to represent the state-transition mechanism
when the agent chooses to \textit{dig} a moat or \textit{erect} a castle, respectively.
As the state-transition mechanism is the same for every stage except for the variables recording the internal state,
we use the superscripts to indicate the stage indices.
We found that the \textit{Sand-Castle} problem has the following property:
the set of clauses $\pf_d^{(i)}$ (resp. $\pf_e^{(i)}$) will be selected
if and only if the agent chooses to dig a moat (resp. erect a castle) at stage~$i$.
In other words, each strategy of the agent (i.e., an assignment to the existentially quantified variables)
will select a distinct set of clauses.
Recall that the proposed clause-containment principle aims at blocking assignments
selecting a superset of clauses that has been selected by a previously explored assignment.
As a result, a learnt clause constructed based on this principle can only block the current assignment itself,
which means that \erssatb degenerates to merely brute-force search.
This theoretic reasoning is confirmed by the solving statistics (visible from the log files),
which show that \erssatb invoked $2^n-1$ model-counting queries.
For \erssat, the situation is worse due to partial assignment pruning,
which invokes additional model-counting queries to strengthen a learnt clause.
From the log files, we found that \erssat invoked twice numbers of model-counting queries than \erssatb,
because it had an additional trial but always ended in vain.

On the other hand, recall that \dcssat is tailored to exploit the structural characteristics of planning problems.
The \textit{Sand-Castle} formulas favors \dcssat, as the subformulas are essentially the same across the stages.
It is not surprising the formula caching and divide-and-conquer method work well with these formulas.

\subsubsection{Approximate solving}

\begin{table}[ht]
    \centering
    \scriptsize
    \caption{Results of solving the \textit{Conformant} family}
    \label{tbl:exist-random-ssat-conformant}
    \begin{adjustbox}{angle=90}
        \pgfplotstabletypeset[
            every head row/.style={before row={\toprule
                            & \multicolumn{4}{c}{\dcssat} & \multicolumn{8}{c}{\erssat} & \multicolumn{8}{c}{\erssatb}\\},after row=\midrule},
            every last row/.style={after row=\bottomrule},
            empty cells with={--},
            formula column/.list={0},
            time column/.list={1,3,7},
            prob column/.list={2,4,8},
            lbound column/.list={5,9},
            lbtime column/.list={6,10}
        ]
        {exist-random-ssat/evaluation/csv/parsed-conformant.csv}
    \end{adjustbox}
\end{table}

\begin{table}[ht]
    \centering
    \scriptsize
    \caption{Results of solving the \textit{Max-Count} family}
    \label{tbl:exist-random-ssat-maxcount}
    \begin{adjustbox}{angle=90}
        \pgfplotstabletypeset[
            every head row/.style={before row={\toprule
                            & \multicolumn{4}{c}{\dcssat} & \multicolumn{8}{c}{\erssat} & \multicolumn{8}{c}{\erssatb}\\},after row=\midrule},
            every last row/.style={after row=\bottomrule},
            empty cells with={--},
            formula column/.list={0},
            time column/.list={1,3,7},
            prob column/.list={2,4,8},
            lbound column/.list={5,9},
            lbtime column/.list={6,10}
        ]
        {exist-random-ssat/evaluation/csv/parsed-MaxCount.csv}
    \end{adjustbox}
\end{table}

\begin{table}[ht]
    \centering
    \scriptsize
    \caption{Results of solving the \textit{MPEC} family}
    \label{tbl:exist-random-ssat-mpec}
    \begin{adjustbox}{angle=90}
        \pgfplotstabletypeset[
            every head row/.style={before row={\toprule
                            & \multicolumn{4}{c}{\dcssat} & \multicolumn{8}{c}{\erssat} & \multicolumn{8}{c}{\erssatb}\\},after row=\midrule},
            every last row/.style={after row=\bottomrule},
            empty cells with={--},
            formula column/.list={0},
            time column/.list={1,3,7},
            prob column/.list={2,4,8},
            lbound column/.list={5,9},
            lbtime column/.list={6,10}
        ]
        {exist-random-ssat/evaluation/csv/parsed-MPEC.csv}
    \end{adjustbox}
\end{table}

Recall that the proposed~\cref{alg:erssat} solves an SSAT formula in a converging manner.
Instead of computing the exact satisfying probability at once,
it keeps deriving lower bounds of the satisfying probability of a formula.
This characteristic integrates exact and approximate solving into one approach.
In the following, we study the approximation ability of \erssat.
We choose families \textit{Conformant}, \textit{Max-Count}, and \textit{MPEC} for detailed investigation,
because all of the compared approaches ran out of CPU time or memory over most of their formulas.

\cref{tbl:exist-random-ssat-conformant,tbl:exist-random-ssat-maxcount,tbl:exist-random-ssat-mpec}
show the approximation results over the above three families, respectively.
For \dcssat, the CPU time and exact satisfying probability are reported.
For \erssat and \erssatb, in addition to the CPU and exact satisfying probability,
the tightest lower bound and the time elapsed to derive the lower bound are also shown in the tables.
A formula is not shown in the tables
if none of the approaches can solve it or derive a non-trivial lower bound for it.

As can be observed from the tables,
\erssat was able to derive tight lower bounds for formulas from these families,
while \dcssat suffered from timeouts over most of them.
The approximation ability of \erssat makes it useful for large formulas,
which cannot be exactly solved by the state-of-the-art approaches.

The above results on the random and application formulas suggest that:
\begin{itemize}
    \item The proposed solver \erssat achieves a similar performance as \dcssat in terms of CPU time and outperforms \dcssat in terms of memory consumption on random formulas.
    \item The proposed solver \erssat is not as good as \dcssat at exactly solving the application formulas, which can be attributed to the overhead caused by the clause-strengthening heuristics.
    \item The proposed solver \erssat is good at deriving tight lower bounds for large formulas. This approximation ability is especially valuable when the size of a formula is beyond the capability of the state-of-the-art exact solver.
\end{itemize}
To sum up, our evaluation results demonstrate the unique value of the proposed clause-containment learning.