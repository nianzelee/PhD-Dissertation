\newcommand{\nrandom}{\num{700}}
\newcommand{\napplication}{\num{160}}
\newcommand{\ntoilet}{\num{77}}
\newcommand{\nmaxcount}{\num{26}}
\newcommand{\nsandcastle}{\num{25}}
\newcommand{\nconformant}{\num{24}}
\newcommand{\nmpec}{\num{8}}

\section{Evaluation}
\label{sect:erssat-evaluation}

We evaluated the proposed~\cref{alg:erssat} against
the state-of-the-art DPLL-based SSAT solver \dcssat~\cite{Majercik2005}
over both random $k$-CNF and application formulas.
The proposed algorithm is implemented in the \texttt{C++} language inside the \abc~\cite{ABC} environment.
The SAT solver \minisat-2.2~\cite{Een2003Solver} is used to answer satisfiability queries.
For weighted model counting,
we tried \cachet~\cite{Sang2004,Sang2005ModelCounting},
but the overall performance was not satisfactory.
Instead, we adopted the well-developed BDD package \cudd~\cite{CUDD}.
Weight computation of a formula is fulfilled a classic approach~\cite{Darwiche2002KnowledgeCompilation} that traverses the BDD built for the formula.
Our prototyping implementation\footnote{Available at: \url{\ssatabcurl}} is named \erssat.
A bare version of \erssat without the enhancement techniques is called \erssatb in the experiments.

\subsection{Benchmark set}

\subsubsection{Random $k$-CNF formulas}

The random $k$-CNF formulas were generated by \cnfgen~\cite{Lauria2017CNFgen}.
A collection of~\nrandom~formulas were generated with $k$,
i.e., the number of literals in a clause,
ranging in $\{3,4,5,6,7,8,9\}$,
the number of variables ranging in $\{10,20,30,40,50\}$,
and clause-to-variable ratio ranging in $\{k-1,k,k+1,k+2\}$.
Five formulas were sampled for each parameter combination.
To convert the propositional formulas into E-MAJSAT formulas,
half of the variables are existentially quantified, and the rest are randomly quantified with probability $0.37$.

\subsubsection{Application formulas}

\begin{table}[t]
    \centering
    \caption{The families of the application formulas}
    \label{tbl:exist-random-ssat-families}
    \begin{tabular}{c|c|c}
        Family               & Description                                            & Number       \\
        \hline
        \textit{Toilet-A}    & Adapted from exist-forall QBFs~\cite{Narizzano2006}    & \ntoilet     \\
        \textit{Conformant}  & Adapted from exist-forall QBFs~\cite{Narizzano2006}    & \nconformant \\
        \textit{Sand-Castle} & A probabilistic planning problem~\cite{Majercik1998}   & \nsandcastle \\
        \textit{Max-Count}   & Adapted from maximum model counting~\cite{Fremont2017} & \nmaxcount   \\
        \textit{MPEC}        & Probabilistic design equivalence checking              & \nmpec       \\
    \end{tabular}
\end{table}

We collected five families of application formulas for evaluation.
Their descriptions and the numbers of the instances in each family are summarized in~\cref{tbl:exist-random-ssat-families}.
The first two families,
\textit{Toilet-A} and \textit{Conformant},
are adapted from exist-forall-exist QBFs~\cite{Narizzano2006}.
We converted the QBFs into exist-random-exist quantified SSAT formulas
by replacing their universal quantifiers with randomized ones with probabilities $0.5$.
The third family \textit{Sand-Castle} is a probabilistic conformant planning domain.
The problem can be encoded as E-MAJSAT formulas~\cite{Majercik1998}.
The family \textit{Max-Count} encodes the problems of maximum satisfiability, quantitative information flow, and program synthesis into maximum model counting~\cite{Fremont2017}.
We represented the maximum model counting instances as E-MAJSAT formulas.
The last family encodes the MPEC problem of analyzing the maximum probability for a probabilistic circuit to produce erroneous outputs.
%The MPEC formulas for evaluation were created from \texttt{ISCAS} benchmark suite with the erroneous and defective rates equal to $0.125$ and $0.01$.

\subsection{Experimental setup}
Our experiments were performed on a machine with
one 2.2\,GHz CPU (Intel Xeon Silver 4210) with 40~processing units and 134616\,MB of RAM.
The operating system was Ubuntu~20.04 (64~bit),
using Linux~5.4.
The programs were compiled with \texttt{g++ 9.3.0}.
Each SSAT-solving task was limited to a CPU core,
a CPU time of \SI{15}{min},
and a memory usage of \SI{8}{GB}.
To achieve reliable benchmarking,
we used a benchmarking framework \benchexec\footnote{Available at: \url{\benchexecurl}}~\cite{Benchmarking-STTT}.
%%% TODO: fix the evaluation commit
%and \experimentRevision of \cpachecker for evaluation.

\subsection{Results}

\begin{figure*}[ht]
    \centering
    \subfloat[CPU time]{
        % This file is part of BenchExec, a framework for reliable benchmarking:
% https://github.com/sosy-lab/benchexec
%
% SPDX-FileCopyrightText: 2007-2020 Dirk Beyer <https://www.sosy-lab.org>
%
% SPDX-License-Identifier: Apache-2.0

% LaTeX code for a quantile plot
% Copy the tikzpicture environment to your own document
% and make sure the siunitx and pgfplots package are loaded,
% possibly with the options suggested here.
\documentclass{standalone}
\usepackage[
    group-digits=integer, group-minimum-digits=4, % group digits by thousands
    free-standing-units, unit-optional-argument, % easier input of numbers with units
]{siunitx}
\usepackage{pgfplots}
\pgfplotsset{
    compat=1.9,
    log ticks with fixed point, % no scientific notation in plots
    table/col sep=tab, % only tabs are column separators
    unbounded coords=jump, % better have skips in a plot than appear to be interpolating
    filter discard warning=false, % Don't complain about empty cells
}
\SendSettingsToPgf % use siunitx formatting settings in PGF, too

\begin{document}

\begin{tikzpicture}
    \begin{semilogyaxis}[
            % Which column to be taken from each file
            /pgfplots/table/y index=3,
            /pgfplots/table/header=false,
            % axis labels
            xlabel=n-th fastest result,
            ylabel=Time (\second),
            % axis ranges
            xmin=0,
            ymin=0.001,
            ymax=1000,
            mark repeat=100,
            % legend
            legend entries={\dcssat,\erssatb,\erssat},
            every axis legend/.append style={at={(1,0)}, anchor=south east, outer xsep=5pt, outer ysep=5pt,},
        ]
        \addplot+ table {exist-random-ssat/evaluation/csv/dcssat.default.Random.quantile.cputime.csv};
        \addplot+ table {exist-random-ssat/evaluation/csv/erssat.bare-BDD.Random.quantile.cputime.csv};
        \addplot+ table {exist-random-ssat/evaluation/csv/erssat.default-BDD.Random.quantile.cputime.csv};
    \end{semilogyaxis}
\end{tikzpicture}

\end{document}

        \label{fig:erssat-quantile-cputime-random}
    }\\
    \subfloat[Memory usage]{
        % This file is part of BenchExec, a framework for reliable benchmarking:
% https://github.com/sosy-lab/benchexec
%
% SPDX-FileCopyrightText: 2007-2020 Dirk Beyer <https://www.sosy-lab.org>
%
% SPDX-License-Identifier: Apache-2.0

% LaTeX code for a quantile plot
% Copy the tikzpicture environment to your own document
% and make sure the siunitx and pgfplots package are loaded,
% possibly with the options suggested here.
\documentclass{standalone}
\usepackage[
    group-digits=integer, group-minimum-digits=4, % group digits by thousands
    free-standing-units, unit-optional-argument, % easier input of numbers with units
]{siunitx}
\usepackage{pgfplots}
\pgfplotsset{
    compat=1.9,
    %log ticks with fixed point, % no scientific notation in plots
    table/col sep=tab, % only tabs are column separators
    unbounded coords=jump, % better have skips in a plot than appear to be interpolating
    filter discard warning=false, % Don't complain about empty cells
}
\SendSettingsToPgf % use siunitx formatting settings in PGF, too

\begin{document}

\begin{tikzpicture}
    \begin{semilogyaxis}[
            % Which column to be taken from each file
            /pgfplots/table/y index=5,
            /pgfplots/table/header=false,
            % axis labels
            xlabel=n-th smallest result,
            ylabel=Memory (B),
            % axis ranges
            xmin=0,
            ymin=100000,
            ymax=8000000000,
            mark repeat=100,
            % legend
            legend entries={\dcssat,\ressatb,\ressat},
            every axis legend/.append style={at={(1,0)}, anchor=south east, outer xsep=5pt, outer ysep=5pt,},
        ]
        \addplot+ table {random-exist-ssat/evaluation/csv/dcssat.default.Random.quantile.memory.csv};
        \addplot+ table {random-exist-ssat/evaluation/csv/ressat.bare-Cachet.Random.quantile.memory.csv};
        \addplot+ table {random-exist-ssat/evaluation/csv/ressat.minimize-Cachet.Random.quantile.memory.csv};
    \end{semilogyaxis}
\end{tikzpicture}

\end{document}

        \label{fig:erssat-quantile-memory-random}
    }
    \caption{Quantile plots of random $k$-CNF formulas}
    \label{fig:erssat-quantile-random}
\end{figure*}

The quantile plot is shown in~\cref{fig:erssat-quantile-random}.

\begin{figure*}[ht]
    \centering
    \subfloat[CPU time]{
        % This file is part of BenchExec, a framework for reliable benchmarking:
% https://github.com/sosy-lab/benchexec
%
% SPDX-FileCopyrightText: 2007-2020 Dirk Beyer <https://www.sosy-lab.org>
%
% SPDX-License-Identifier: Apache-2.0

% LaTeX code for a quantile plot
% Copy the tikzpicture environment to your own document
% and make sure the siunitx and pgfplots package are loaded,
% possibly with the options suggested here.
\documentclass{standalone}
\usepackage[
    group-digits=integer, group-minimum-digits=4, % group digits by thousands
    free-standing-units, unit-optional-argument, % easier input of numbers with units
]{siunitx}
\usepackage{pgfplots}
\pgfplotsset{
    compat=1.9,
    log ticks with fixed point, % no scientific notation in plots
    table/col sep=tab, % only tabs are column separators
    unbounded coords=jump, % better have skips in a plot than appear to be interpolating
    filter discard warning=false, % Don't complain about empty cells
}
\SendSettingsToPgf % use siunitx formatting settings in PGF, too

\begin{document}

\begin{tikzpicture}
    \begin{semilogyaxis}[
            % Which column to be taken from each file
            /pgfplots/table/y index=3,
            /pgfplots/table/header=false,
            % axis labels
            xlabel=n-th fastest result,
            ylabel=Time (\second),
            % axis ranges
            xmin=0,
            ymin=0.001,
            ymax=1000,
            mark repeat=100,
            % legend
            legend entries={\dcssat,\erssatb,\erssat},
            every axis legend/.append style={at={(1,0)}, anchor=south east, outer xsep=5pt, outer ysep=5pt,},
        ]
        \addplot+ table {exist-random-ssat/evaluation/csv/dcssat.default.Application.quantile.cputime.csv};
        \addplot+ table {exist-random-ssat/evaluation/csv/erssat.bare-BDD.Application.quantile.cputime.csv};
        \addplot+ table {exist-random-ssat/evaluation/csv/erssat.default-BDD.Application.quantile.cputime.csv};
    \end{semilogyaxis}
\end{tikzpicture}

\end{document}

        \label{fig:erssat-quantile-cputime-application}
    }\\
    \subfloat[Memory usage]{
        % This file is part of BenchExec, a framework for reliable benchmarking:
% https://github.com/sosy-lab/benchexec
%
% SPDX-FileCopyrightText: 2007-2020 Dirk Beyer <https://www.sosy-lab.org>
%
% SPDX-License-Identifier: Apache-2.0

% LaTeX code for a quantile plot
% Copy the tikzpicture environment to your own document
% and make sure the siunitx and pgfplots package are loaded,
% possibly with the options suggested here.
\documentclass{standalone}
\usepackage[
    group-digits=integer, group-minimum-digits=4, % group digits by thousands
    free-standing-units, unit-optional-argument, % easier input of numbers with units
]{siunitx}
\usepackage{pgfplots}
\pgfplotsset{
    compat=1.9,
    %log ticks with fixed point, % no scientific notation in plots
    table/col sep=tab, % only tabs are column separators
    unbounded coords=jump, % better have skips in a plot than appear to be interpolating
    filter discard warning=false, % Don't complain about empty cells
}
\SendSettingsToPgf % use siunitx formatting settings in PGF, too

\begin{document}

\begin{tikzpicture}
    \begin{semilogyaxis}[
            % Which column to be taken from each file
            /pgfplots/table/y index=5,
            /pgfplots/table/header=false,
            % axis labels
            xlabel=n-th smallest result,
            ylabel=Memory (B),
            % axis ranges
            xmin=0,
            ymin=100000,
            ymax=8000000000,
            mark repeat=100,
            % legend
            legend entries={\dcssat,\erssatb,\erssat},
            every axis legend/.append style={at={(1,0)}, anchor=south east, outer xsep=5pt, outer ysep=5pt,},
        ]
        \addplot+ table {exist-random-ssat/evaluation/csv/dcssat.default.Application.quantile.memory.csv};
        \addplot+ table {exist-random-ssat/evaluation/csv/erssat.bare-BDD.Application.quantile.memory.csv};
        \addplot+ table {exist-random-ssat/evaluation/csv/erssat.default-BDD.Application.quantile.memory.csv};
    \end{semilogyaxis}
\end{tikzpicture}

\end{document}

        \label{fig:erssat-quantile-memory-application}
    }
    \caption{Quantile plots of application formulas}
    \label{fig:erssat-quantile-application}
\end{figure*}

The quantile plot is shown in~\cref{fig:erssat-quantile-application}.

\begin{figure*}[ht]
    \centering
    \subfloat[\erssatb]{
        % This file is part of BenchExec, a framework for reliable benchmarking:
% https://github.com/sosy-lab/benchexec
%
% SPDX-FileCopyrightText: 2007-2020 Dirk Beyer <https://www.sosy-lab.org>
%
% SPDX-License-Identifier: Apache-2.0

% LaTeX code for a scatter plot.
% Copy the tikzpicture environment to your own document
% and make sure the siunitx and pgfplots package are loaded,
% possibly with the options suggested here.
\documentclass{standalone}

%%% Use option "handout" to disable pause/action
\documentclass{beamer}
%\documentclass[handout]{beamer}
\usepackage[utf8]{inputenc}
\usepackage{utopia}
\usetheme{Madrid}
\usecolortheme{default}

% packages
\usepackage{colortbl}
\usepackage{etoolbox}
\usepackage[square,numbers,sort&compress]{natbib}
\usepackage{algorithm}
\usepackage{algorithmic}
\usepackage{booktabs}
\usepackage{rotating}
\usepackage{adjustbox}
\usepackage{xspace}
\usepackage{tcolorbox}
\usepackage{subfig}
\usepackage{hyperref}
\usepackage[capitalise,nosort,nameinlink]{cleveref}
\usepackage{tikz}
\usepackage{tikz-qtree}
\usepackage{circuitikz}
\usepackage{pgfplotstable}
\usetikzlibrary{positioning,arrows}
% Plots with BenchExec
\usepackage{standalone}
\usepackage[
      group-digits=integer, group-minimum-digits=4, % group digits by thousands
      free-standing-units, unit-optional-argument, % easier input of numbers with units
]{siunitx}
\usepackage{pgfplots}
\pgfplotsset{
      compat=1.9,
      %log ticks with fixed point, % no scientific notation in plots
      table/col sep=tab, % only tabs are column separators
      unbounded coords=jump, % better have skips in a plot than appear to be interpolating
      filter discard warning=false, % Don't complain about empty cells
}
\SendSettingsToPgf % use siunitx formatting settings in PGF, too
% end packages

% symbols
\newcommand{\booldom}{\mathbb{B}} % Boolean domain
\newcommand{\limply}{\rightarrow} % logical implication
\newcommand{\vl}[1]{\texttt{var}(#1)} % variable of a literal
\newcommand{\as}{\tau} % an assignment
\newcommand{\av}[1]{\mathcal{A}(#1)} % all assignments over a variable set
\newcommand{\pf}{\phi} % propositional formula (quantifier-free)
\newcommand{\vf}[1]{\texttt{vars}(#1)} % variable of a formula
\newcommand{\pcf}[2]{#1|_{#2}} % positive cofactor
\newcommand{\ncf}[2]{#1|_{\lnot#2}} % negative cofactor
\newcommand{\Qf}{\Phi} % quantified formula
\newcommand{\base}{X_d} % a base set of a formula
\newcommand{\random}[1]{\rotatebox[origin=c]{180}{$\mathsf{R}$}^{#1}} % randomized quantifier
\newcommand{\spb}[1]{\Pr[#1]} % satisfying probability
\newcommand{\wt}{\omega} % weight of a variable
\newcommand{\sat}[1]{\texttt{SAT}(#1)} % formula is satisfiable
\newcommand{\unsat}[1]{\texttt{UNSAT}(#1)} % formula is unsatisfiable
\newcommand{\model}[1]{#1.\mathrm{model}} % formula is unsatisfiable
\newcommand{\select}{\psi} % selection formula
\newcommand{\cx}{C^X} % sub-clause of X variables
\newcommand{\cy}{C^Y} % sub-clause of Y variables
\newcommand{\sv}[1]{s_{#1}} % selection variable
\newcommand{\dep}[1]{D_{#1}} % dependency set
\newcommand{\uvs}{V_{\Qf}^{\forall}} % universal variable set
\newcommand{\evs}{V_{\Qf}^{\exists}} % existential variable set
\newcommand{\rvs}{V_{\Qf}^{\random{}}} % random variable set
\newcommand{\skf}{\mathcal{F}} % Skolem function set
\newcommand{\nodeval}[1]{#1.\mathrm{value}} % BDD node value
\newcommand{\nodevar}[1]{#1.\mathrm{var}} % BDD node variable
\newcommand{\nodevisit}[1]{#1.\mathrm{visited}} % BDD node visited flag
\newcommand{\nodethen}[1]{#1.\mathrm{then}} % BDD node then
\newcommand{\nodeelse}[1]{#1.\mathrm{else}} % BDD node else
\newcommand{\nodesp}[1]{#1.\mathrm{sp}} % BDD node satisfying probability
\newcommand{\er}{\epsilon} % error rate of a gate
\newcommand{\dr}{\delta} % defect rate of a design
\DeclareMathOperator*{\argmax}{arg\,max} % maximizing argument
\DeclareMathOperator*{\argmin}{arg\,min} % minimizing argument

% tools
\newcommand{\tool}[1]{\texttt{#1}\xspace}
\newcommand{\definetool}[2]{\newcommand{#1}{\tool{#2}}\xspace}
\definetool{\ressat}{reSSAT}
\definetool{\ressatb}{reSSAT-b}
\definetool{\erssat}{erSSAT}
\definetool{\erssatb}{erSSAT-b}
\definetool{\bddsp}{BDDsp}
\definetool{\bddspnr}{BDDsp-nr}
\definetool{\maxplan}{MAXPLAN}
\definetool{\zander}{ZANDER}
\definetool{\dcssat}{DC-SSAT}
\definetool{\complan}{ComPlan}
\definetool{\minisat}{MiniSat}
\definetool{\cachet}{Cachet}
\definetool{\approxmc}{ApproxMC}
\definetool{\ctwod}{c2d}
\definetool{\dpmc}{DPMC}
\definetool{\procount}{ProCount}
\definetool{\cnfgen}{CNFgen}
\definetool{\benchexec}{BenchExec}
\definetool{\abc}{ABC}
\definetool{\cudd}{CUDD}
\definetool{\prism}{PRISM}
\definetool{\timeout}{TO}
\definetool{\memout}{MO}

% URL
\newcommand{\ssatabcurl}{https://github.com/NTU-ALComLab/ssatABC}
\newcommand{\ssatbenchmarkurl}{https://github.com/NTU-ALComLab/ssat-benchmarks}
\newcommand{\benchexecurl}{https://github.com/sosy-lab/benchexec}

% Evaluation setup, environment, and version
\newcommand{\machineSpec}{one 2.2\,GHz CPU (Intel Xeon Silver 4210) with 40~processing units and \num{134616}\,MB of RAM}
\newcommand{\osInfo}{Ubuntu~20.04 (64~bit), running Linux~5.4}
\newcommand{\compiler}{\texttt{g++ 9.3.0}}
\newcommand{\timelimit}{\SI{15}{min}}
\newcommand{\memlimit}{\SI{15}{GB}}
\newcommand{\ssatABCRevision}{commit \texttt{2ff8e74} of branch \texttt{master}\xspace}
\newcommand{\ssatBenchRevision}{commit \texttt{ea9fbae} of branch \texttt{master}\xspace}

% constant words
\newcommand{\word}[1]{\textsc{#1}\xspace}
\newcommand{\defineword}[2]{\newcommand{#1}{\word{#2}}\xspace}
\defineword{\true}{true}
\defineword{\false}{false}
\defineword{\disjoin}{or}
\defineword{\conjoin}{and}
\defineword{\nand}{nand}
\defineword{\xor}{xor}

% customize package "amsthm"
\renewcommand\qedsymbol{$\blacksquare$}

% customize package "cleveref"
\newcommand{\creflastconjunction}{, and~}
\crefname{equation}{Eq.}{Eqs.}
\crefname{algorithm}{Alg.}{Algs.}
\crefname{line}{line}{lines}
\crefalias{ALC@unique}{line}
\crefalias{ALC@line}{line}

% customize package "algorithmic"
\renewcommand{\algorithmicrequire}{\textbf{Input:}}
\renewcommand{\algorithmicensure}{\textbf{Output:}}

% customize package "pgfplotstable"
\pgfplotstableset{
      circuit column/.style={
                  /pgfplots/table/display columns/#1/.style={
                              string type,column type=l,column name=\textsc{Circuit}
                        }
            },
      formula column/.style={
                  /pgfplots/table/display columns/#1/.style={
                              string type,column type=l,column name=\textsc{Formula}
                        }
            },
      pi column/.style={
                  /pgfplots/table/display columns/#1/.style={fixed,column type=r,column name=\#PI}
            },
      ai column/.style={
                  /pgfplots/table/display columns/#1/.style={fixed,column type=r,column name=\#AI}
            },
      po column/.style={
                  /pgfplots/table/display columns/#1/.style={fixed,column type=r,column name=\#PO}
            },
      and column/.style={
                  /pgfplots/table/display columns/#1/.style={fixed,column type=r,column name=\#And}
            },
      level column/.style={
                  /pgfplots/table/display columns/#1/.style={fixed,column type=r,column name=\#Level}
            },
      time column/.style={
                  /pgfplots/table/display columns/#1/.style={
                              string replace={nan}{},sci,sci zerofill,sci sep align,precision=2,sci e,column name=T (s)
                        }
            },
      prob column/.style={
                  /pgfplots/table/display columns/#1/.style={
                              string replace={nan}{},sci,sci zerofill,sci sep align,precision=2,sci e,column name=$\Pr$
                        }
            },
      ubound column/.style={
                  /pgfplots/table/display columns/#1/.style={
                              string replace={nan}{},sci,sci zerofill,sci sep align,precision=2,sci e,column name=UB
                        }
            },
      lbound column/.style={
                  /pgfplots/table/display columns/#1/.style={
                              string replace={nan}{},sci,sci zerofill,sci sep align,precision=2,sci e,column name=LB
                        }
            },
      ubtime column/.style={
                  /pgfplots/table/display columns/#1/.style={
                              string replace={nan}{},sci,sci zerofill,sci sep align,precision=2,sci e,column name=T-UB (s)
                        }
            },
      lbtime column/.style={
                  /pgfplots/table/display columns/#1/.style={
                              string replace={nan}{},sci,sci zerofill,sci sep align,precision=2,sci e,column name=T-LB (s)
                        }
            }
}

% Fix line counters if multiple algorithms exist
\makeatletter % Make '@' a normal letter so that it can be used in *.tex files
\@addtoreset{ALC@line}{algorithm}
\@addtoreset{ALC@unique}{algorithm}
\makeatother % Undo the change to '@'

% Title page information
\title[Stochastic Boolean Satisfiability]{Stochastic Boolean Satisfiability}
\subtitle{Decision Procedures, Generalization, and Applications}

\author[Nian-Ze Lee]{Nian-Ze Lee\\~\\{\footnotesize Advisor: Prof. Jie-Hong Roland Jiang}}

\institute[NTU GIEE]{Graduate Institute of Electronics Engineering, National Taiwan University}

\date[Oral Defense, Jun. 2021]{Doctoral Dissertation Oral Defense, 2nd June 2021}

\setbeamerfont{section in toc}{size=\footnotesize}
\setbeamerfont{subsection in toc}{size=\scriptsize}

\makeatletter
\patchcmd{\beamer@sectionintoc}{\vskip1.5em}{\vskip0.5em}{}{}
\makeatother

\AtBeginSection[]
{
      \begin{frame}
            \frametitle{Outline}
            \tableofcontents[currentsection,hideallsubsections]
      \end{frame}
}

\AtBeginSubsection[]
{
      \begin{frame}
            \frametitle{Outline}
            \tableofcontents[currentsection,currentsubsection,subsectionstyle=show/shaded/hide]
      \end{frame}
}

\begin{document}

\begin{tikzpicture}
\begin{loglogaxis}[
    % axis labels
    %xlabel=CPU time of erssat-b (\second),
    %ylabel=CPU time of erssat (\second),
    % axis ranges
    xmin=0.01,
    xmax=1000,
    ymin=0.01,
    ymax=1000,
    domain=0.01:1001,
    %
    clip mode=individual,
    axis equal image,
    ]
    \addplot+[blue, mark=+,only marks]
         table[
             header=false,
             skip first n=3, % ignore CSV table header produced by table-generator
             x index=2, % index of x column
             y index=1  % index of y column
             ] {../csv/erssat.scatter.table.csv};
    \addplot[gray] {x};
    \addplot[gray] {10*x};
    \addplot[gray] {x/10};
\end{loglogaxis}
\end{tikzpicture}

\end{document}

        \label{fig:erssat-scatter-cputime-application}
    }\\
    \subfloat[\dcssat]{
        % This file is part of BenchExec, a framework for reliable benchmarking:
% https://github.com/sosy-lab/benchexec
%
% SPDX-FileCopyrightText: 2007-2020 Dirk Beyer <https://www.sosy-lab.org>
%
% SPDX-License-Identifier: Apache-2.0

% LaTeX code for a scatter plot.
% Copy the tikzpicture environment to your own document
% and make sure the siunitx and pgfplots package are loaded,
% possibly with the options suggested here.
\documentclass{standalone}
\usepackage[
    group-digits=integer, group-minimum-digits=4, % group digits by thousands
    free-standing-units, unit-optional-argument, % easier input of numbers with units
    ]{siunitx}
\usepackage{pgfplots}
\pgfplotsset{
    compat=1.9,
    log ticks with fixed point, % no scientific notation in plots
    table/col sep=tab, % only tabs are column separators
    unbounded coords=jump, % better have skips in a plot than appear to be interpolating
    filter discard warning=false, % Don't complain about empty cells
    }
\SendSettingsToPgf % use siunitx formatting settings in PGF, too

\begin{document}

\begin{tikzpicture}
\begin{loglogaxis}[
    % axis labels
    %xlabel=CPU time of dcssat (\second),
    %ylabel=CPU time of erssat (\second),
    % axis ranges
    xmin=0.001,
    xmax=1000,
    ymin=0.001,
    ymax=1000,
    domain=0.001:1001,
    %
    clip mode=individual,
    axis equal image,
    ]
    \addplot+[blue, mark=+,only marks]
         table[
             header=false,
             skip first n=3, % ignore CSV table header produced by table-generator
             x index=2, % index of x column
             y index=1  % index of y column
             ] {exist-random-ssat/evaluation/csv/dcssat.scatter.table.csv};
    \addplot[gray] {x};
    \addplot[gray] {10*x};
    \addplot[gray] {x/10};
\end{loglogaxis}
\end{tikzpicture}

\end{document}

        \label{fig:dcssat-scatter-cputime-application}
    }
    \caption{CPU-time scatter plots of application formulas with \erssat in y-axis and compared approaches in x-axis}
    \label{fig:erssat-scatter-application}
\end{figure*}

The scatter plot is shown in~\cref{fig:erssat-scatter-application}.