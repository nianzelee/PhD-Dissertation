\chapter{Conclusion and Future Work}
\label{chap:conclusion-future-work}

In this dissertation,
the research needs highlighted in~\cref{chap:introduction} have been addressed.
Specifically,
the applicability of SSAT to the analysis of VLSI systems was examined.
We formulated a framework for the property evaluation of probabilistic design,
and encoded the average-case and the worst-case analyses
as random-exist and exist-random quantified SSAT formulas, respectively.

Motivated by the emerging VLSI applications,
we further designed novel algorithms for random-exist and exist-random quantified SSAT formulas.
The proposed algorithms aim at leveraging the success from SAT/QBF-solving and model-counting communities,
so as to advance the state-of-the-art of SSAT solving beyond the conventional DPLL-based search.
For random-exist quantified SSAT formulas,
we used minterm generalization and weighted model counting as subroutines,
and employed SAT solvers and weighted model counters as plug-in engines.
For exist-random quantified SSAT formulas,
we devised the clause-containment learning,
which was inspired by the clause-selection technique of QBF.
Under the framework of clause-containment learning,
three enhancement techniques were proposed to improve the computational efficiency.
Moreover, unlike previous exact SSAT methods,
the proposed algorithms can solve approximate SSAT by deriving upper and lower bounds of satisfying probability.
Experiment results showed the benefits of our solvers over a wide range of benchmarking instances.
Our implementations and the benchmark suite of SSAT instances are open-source
for other researchers to base their work on top of our results.

To generalize SSAT beyond the PSPACE-complete complexity class for more complex problems,
we extended DQBF to its stochastic variant DSSAT and proved its NEXPTIME-completeness.
Compared to the PSPACE-complete SSAT,
DSSAT is more powerful to succinctly model NEXPTIME-complete decision problems with uncertainty.
We demonstrated the DSSAT formulation of the analysis to probabilistic/approximate partial design,
and gave a polynomial-time reduction from the NEXPTIME-complete Dec-POMDP to DSSAT.

For future work,
we plan to incorporate Monte-Carlo simulation into our framework of probabilistic property evaluation,
and develop solvers for arbitrary quantified SSAT and DSSAT formulas.
We note that the clause-containment learning has been generalized to solve general SSAT~\cite{Chen2021},
and the recent developments of DQBF~\cite{Tentrup2019} might provide a promising framework for DSSAT solving.

\iffalse
    We have proposed a formal framework to probabilistic property
    evaluation, under the worst-case and average-case scenarios.
    Connections between probabilistic property evaluation and existing
    solving techniques have been established. A novel BDD-based SSAT solver is proposed. A comparative experimental study has been performed to assess the capabilities
    of different methods. Among the considered solutions, the proposed BDD-based SSAT solver, which makes use of circuit structures to construct BDD, currently tends to be the most robust in our experiments. Nevertheless, there are cases solvable only by approximate weighted model counting, but not by other methods. As the BDD-based method has its memory explosion problem, SSAT and model
    counting approaches based on CNF formula might be more viable than the BDD one if their efficiency would be improved in the future. Our results may benefit the synthesis of probabilistic design, perhaps not only for silicon but also for genetic circuits, which are intrinsically stochastic. For future investigation, Monte-Carlo simulation may be incorporated to our proposed formal methods.

    In this paper, we focused on solving random-exist quantified SSAT formulas.
    In contrast to the previous DPLL-based algorithms, we proposed a novel algorithm using SAT solver and weighted model counter as underlying engines to improve computational efficiency.
    Leveraging the great success of modern SAT solving techniques, the proposed algorithm outperforms the state-of-the-art method in the experiment on random $k$-CNF and strategic companies formulas.
    Moreover, unlike previous exact SSAT methods, the proposed algorithm can be easily modified to solve approximate SSAT by deriving upper and lower bounds of satisfying probability.
    We demonstrated the applicability of our SSAT solver to VLSI circuit analysis.
    While the state-of-the-art solver fails to compute the exact satisfying probability, the proposed method succeeded in finding bounds of the formulas.
    In several cases, the derived bounds are very close to, or even match the exact satisfying probability.
    This approximation flexibility of our method can be helpful when SSAT is applied to real-world applications.
    %This work might shed light on the application of solving approximate SSAT to real-world problems.
    For future work, we intend to extend the proposed algorithm to arbitrary quantified SSAT formulas.

    We developed a new approach to solving E-MAJSAT formulas. In contrast to prior methods based on DPLL search or knowledge compilation, we proposed the clause containment learning technique, inspired by clause selection recently developed in QBF evaluation, and design a novel algorithm to solve E-MAJSAT efficiently. Under the framework of clause containment learning, three enhancement techniques were proposed to improve the computational efficiency.
    Experiment results show the benefit of our method.
    %that our method achieves significant performance gains and memory savings over prior SSAT methods, and also provides useful lower bound information for cases where no information can be given by prior methods.
    For future work, we intend to solve SSAT with general prefix structure.

    In this paper, we extended DQBF to its stochastic variant DSSAT and proved its NEXPTIME-completeness.
    Compared to the PSPACE-complete SSAT, DSSAT is more powerful to succinctly model NEXPTIME-complete decision problems with uncertainty.
    The new formalism can be useful in applications such as artificial intelligence and system design.
    Specifically, we demonstrated the DSSAT formulation of the analysis to probabilistic/approximate partial design, and gave a polynomial-time reduction from the NEXPTIME-complete Dec-POMDP to DSSAT.
    We envisage the potential broad applications of DSSAT and plan solver development for future work.
    %\textcolor{blue}{
    We note that recent developments of \textit{clausal abstraction} for QBF~\cite{JanotaM15,RabeT15} and DQBF~\cite{Tentrup19} might provide a promising framework for DSSAT solving.
    Clausal abstraction has been lifted to SSAT~\cite{ChenHJ21}, and we are investigating its feasibility for DSSAT.
    %}
\fi