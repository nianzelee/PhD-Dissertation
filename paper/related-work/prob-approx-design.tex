\section{Probabilistic/Approximate design}
\label{sect:related-work-prob-approx}

While the shrinkage of device's feature size according to Moore's law~\cite{Moore1965}
has driven the prosperity of microelectronic industry,
the variability and uncertainty of devices at the atomic level
pose serious challenges on the continuation of the law.

New computational paradigms are proposed in response to the challenges in the post-Moore's era.
Among many other efforts,
\textit{approximate design} allows deterministic deviation of an implemented circuit from its specification.
A real-world example of approximate design is the deployment of neural networks to edge devices.
Quantization is often applied to convert floating-point parameters to fixed-point ones to reduce hardware cost.
Along this direction, circuit architectures for reconfigurable adders whose accuracy can be adjusted based on application scenarios~\cite{Kahng2012,Ye2013} and energy-efficient adders with a moderate error rate~\cite{Kim2013} are proposed.
The automatic synthesis and analysis of approximate circuits have also been studied~\cite{Venkatesan2011ApproxDesign,Venkataramani2012,Miao2013,Miao2014,Li2014,Mrazek2016,Rehman2016}.
On the other hand, \textit{probabilistic design} allows nondeterministic deviation from the specification.
It can be applied to, for example, low-power video decoding, where correctness can be sacrificed for power efficiency.
Chakrapani et al.~\cite{Chakrapani2006ProbDesign} consider CMOS devices with probabilistic behavior,
establish the relation between energy consumption and probability of correct switching,
and exploit it to trade power efficiency against correctness.

Despite the advancements made by prior endeavors, most of them focus on approximate design.
The analysis and synthesis of probabilistic design have gained relatively less attention.
Nevertheless, the study of probabilistic design is important in the following respects.
First, randomness is a valuable resource to trade for computation efficiency.
For example, there exist problems that can be solved efficiently by randomized algorithms
but not deterministic algorithms.
Second, there are applications, such as data mining, compressive sensing, etc.,
that may not be sensitive to minor random fluctuations.
Third, devices at their quantum foundation or systems in their biological nature are intrinsically probabilistic.
We aim at providing a formalism for the analysis and verification of probabilistic design.
Notice that in the design automation process,
verification is essential to the entire design flow.
For example, equivalence checking~\cite{Kuehlmann1997,Mishchenko2006}
should be applied to validate the correctness of synthesized circuits.
Hence, establishing a verification framework is a crucial step in automated synthesis of probabilistic design.

Probabilistic behavior of a design has also been studied along the research of circuit reliability analysis,
which focuses on analyzing the robustness of a circuit against permanent defects or transient faults.
The reliability of a circuit is characterized by the probability of the occurrence of an error at the primary outputs.
Therefore, the study of circuit reliability is very related to the evaluation of probabilistic design.
Classical approaches to reliability analysis apply fault injection and Monte Carlo simulation~\cite{Mohanram2003}.
Symbolic analysis methods exploiting mathematical tools,
such as Markov random fields~\cite{Bahar2003},
probability transfer matrices~\cite{Krishnaswamy2005},
Bayesian networks~\cite{Rejimon2005},
and algebraic decision diagrams~\cite{Miskov-Zivanov2006},
have also been investigated.
Choudhury and Mohanram~\cite{Choudhury2009} propose three accurate and scalable algorithms to address the scalability issue of symbolic methods.

However, prior methods for circuit reliability analysis are inadequate to probabilistic circuits
for the following two reasons.
First, most prior approaches assume single-gate failures.
This assumption makes prior methods inapplicable to probabilistic design,
where multiple probabilistic gates may be commonly present.
Second, most prior efforts consider only the average error rate of a design.
The necessity of analyzing the maximum error rate comes from the increasing demand of safety-centric systems,
e.g., utilized in health care or automotive industries~\cite{Lingasubramanian2007,Lingasubramanian2011}.
Unfortunately, its computational scalability is much limited due to the underlying symbolic modeling via Bayesian network~\cite{Jensen1996}.