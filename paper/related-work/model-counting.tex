\section{Model counting}
\label{sect:related-work-model-counting}

Model-counting~\cite{SATHandbook-ModelCounting} algorithms can be classified into two categories:
exact counting and approximate counting.
The former adopts DPLL-based search with additional techniques,
such as component analysis and caching, to improve the counting efficiency~\cite{Sang2004,Sang2005ModelCounting}.
Knowledge compilation has also been applied to exact model counting.
For example, \ctwod~\cite{Darwiche2001,Darwiche2002dDNNF} compiles a CNF formula into its d-DNNF and performs counting on this data structure.
The latter takes a different strategy,
aiming at providing lower and/or upper bounds with guarantee on confidence level.
Ideas from statistics~\cite{Chakraborty2013,Chakraborty2016} have been adopted to increase the capacity limit of model counting.

\begin{table}[t]
    \centering
    \caption{Model-counting variants and their corresponding SSAT formulas}
    \label{tbl:related-work-model-counting}
    \begin{tabular}{c|c}
        Model-counting variant & SSAT encoding                                                                                              \\
        \hline
        Unweighted             & $\random{0.5}x_1,\ldots,\random{0.5}x_n.\pf(x_1,\ldots,x_n)$                                               \\
        Weighted               & $\random{p_1}x_1,\ldots,\random{p_n}x_n.\pf(x_1,\ldots,x_n)$                                               \\
        Projected              & $\random{0.5}x_1,\ldots,\random{0.5}x_n,\exists y_1,\ldots,\exists y_m.\pf(x_1,\ldots,x_n,y_1,\ldots,y_m)$ \\
        Maximum                & $\exists x_1,\ldots,\exists x_n,\random{0.5}y_1,\ldots,\random{0.5}y_m.\pf(x_1,\ldots,x_n,y_1,\ldots,y_m)$ \\
        Weighted projected     & $\random{p_1}x_1,\ldots,\random{p_n}x_n,\exists y_1,\ldots,\exists y_m.\pf(x_1,\ldots,x_n,y_1,\ldots,y_m)$ \\
        Maximum weighted       & $\exists x_1,\ldots,\exists x_n,\random{p_1}y_1,\ldots,\random{p_m}y_m.\pf(x_1,\ldots,x_n,y_1,\ldots,y_m)$ \\
    \end{tabular}
\end{table}

There are many variants of model counting.
For example,
weighted model counting asks to aggregate the weight of every satisfying assignment.
It has been widely adopted in probabilistic inference~\cite{Sang2005BayesianInference,Chavira2008}.
\textit{Projected model counting}~\cite{Aziz2015} computes the numbers of satisfying assignments
projected on a subset of original variables.
\textit{Maximum model counting}~\cite{Fremont2017} finds an assignment to a subset of variables in a formula
such that the number of satisfying assignments of the residual formula cofactored with the assignment is maximized.

The above variants of model counting can be expressed via SSAT,
because the randomized quantifiers of SSAT essentially aggregate the results from different branches with weights.
\Cref{tbl:related-work-model-counting} shows the variants of model-counting problems and their respective SSAT encodings.
Note that weighted projected model counting and maximum weighted model counting are equivalent to
random-exist quantified SSAT and exist-random quantified SSAT, respectively.

Model counting is under active research.
Recent advancements include \dpmc~\cite{Dudek2020},
a dynamic-programming framework for exact weighted model counting based on \textit{project-join} trees.
\dpmc applies tree decomposition to the constraint graph of a CNF formula to obtain a project-join tree
and aggregates the weight of the formula with \textit{arithmetic decision diagrams} using dynamic programming.
\dpmc is further extended to a weighted projected model counter \procount~\cite{Dudek2021}
by requiring the projected variables to be placed on top of the non-projected variables in a project-join tree.
Other latest developments of model counting can be found in the report~\cite{MC-COMP2020} of the 2020 Model Counting Competition.