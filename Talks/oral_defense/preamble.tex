%%% Use option "handout" to disable pause/action
\documentclass{beamer}
%\documentclass[handout]{beamer}
\usepackage[utf8]{inputenc}
\usepackage{utopia}
\usetheme{Madrid}
\usecolortheme{default}

% packages
\usepackage{etoolbox}
\usepackage[square,numbers,sort&compress]{natbib}
\usepackage{algorithm}
\usepackage{algorithmic}
\usepackage{booktabs}
\usepackage{rotating}
\usepackage{adjustbox}
\usepackage{xspace}
\usepackage{tcolorbox}
\usepackage{subfig}
\usepackage{hyperref}
\usepackage[capitalise,nosort,nameinlink]{cleveref}
\usepackage{tikz}
\usepackage{tikz-qtree}
\usepackage{circuitikz}
\usepackage{pgfplotstable}
\usetikzlibrary{positioning,arrows}
% Plots with BenchExec
\usepackage{standalone}
\usepackage[
    group-digits=integer, group-minimum-digits=4, % group digits by thousands
    free-standing-units, unit-optional-argument, % easier input of numbers with units
]{siunitx}
\usepackage{pgfplots}
\pgfplotsset{
    compat=1.9,
    %log ticks with fixed point, % no scientific notation in plots
    table/col sep=tab, % only tabs are column separators
    unbounded coords=jump, % better have skips in a plot than appear to be interpolating
    filter discard warning=false, % Don't complain about empty cells
}
\SendSettingsToPgf % use siunitx formatting settings in PGF, too
% end packages

% symbols
\newcommand{\booldom}{\mathbb{B}} % Boolean domain
\newcommand{\limply}{\rightarrow} % logical implication
\newcommand{\vl}[1]{\texttt{var}(#1)} % variable of a literal
\newcommand{\as}{\tau} % an assignment
\newcommand{\av}[1]{\mathcal{A}(#1)} % all assignments over a variable set
\newcommand{\pf}{\phi} % propositional formula (quantifier-free)
\newcommand{\vf}[1]{\texttt{vars}(#1)} % variable of a formula
\newcommand{\pcf}[2]{#1|_{#2}} % positive cofactor
\newcommand{\ncf}[2]{#1|_{\lnot#2}} % negative cofactor
\newcommand{\Qf}{\Phi} % quantified formula
\newcommand{\base}{X_d} % a base set of a formula
\newcommand{\random}[1]{\rotatebox[origin=c]{180}{$\mathsf{R}$}^{#1}} % randomized quantifier
\newcommand{\spb}[1]{\Pr[#1]} % satisfying probability
\newcommand{\wt}{\omega} % weight of a variable
\newcommand{\sat}[1]{\texttt{SAT}(#1)} % formula is satisfiable
\newcommand{\unsat}[1]{\texttt{UNSAT}(#1)} % formula is unsatisfiable
\newcommand{\model}[1]{#1.\mathrm{model}} % formula is unsatisfiable
\newcommand{\select}{\psi} % selection formula
\newcommand{\cx}{C^X} % sub-clause of X variables
\newcommand{\cy}{C^Y} % sub-clause of Y variables
\newcommand{\sv}[1]{s_{#1}} % selection variable
\newcommand{\dep}[1]{D_{#1}} % dependency set
\newcommand{\uvs}{V_{\Qf}^{\forall}} % universal variable set
\newcommand{\evs}{V_{\Qf}^{\exists}} % existential variable set
\newcommand{\rvs}{V_{\Qf}^{\random{}}} % random variable set
\newcommand{\skf}{\mathcal{F}} % Skolem function set
\newcommand{\nodeval}[1]{#1.\mathrm{value}} % BDD node value
\newcommand{\nodevar}[1]{#1.\mathrm{var}} % BDD node variable
\newcommand{\nodevisit}[1]{#1.\mathrm{visited}} % BDD node visited flag
\newcommand{\nodethen}[1]{#1.\mathrm{then}} % BDD node then
\newcommand{\nodeelse}[1]{#1.\mathrm{else}} % BDD node else
\newcommand{\nodesp}[1]{#1.\mathrm{sp}} % BDD node satisfying probability
\newcommand{\er}{\epsilon} % error rate of a gate
\newcommand{\dr}{\delta} % defect rate of a design
\DeclareMathOperator*{\argmax}{arg\,max} % maximizing argument
\DeclareMathOperator*{\argmin}{arg\,min} % minimizing argument

% tools
\newcommand{\tool}[1]{\texttt{#1}\xspace}
\newcommand{\definetool}[2]{\newcommand{#1}{\tool{#2}}\xspace}
\definetool{\ressat}{reSSAT}
\definetool{\ressatb}{reSSAT-b}
\definetool{\erssat}{erSSAT}
\definetool{\erssatb}{erSSAT-b}
\definetool{\bddsp}{BDDsp}
\definetool{\bddspnr}{BDDsp-nr}
\definetool{\maxplan}{MAXPLAN}
\definetool{\zander}{ZANDER}
\definetool{\dcssat}{DC-SSAT}
\definetool{\complan}{ComPlan}
\definetool{\minisat}{MiniSat}
\definetool{\cachet}{Cachet}
\definetool{\approxmc}{ApproxMC}
\definetool{\cnfgen}{CNFgen}
\definetool{\benchexec}{BenchExec}
\definetool{\abc}{ABC}
\definetool{\cudd}{CUDD}
\definetool{\prism}{PRISM}
\definetool{\timeout}{TO}
\definetool{\memout}{MO}

% URL
\newcommand{\ssatabcurl}{https://github.com/NTU-ALComLab/ssatABC}
\newcommand{\ssatbenchmarkurl}{https://github.com/NTU-ALComLab/ssat-benchmarks}
\newcommand{\benchexecurl}{https://github.com/sosy-lab/benchexec}

% Evaluation setup, environment, and version
\newcommand{\machineSpec}{one 2.2\,GHz CPU (Intel Xeon Silver 4210) with 40~processing units and \num{134616}\,MB of RAM}
\newcommand{\osInfo}{Ubuntu~20.04 (64~bit), running Linux~5.4}
\newcommand{\compiler}{\texttt{g++ 9.3.0}}
\newcommand{\timelimit}{\SI{15}{min}}
\newcommand{\memlimit}{\SI{15}{GB}}
\newcommand{\ssatABCRevision}{commit \texttt{2ff8e74} of branch \texttt{master}\xspace}
\newcommand{\ssatBenchRevision}{commit \texttt{ea9fbae} of branch \texttt{master}\xspace}

% constant words
\newcommand{\word}[1]{\textsc{#1}\xspace}
\newcommand{\defineword}[2]{\newcommand{#1}{\word{#2}}\xspace}
\defineword{\true}{true}
\defineword{\false}{false}
\defineword{\disjoin}{or}
\defineword{\conjoin}{and}
\defineword{\nand}{nand}
\defineword{\xor}{xor}

% customize package "amsthm"
\renewcommand\qedsymbol{$\blacksquare$}

% customize package "cleveref"
\newcommand{\creflastconjunction}{, and~}
\crefname{equation}{Eq.}{Eqs.}
\crefname{algorithm}{Alg.}{Algs.}
\crefname{line}{line}{lines}
\crefalias{ALC@unique}{line}
\crefalias{ALC@line}{line}

% customize package "algorithmic"
\renewcommand{\algorithmicrequire}{\textbf{Input:}}
\renewcommand{\algorithmicensure}{\textbf{Output:}}

% customize package "pgfplotstable"
\pgfplotstableset{
    circuit column/.style={
            /pgfplots/table/display columns/#1/.style={
                    string type,column type=l,column name=\textsc{Circuit}
                }
        },
    formula column/.style={
            /pgfplots/table/display columns/#1/.style={
                    string type,column type=l,column name=\textsc{Formula}
                }
        },
    pi column/.style={
            /pgfplots/table/display columns/#1/.style={fixed,column type=r,column name=\#PI}
        },
    ai column/.style={
            /pgfplots/table/display columns/#1/.style={fixed,column type=r,column name=\#AI}
        },
    po column/.style={
            /pgfplots/table/display columns/#1/.style={fixed,column type=r,column name=\#PO}
        },
    and column/.style={
            /pgfplots/table/display columns/#1/.style={fixed,column type=r,column name=\#And}
        },
    level column/.style={
            /pgfplots/table/display columns/#1/.style={fixed,column type=r,column name=\#Level}
        },
    time column/.style={
            /pgfplots/table/display columns/#1/.style={
                    string replace={nan}{},fixed,fixed zerofill,dec sep align,precision=2,column name=T (s)
                }
        },
    prob column/.style={
            /pgfplots/table/display columns/#1/.style={
                    string replace={nan}{},sci,sci zerofill,sci sep align,precision=2,sci e,column name=$\Pr$
                }
        },
    ubound column/.style={
            /pgfplots/table/display columns/#1/.style={
                    string replace={nan}{},sci,sci zerofill,sci sep align,precision=2,sci e,column name=UB
                }
        },
    lbound column/.style={
            /pgfplots/table/display columns/#1/.style={
                    string replace={nan}{},sci,sci zerofill,sci sep align,precision=2,sci e,column name=LB
                }
        },
    ubtime column/.style={
            /pgfplots/table/display columns/#1/.style={
                    string replace={nan}{},fixed,fixed zerofill,dec sep align,precision=2,column name=T-UB (s)
                }
        },
    lbtime column/.style={
            /pgfplots/table/display columns/#1/.style={
                    string replace={nan}{},fixed,fixed zerofill,dec sep align,precision=2,column name=T-LB (s)
                }
        }
}

% Fix line counters if multiple algorithms exist
\makeatletter % Make '@' a normal letter so that it can be used in *.tex files
\@addtoreset{ALC@line}{algorithm}
\@addtoreset{ALC@unique}{algorithm}
\makeatother % Undo the change to '@'

% Title page information
\title[Stochastic Boolean Satisfiability]{Stochastic Boolean Satisfiability}
\subtitle{Decision Procedure, Generalization, and Applications}

\author[Nian-Ze Lee]{Nian-Ze Lee}

\institute[NTU GIEE]{Graduate Institute of Electronics Engineering, National Taiwan University}

\date[Oral Defense, Jun. 2021]{Doctoral Dissertation Oral Defense, 2nd June 2021}

\setbeamerfont{section in toc}{size=\footnotesize}
\setbeamerfont{subsection in toc}{size=\scriptsize}

\makeatletter
\patchcmd{\beamer@sectionintoc}{\vskip1.5em}{\vskip0.5em}{}{}
\makeatother

\AtBeginSection[]
{
    \begin{frame}
        \frametitle{Outline}
        \tableofcontents[currentsection,hideallsubsections]
    \end{frame}
}

\AtBeginSubsection[]
{
    \begin{frame}
        \frametitle{Outline}
        \tableofcontents[currentsection,currentsubsection,subsectionstyle=show/shaded/hide]
    \end{frame}
}